\chapter{Introduzione}

La presente tesi nasce dall’esperienza di tirocinio svolta presso un’azienda specializzata nello sviluppo di soluzioni per il filtraggio DNS. Durante tale esperienza, è stato possibile contribuire alle prime fasi di reingegnerizzazione del pannello Web per la configurazione di un filtro DNS. Questo pannello, concepito per offrire funzionalità di gestione e controllo dei domini da filtrare, presentava tuttavia limiti strutturali e tecnologici che ne ostacolavano la manutenibilità e l’estendibilità. L’esigenza di modernizzare l’architettura, migliorare l’esperienza utente e garantire maggiore sicurezza è nata anche dalla volontà di posizionare l’azienda in un mercato più orientato alle medie-grandi organizzazioni. Questo obiettivo mira a favorirne la crescita e a consolidarne il ruolo come punto di riferimento a livello globale nel settore del filtraggio DNS.

Il presente lavoro si inserisce in questo contesto, con l’obiettivo di progettare e sviluppare una nuova versione del pannello di configurazione del filtro DNS aziendale, adottando un’architettura moderna basata su microservizi. Questo approccio mira a garantire maggiore efficienza, sicurezza e manutenibilità rispetto alla versione precedente, superando le rigidità del legacy e introducendo nuove funzionalità essenziali.

Scopo dell’elaborato è fornire una visione organica del processo di reingegnerizzazione svolto, illustrando le motivazioni alla base della necessità di aggiornare il pannello, i principali obiettivi progettuali e i risultati raggiunti nelle prime fasi di sviluppo.

Il contesto teorico di riferimento comprende sia il Domain Name System (DNS), con le sue logiche di funzionamento e la sua importanza nella gestione del traffico in rete, sia le metodologie di reingegnerizzazione software, che offrono linee guida per affrontare in modo sistematico la modernizzazione di soluzioni esistenti.

Dal punto di vista metodologico, il progetto ha richiesto un'attenta fase di analisi, seguita dalla definizione di un’architettura più flessibile e scalabile, fino all’implementazione di un primo set di funzionalità.

Durante lo sviluppo, sono state affrontate diverse sfide, tra cui la transizione da un’architettura monolitica a una basata su microservizi, la riprogettazione della base dati e la creazione di un sistema avanzato di gestione degli errori, in grado di operare in modo strutturato e pervasivo su tutti i livelli del sistema.

\section{Struttura dell'elaborato}
L'elaborato è suddiviso nei seguenti capitoli, ciascuno dedicato a un aspetto specifico del progetto:

\begin{itemize}
  \item \textbf{Capitolo 2: Background} – Introduce il funzionamento del DNS e il filtraggio DNS, oltre alle metodologie di reingegnerizzazione software.
  \item \textbf{Capitolo 3: Analisi} – Esamina il sistema legacy, evidenziandone l’architettura, le tecnologie adottate e le limitazioni riscontrate. Inoltre, introduce i requisiti del nuovo sistema e la modellazione del dominio.
  \item \textbf{Capitolo 4: Design} – Presenta la nuova architettura proposta, sia lato frontend che backend, illustrando le interazioni tra i componenti e le principali scelte tecnologiche adottate.
  \item \textbf{Capitolo 5: Implementazione} – Descrive il processo di sviluppo del nuovo sistema, soffermandosi sull’infrastruttura, sull’organizzazione della repository e sull'implementazione del sistema di gestione degli errori.
  \item \textbf{Capitolo 6: Valutazione} – Analizza i requisiti soddisfatti e le funzionalità implementate, confrontandole con gli obiettivi iniziali. Inoltre, discute i miglioramenti pianificati e gli aspetti ancora da sviluppare.
  \item \textbf{Capitolo 7: Conclusione} – Riassume il lavoro svolto, evidenziando le principali lezioni apprese e proponendo possibili sviluppi futuri del sistema.
\end{itemize}

Grazie a questa suddivisione, l’elaborato mira a fornire una panoramica completa del percorso di reingegnerizzazione del sistema, dall’analisi dei requisiti iniziali fino alle valutazioni finali e ai suggerimenti per successivi sviluppi. L’obiettivo ultimo è offrire un contributo concreto alla modernizzazione del software, ponendo le basi per futuri interventi di mantenimento e potenziamento del sistema.
