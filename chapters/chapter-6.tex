\chapter{Valutazione}

L'obiettivo del presente capitolo è discutere i risultati finora ottenuti, in particolare sulla base dei requisiti già soddisfatti e delle migliorie introdotte dal nuovo sistema rispetto alla versione legacy. Poiché il sistema è ancora in fase di sviluppo, non è possibile fornire una valutazione complessiva, effettuare un confronto esaustivo con il vecchio sistema, né presentare risultati di test prestazionali e di usabilità. Tuttavia, è possibile analizzare le funzionalità già implementate e progettate, valutandone il contributo rispetto agli obiettivi iniziali e identificando le garanzie che il nuovo sistema già fornisce.

\section{Requisiti soddisfatti}
Sin dall'inizio, era stato previsto che la presente tesi non avrebbe portato alla realizzazione completa del sistema, ma avrebbe riguardato principalmente le fasi fondamentali di analisi e progettazione, accompagnate dall'implementazione di una prima versione funzionale. Nonostante ciò, è possibile tracciare un bilancio dei requisiti già soddisfatti, i quali evidenziano le potenzialità del nuovo sistema rispetto al precedente. A tal fine, i requisiti verranno suddivisi in due categorie: quelli già soddisfatti sia dal punto di vista progettuale che implementativo e quelli attualmente soddisfatti solo dal punto di vista progettuale.

\subsection{Elementi progettati e implementati}
Tra i requisiti soddisfatti, alcuni non si limitano alla fase di progettazione ma sono già stati concretamente implementati, costituendo il nucleo della prima versione del sistema. Questi elementi forniscono un primo miglioramento rispetto al legacy e rappresentano una base solida per l’evoluzione futura del progetto. Di seguito vengono analizzate le funzionalità già operative, evidenziandone l'impatto e le garanzie offerte.

\subsubsection{Gestione della multiutenza}
Uno dei principali requisiti che ha motivato la realizzazione del nuovo sistema è stato l’introduzione della multiutenza. Questo requisito, fatta eccezione per il sistema di autorizzazione, è stato ampiamente soddisfatto. Attualmente, il nuovo sistema consente di gestire più credenziali di accesso per una singola organizzazione, permettendo agli utenti autorizzati di accedere al pannello e visualizzare la dashboard. Inoltre, sono già state implementate le operazioni di creazione, modifica ed eliminazione degli utenti. Questo posiziona il nuovo sistema in una posizione di vantaggio rispetto al precedente, nel quale la gestione degli utenti era più limitata.

Dal punto di vista delle garanzie offerte, il controllo degli accessi non è ancora stato implementato, ma la struttura del sistema è predisposta per integrarlo in futuro con meccanismi flessibili e granulari. Inoltre, la sicurezza relativa alle credenziali di accesso è stata significativamente migliorata rispetto al legacy, grazie all’utilizzo dell'algoritmo di hashing avanzato \texttt{bcrypt}. Quest’ultimo consente non solo di rendere le password crittograficamente più sicure, ma permette anche di incrementare il costo computazionale dell’hashing, rendendolo scalabile nel tempo e aumentando la resistenza ad attacchi di forza bruta.

\subsubsection{Dashboard panoramica per MSP e Dealer}
Un'altra funzionalità significativa già implementata è la dashboard panoramica, progettata specificamente per MSP e Dealer. Nel sistema legacy, questi utenti non disponevano di una visione aggregata ed efficace dei clienti sotto la loro gestione. Il nuovo sistema, invece, include una schermata con statistiche e grafici che permettono di ottenere una panoramica dettagliata delle attività gestite. In particolare, la dashboard mostra le seguenti informazioni:

\begin{itemize}
  \item Nella parte superiore, vengono presentate statistiche di base relative alle richieste elaborate dal filtro DNS. Tra queste: il numero totale di richieste ricevute, il numero di minacce bloccate, il numero di categorie e indirizzi IP bloccati, il numero di categorie consentite, il numero di richieste DNS non risolte (NXDOMAIN) e, infine, il numero di richieste che hanno forzato l’utilizzo della SafeSearch su motori di ricerca e YouTube. Questi dati offrono un quadro generale sul funzionamento del sistema e sull'efficacia del filtraggio.

  \item Nella parte centrale, sono presenti grafici per monitorare l’andamento delle richieste DNS. In particolare, vi è:
    \begin{itemize}
      \item Un grafico a barre che mostra le cinque categorie più bloccate.
      \item Un grafico a torta che rappresenta la distribuzione percentuale delle richieste DNS in base alla loro tipologia (richieste bloccate, richieste consentite, blocchi per IP, domini non risolti).
    \end{itemize}

  \item Nella parte inferiore, è presente un grafico che mostra l’andamento delle richieste DNS nel tempo, distinguendo tra quelle bloccate e quelle consentite.

  \item Tutti e tre i grafici consentono di modificare l'intervallo temporale di visualizzazione, permettendo di analizzare i dati su finestre temporali di 24, 48 o 72 ore.
\end{itemize}
%
Oltre a offrire una panoramica complessiva, la dashboard consente di visualizzare statistiche e report specifici per ciascun cliente gestito dall’MSP o dal Dealer. È infatti possibile selezionare un cliente dalla lista e ottenere un quadro dettagliato delle sue attività. Questa funzionalità risulta particolarmente utile per monitorare eventuali anomalie o problemi specifici di un singolo cliente.

Al momento, la dashboard è ancora in una fase iniziale e non è possibile esprimere garanzie sulla qualità dell'aggregazione dei dati, che verrà affinata con le iterazioni successive dello sviluppo.

\subsubsection{Supporto multilingua}
Un ulteriore requisito già soddisfatto è il supporto multilingua. Il frontend del nuovo sistema supporta attualmente due lingue, italiano e inglese. Sebbene possa sembrare un dettaglio secondario, questa funzionalità è fondamentale per un prodotto destinato a un mercato internazionale, in cui gli utenti parlano lingue diverse. Inoltre, l’architettura progettata consente di aggiungere nuove lingue in modo semplice, senza necessità di modifiche al codice sorgente, ma solo integrando nuovi file di traduzione.

\subsubsection{Gestione avanzata degli errori}
Un altro elemento progettato e implementato nel nuovo sistema è la gestione avanzata degli errori, che introduce una strategia scalabile e strutturata per il trattamento delle anomalie sia lato backend che frontend.
%
Questa architettura garantisce diversi vantaggi:
\begin{itemize}
  \item \textbf{Robustezza e controllo a tutti i livelli del sistema}: la gestione degli errori è strutturata in modo gerarchico, coprendo dalla validazione degli input nelle API fino ai problemi a livello di database, evitando che errori non gestiti possano propagarsi in modo incontrollato.
  \item \textbf{Standardizzazione e interoperabilità}: gli errori vengono generati in modo schematizzato e dinamico, garantendo una rappresentazione uniforme e facilmente interpretabile da qualsiasi componente del sistema, sia interno che esterno.
  \item \textbf{Compatibilità con il sistema multilingua}: grazie alla struttura modulare degli errori, il frontend può costruire dinamicamente messaggi testuali in più lingue, senza dipendere da stringhe fisse predefinite nel backend.
  \item \textbf{Affidabilità del codice grazie a TypeScript}: il sistema di gestione degli errori sfrutta la tipizzazione statica di TypeScript, garantendo una maggiore sicurezza nella gestione delle eccezioni e facilitando l’individuazione di problemi già in fase di sviluppo, grazie al supporto degli strumenti di analisi statica degli IDE.
\end{itemize}

Rispetto al sistema legacy, che gestiva gli errori in modo frammentato e poco strutturato, questa soluzione garantisce una maggiore affidabilità complessiva, riducendo il rischio di comportamenti inattesi e migliorando la manutenibilità del codice. Inoltre, la sua scalabilità consente di adattarlo facilmente a nuove esigenze, rappresentando un elemento chiave per la futura evoluzione del sistema.

\subsection{Elementi solo progettati}
Oltre ai requisiti già implementati, vi sono alcuni elementi che, pur non essendo ancora concretamente sviluppati, sono stati completamente progettati e definiti nei dettagli. Sebbene al momento non sia possibile valutarne l’efficacia operativa, è possibile analizzare le loro potenzialità e il valore aggiunto che apporteranno rispetto al sistema legacy.

\subsubsection{Gestione dei profili condivisi}
Uno degli elementi più significativi in questa categoria è l’introduzione dei profili condivisi. Questa novità consente di creare profili di protezione condivisibili tra più organizzazioni all'interno della stessa gerarchia. Essi possono essere visti come template di protezione, utili soprattutto per gli MSP, che potranno così offrire configurazioni standardizzate e omogenee ai loro clienti. Un altro vantaggio di questa funzionalità è la semplificazione della gestione della protezione: eventuali modifiche a un profilo condiviso vengono automaticamente propagate a tutte le organizzazioni che lo utilizzano. Questo rappresenta un miglioramento significativo rispetto al vecchio sistema, dove le configurazioni erano gestite separatamente per ogni organizzazione, con difficoltà nella loro manutenzione e uniformità.

L’introduzione dei profili condivisi permette inoltre al sistema di allinearsi con i competitor, alcuni dei quali già dispongono di questa funzionalità. Benché questa caratteristica non sia ancora implementata, la sua progettazione dettagliata consente di avere una chiara roadmap per la sua realizzazione e integrazione futura.

\subsubsection{Riprogettazione del database}
Il design di questa funzionalità rientra in un più ampio processo di rimodellazione del database, che garantisce maggiore scalabilità e coerenza dei dati. Inoltre, la nuova struttura permette l’implementazione di funzionalità avanzate che il vecchio sistema, per sua natura, non poteva supportare. Questa revisione assicura che il sistema possa evolversi senza le limitazioni strutturali della versione legacy, rendendolo più flessibile, scalabile e adatto alle esigenze future.

Per quanto la riprogettazione del database sia stata completata principalmente a livello di schema concettuale, alcune componenti essenziali risultano già integrate nel sistema. In particolare, le nuove tabelle dedicate alla gestione degli utenti e delle organizzazioni sono già operative e costituiscono la base su cui si fonda il meccanismo di multiutenza. Queste modifiche hanno consentito di superare i vincoli del vecchio modello, che non era stato concepito per supportare una gestione avanzata degli accessi e delle gerarchie organizzative.

L’adozione del nuovo schema dati non solo migliora la struttura e la leggibilità del database, ma assicura anche una maggiore coerenza e manutenibilità, agevolando l’implementazione futura di altre funzionalità chiave. Inoltre, grazie al suo design modulare e scalabile, il database può adattarsi con facilità a nuove esigenze, garantendo flessibilità operativa e una gestione più efficiente dei dati.
