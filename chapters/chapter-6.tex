\chapter{Valutazione}

l'obiettivo del capitolo in questione è quello di discutere i risultati fino a questo momento ottenuti, in particolar modo sulla base dei requisiti che sono già stati soddisfatti e anche sulla base di quello che già il nuovo sistema offre in più rispetto al legacy. In effetti, non è possibile fornire una valutazione complessiva di esso, di effettuare un confronto completo con il vecchio, di fornire risultati di test prestazionali e di usabilità, in quanto il sistema non è ancora stato completato.

\section{Requisiti soddisfatti}
Era già stato preventivato il fatto che la presente tesi non avrebbe visto la realizzazione completa del sistema, ma solo una parte di esso, principalmente incentrata sulle fondamentali fasi di analisi e progettazione. Tuttavia, è possibile fare un bilancio dei requisiti che sono stati soddisfatti fino a questo momento, i quali mostrano le potenzialità del nuovo sistema rispetto al vecchio. Essi verranno organizzati in due categorie: i requisiti soddisfatti sia dal punto di vista progettuale che implementativo e quelli soddisfatti solo dal punto di vista progettuale.

\subsection{Elementi progettati e implementati}
Uno dei fondamentali requisiti che ha spinto la realizzazione del nuovo sistema è sicuramente l'implementazione della multiutenza. Esso, fatta eccezione per il sistema di autorizzazione, è stato ampiamente soddisfatto. Lo stato attuale del nuovo sistema, infatti, consente di avere più credenziali di accesso per una singola organizzazione, con le quali è possibile accedere al pannello e visionare la ddaashboard. Inoltre, è possibile creare nuovi utenti, modificarne i dati degli account già presenti ed eliminarli. La gestione degli utenti risulta quindi completa, posizionando già da adesso il sistema in una posizione di vantaggio rispetto al vecchio.

Sempre per quanto riguarda i requisiti funzionali che sono stati implementati, è possibile citare la dashboard panoramica, specificamnte rivolta agli MSP ed hai Dealer, che con il vecchio sistema non erano in grado di avere una visione complessiva ed efficace dei clienti al di sotto di essi. In questo contesto, il nuovo sistema è già in possesso di una schermata con statistiche e grafici aggregati che permettono di avere una visione d'insieme dei clienti gestiti. In particolare, la dashboard in esame mostra le seguenti informazioni:
\begin{itemize}
  \item nella parte alta vengono mostrate alcune statistiche base per quanto riguarda il numero di richieste elaborate dal filtro DNS. Tra queste, il numero totale di richieste ricevute, il numero delle minacce bloccate, il numero delle categorie e degli indirizzi IP bloccati, così come quello delle categorie consentite, il numero di richieste DNS non risolte (NXDOMAIN) e infine quello delle richieste che hanno forzato l'utilizzo della SafeSearch sui principali motori di ricerca e su Youtube. Questi dati numerici forniscono una visione d'insieme sulle attività svolte dal filtro DNS, permettendo di capire se il sistema sta funzionando correttamente e se sta bloccando le minacce in modo efficace.
  \item nella parte centrale della dashboard sono presenti alcuni grafici per monitorare l'andamento delle richieste DNS. In particolare, è presente un grafico a barre che mostra le 5 categorie più bloccate, insieme ad un grafico a torna che mostra la percentuale delle richieste totali, sulla base della loro tipologia. In particolare, esso mostra la percentuale delle richieste bloccate, di quelle consentite, dei blocchi per IP, e dei domini non risolti.
  \item Infine, ancora più in basso, è presente un grafico che mostra l'andamento delle richieste DNS nel tempo. In particolare, esso mostra il numero di richieste bloccate e di quelle consentite, in un intervallo di tempo che può essere impostato dall'utente.
  \item in generale, tutti e tre i grafici consentono di modificare lintervallo di tempo in cui visualizzare i dati, permettendo di avere una visione più dettagliata o più generale delle attività svolte dal filtro DNS. Gli intervalli temporali attualmente disponibili sono: 24, 48 e 72 ore.
\end{itemize}

Sebbene la presenta dashboard risulti una novità per la sua capacità di mostrare statistiche e report aggregati, è anche in grado di visualizzare tali informazioni contestualizzate ad uno specifico cliente al di sotto dell'utente che sta visionando la dashboard. In particolare, è possibile selezionare un cliente dalla lista di quelli gestiti e visualizzare le statistiche e i grafici relativi a lui. Questa funzionalità è stata implementata per permettere agli MSP e ai Dealer di avere una visione più dettagliata delle attività svolte da un cliente, in modo da poter intervenire in caso di problemi o di anomalie.

Un altro requisito che risulta pienamente soddisfatto è il supporto alla multilingua per il nuovo sistema. Il suo frontend, infatti, supporta già da adesso la traduzione in due lingue, ovvero l'italiano e l'inglese. Sebbene possa sembrare una funzione di poco conto, esso risulta fondamentale per un sistema che è destinato ad essere utilizzato in contesti internazionali, in cui gli utenti possono provenire da paesi diversi e parlare lingue diverse. Inoltre, per come è stato progettato, l'aggiunta di nuove lingue risulta essere un'operazione semplice e veloce, che non richiede la modifica del codice sorgente ma solo l'aggiunta di nuovi file di traduzione.

\subsection{Elementi solo progettati}
% parlare della gestione dei profili condivisi, che però sono solo stati progettati. A questo, ricollegare la completa rimodellazione del database, elemento anch'esso progettato ma non ancora implementato nella base di dati usata per lo sviluppo.
