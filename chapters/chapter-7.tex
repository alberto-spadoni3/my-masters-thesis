\chapter{Conclusione}

Il presente lavoro di tesi ha avuto come obiettivo la reingegnerizzazione del pannello di configurazione di un sistema legacy per il filtraggio DNS, progettando e implementando un'infrastruttura moderna e scalabile in grado di superare le limitazioni della versione precedente. Il progetto si è focalizzato sull’analisi e la riprogettazione dell’architettura, gettando le basi per una transizione completa verso un sistema più efficiente e flessibile. Questo capitolo riepiloga i principali risultati ottenuti, le sfide affrontate e gli sviluppi futuri previsti per il completamento e l’evoluzione del sistema.

\section{Sintesi del lavoro svolto e lezioni apprese}
Lo sviluppo di un sistema software segue tipicamente un processo articolato in più fasi, tra cui la raccolta dei requisiti, l’analisi e progettazione, l’implementazione e il testing. La presente tesi è intervenuta dopo la fase di raccolta dei requisiti, concentrandosi sull'analisi e la progettazione di un sottoinsieme di caratteristiche chiave della nuova piattaforma, nonché sulla loro prima implementazione. L’obiettivo principale è stato quello di costruire una base solida su cui fondare l’evoluzione del sistema, risolvendo le principali criticità della versione legacy e introducendo soluzioni volte a migliorare scalabilità, sicurezza e manutenibilità.
%
In particolare, le milestone principali raggiunte includono:

\begin{itemize}
  \item \textbf{Multiutenza e gestione degli accessi}: una delle caratteristiche chiave del nuovo sistema rispetto al legacy è la possibilità di supportare un modello multi-tenant, permettendo la gestione di account con permessi differenziati. Attualmente, il sistema di autenticazione è stato implementato con successo e consente un accesso sicuro alla piattaforma. Tuttavia, il livello di autorizzazione, essenziale per regolare con precisione i permessi di ciascun utente in base al ruolo e all’organizzazione di appartenenza, è ancora in fase di definizione. La sua integrazione futura consentirà di sfruttare appieno il potenziale della multiutenza, offrendo una gestione più strutturata degli accessi e migliorando la sicurezza del sistema.

  \item \textbf{Dashboard panoramica per MSP}: l’introduzione di una dashboard dedicata ai Managed Service Provider ha reso possibile una visione aggregata delle attività dei clienti, migliorando il controllo e la gestione della protezione DNS. Questo strumento fornisce ai MSP dati centralizzati e statistiche dettagliate, semplificando il monitoraggio delle configurazioni e delle minacce bloccate. Oltre a migliorare l’esperienza degli amministratori, questa funzionalità posiziona strategicamente il prodotto in un mercato più orientato al settore enterprise, che rappresenta uno degli obiettivi primari dell’azienda in termini di crescita e competitività.

  \item \textbf{Nuova architettura del database}: la rimodellazione della base dati ha consentito di superare le rigidità della vecchia struttura, garantendo maggiore coerenza e supporto nativo alle nuove funzionalità. Oltre a risolvere le limitazioni del database legacy, la nuova architettura è stata progettata seguendo le migliori pratiche nel disegno concettuale delle strutture dati. Sono stati adottati nomi significativi per le entità, così come tipi di dato più coerenti e precisi, il tutto per evitare ambiguità e migliorare l'integrità del database. Inoltre, la suddivisione delle tabelle è stata ottimizzata per garantire un'organizzazione più chiara e una gestione più efficiente delle relazioni tra i dati, ponendo le basi per un sistema scalabile e facilmente estendibile.

  \item \textbf{Gestione avanzata e strutturata degli errori}: è stato sviluppato un sistema innovativo per la gestione degli errori, progettato per essere scalabile, facilmente integrabile tra frontend e backend e compatibile con la gestione multilingua. Questo sistema è pervasivo, operando a tutti i livelli del backend del sistema, dalla validazione degli input nelle API alla gestione delle eccezioni interne e degli errori a livello di database. La sua implementazione sfrutta la tipizzazione avanzata del linguaggio di programmazione utilizzato, garantendo un controllo rigoroso e una maggiore robustezza nel trattamento degli errori. Inoltre, l’approccio scelto lo rende poco invasivo dal punto di vista implementativo, permettendo agli sviluppatori di utilizzarlo in modo efficace senza introdurre complessità eccessive nel codice esistente. Questo aspetto ha rappresentato uno dei contributi più significativi apportati al progetto, migliorando la manutenibilità e l'affidabilità dell’intero sistema.
\end{itemize}

\subsubsection{Lezioni apprese}
Durante lo sviluppo del sistema sono emerse diverse criticità, che hanno richiesto un'analisi approfondita e un approccio metodologico mirato. Tra le principali difficoltà affrontate vi sono:

\begin{itemize}
  \item \textbf{Apprendimento di nuove tecnologie}: la realizzazione del nuovo sistema ha richiesto l’utilizzo di strumenti e paradigmi non presenti nella versione legacy, rendendo necessaria un'iniziale fase di formazione e consolidamento delle conoscenze. In particolare, l’ambiente di sviluppo è stato completamente rinnovato, introducendo linguaggi di programmazione diversi, framework moderni e un'architettura completamente differente rispetto al passato. Questo ha comportato non solo l'acquisizione di nuove competenze tecniche, ma anche la necessità di rivedere le metodologie di lavoro per adattarsi a un paradigma di sviluppo più strutturato e modulare.

  \item \textbf{Strutturazione avanzata degli errori}: una delle principali difficoltà in questo contesto è stata la definizione di un meccanismo di gestione degli errori che fosse al contempo semplice ed efficace. È stato necessario individuare una quantità esaustiva di potenziali errori da codificare, garantendo una categorizzazione chiara e uniforme per facilitarne la gestione. Inoltre, la strutturazione del sistema di error handling doveva assicurare un'interoperabilità coerente tra backend e frontend, evitando ambiguità nella comunicazione. Un ulteriore vincolo è stato quello di progettare un meccanismo che permettesse di costruire i messaggi lato frontend in modo semplice e intuitivo, facilitando al contempo la traduzione automatizzata e la gestione multilingua dell’applicazione.
\end{itemize}

\subsubsection{Lezioni apprese}
Dall’esperienza maturata in questo progetto sono emerse alcune scelte architetturali particolarmente efficaci, che si sono rivelate strategiche per la solidità del nuovo sistema:

\begin{itemize}
  \item \textbf{Rinnovo del database fin dalle prime fasi di sviluppo}: questo approccio ha garantito la costruzione di un sistema privo delle inefficienze della versione legacy, con una base dati progettata su misura per supportare le nuove funzionalità senza compromessi. Il ripensamento dell’architettura dati fin dall'inizio ha permesso di eliminare le rigidità strutturali preesistenti, migliorando le prestazioni e la scalabilità del sistema. Inoltre, ha reso più semplice l’integrazione con le nuove componenti software, riducendo la necessità di soluzioni temporanee o workaround che avrebbero potuto complicare la manutenzione futura.

  \item \textbf{Adozione di un'architettura a microservizi}: questa scelta ha permesso di garantire modularità, scalabilità (anche geografica) e una più semplice manutenzione del sistema, rendendolo adattabile alle esigenze future. Inoltre, con il team di sviluppo in continua crescita, l'approccio a microservizi ha facilitato la suddivisione del lavoro e la gestione delle competenze, permettendo a ciascun membro di concentrarsi su un'area specifica del sistema.

  \item \textbf{Strategia di reingegnerizzazione ibrida}: la scelta di un processo di transizione netta, senza una fase di coesistenza tra vecchio e nuovo sistema, si sta rivelando vincente poiché combina i benefici della modularità con una sostituzione completa e controllata. Questo approccio si adatta particolarmente bene a contesti in cui è necessario trasformare un vecchio sistema monolitico in un’architettura a microservizi, senza introdurre complessità dovute a un periodo di interoperabilità tra le due versioni. Eliminando la necessità di mantenere attivi due sistemi contemporaneamente, si riducono i costi operativi e si accelera la piena adozione del nuovo software.
\end{itemize}

\section{Evoluzioni e prospettive future}
Il lavoro svolto finora rappresenta solo il primo passo verso la realizzazione di un sistema completo e pienamente operativo. Sebbene siano già pianificati sviluppi successivi per il perfezionamento delle funzionalità esistenti, vi sono ulteriori direzioni di evoluzione che potrebbero essere esplorate in futuro per potenziare ulteriormente il sistema. Tra le possibili estensioni e miglioramenti si evidenziano:

\begin{itemize}
  \item \textbf{Integrazione con sistemi di gestione delle identità e directory aziendali}: un'importante evoluzione potrebbe riguardare l'integrazione con servizi come Active Directory o altri Identity Provider aziendali. Questa estensione permetterebbe di applicare il filtraggio DNS in maniera ancora più granulare, assegnando policy di navigazione non solo per rete, ma anche per singolo utente o gruppi definiti all’interno dell’organizzazione. In questo modo, le configurazioni potrebbero essere centralizzate e automatizzate, migliorando la gestione della sicurezza e la conformità aziendale.

  \item \textbf{Implementazione di un sistema di autenticazione centralizzato}: attualmente l’autenticazione degli utenti è gestita in maniera autonoma dal sistema, ma un possibile sviluppo futuro potrebbe prevedere la realizzazione di un servizio dedicato alla gestione centralizzata dell'autenticazione e dei permessi. Questo sistema potrebbe fungere da Single Sign-On (SSO) per tutti i servizi aziendali, permettendo agli utenti di autenticarsi una sola volta per accedere a tutte le applicazioni interne. Oltre a migliorare l’esperienza utente, un simile approccio garantirebbe una gestione più sicura e uniforme delle credenziali e dei livelli di accesso, riducendo i rischi legati alla duplicazione e alla dispersione delle informazioni di autenticazione.

  \item \textbf{Automazione dell’allineamento tra specifiche API e implementazione}: attualmente, la documentazione delle API e l’implementazione delle rotte backend devono essere mantenute manualmente, con il rischio di disallineamenti nel tempo. Un'evoluzione strategica potrebbe essere l'adozione di strumenti come Swagger o OpenAPI, che consentono di generare automaticamente la documentazione a partire dal codice sorgente, garantendo così una perfetta corrispondenza tra specifiche e implementazione reale. Questo approccio migliorerebbe la manutenibilità del sistema, riducendo gli errori di integrazione con i client esterni e agevolando il lavoro degli sviluppatori nel comprendere e utilizzare le API disponibili.
\end{itemize}

L’insieme di questi sviluppi contribuirebbe a rendere il nuovo sistema una piattaforma sempre più completa, performante e adattabile alle esigenze operative degli utenti. Oltre a migliorare l’efficienza e la sicurezza, queste integrazioni favorirebbero una maggiore interoperabilità con altri servizi e infrastrutture aziendali, rendendo il sistema più scalabile e competitivo rispetto alle soluzioni attualmente disponibili sul mercato. Sebbene al momento queste evoluzioni non siano ancora state pianificate nei dettagli, esse rappresentano delle opportunità strategiche che potrebbero essere prese in considerazione per il futuro, in funzione delle esigenze emergenti e delle prospettive di crescita del progetto.
