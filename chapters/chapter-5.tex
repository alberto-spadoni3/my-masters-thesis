\chapter{Implementazione}

Questo capitolo illustra il piano di sviluppo del nuovo sistema, evidenziando gli elementi tecnici di maggiore rilievo per il valore complessivo della presente tesi. In particolare, viene analizzata la gestione degli errori, descrivendo le soluzioni adottate sia nel backend sia nel frontend. L'approfondimento proposto amplia e integra quanto discusso nella \Cref{sec:error-management}, dove era stata esaminata la fase di progettazione in modo preliminare.

\section{Piano di Sviluppo}
Lo sviluppo del nuovo sistema è stato condotto seguendo la metodologia Agile \textit{SCRUM}\footnote{\url{https://scrumguides.org/scrum-guide.html}}, che prevede un'organizzazione del lavoro in cicli iterativi e incrementali della durata di due settimane (\textit{sprint}). Questo approccio ha permesso di suddividere il progetto in incrementi funzionali rilasciabili progressivamente, favorendo una maggiore flessibilità nei confronti dei requisiti emergenti e un continuo miglioramento del prodotto.

Al termine di ogni sprint, si è tenuto un meeting di revisione con il \textit{Project Manager} (PM), durante il quale sono state analizzate le criticità emerse, verificata la conformità delle ore lavorate rispetto a quelle preventivate e pianificate le attività per lo sprint successivo. Questo processo ha garantito un flusso di sviluppo costante e ben organizzato, consentendo al team di intervenire tempestivamente su eventuali problematiche e di mantenere un'elevata qualità del codice.

Il team di sviluppo è composto esclusivamente da \textit{Full Stack Developer}, senza una suddivisione rigida tra frontend e backend. Le attività di progettazione grafica non sono state gestite internamente, ma affidate a un designer esterno incaricato del rinnovo della brand identity. Di conseguenza, per quanto riguarda la UI/UX, il team si è occupato esclusivamente dell'implementazione delle nuove interfacce grafiche, attenendosi alle specifiche fornite dal designer e garantendo la corretta integrazione delle soluzioni proposte all'interno del sistema.

Per la gestione e il monitoraggio delle attività, è stato utilizzato lo strumento di project management \textit{ClickUp}\footnote{\url{https://clickup.com}}, che ha permesso di tracciare i task, assegnare priorità e garantire un'organizzazione efficace del lavoro.

Le attività svolte durante gli sprint hanno incluso lo sviluppo di nuove funzionalità, la risoluzione di bug, il refactoring del codice e l'ottimizzazione delle performance. Particolare attenzione è stata dedicata alla qualità del software, con l'integrazione di test automatici e revisioni periodiche del codice prima del rilascio delle feature.

\subsection{Ambiente di sviluppo e infrastruttura}
L'ambiente di sviluppo è stato configurato in modo da rispecchiare il più possibile l'ambiente di produzione, garantendo una maggiore affidabilità nei test e riducendo il rischio di anomalie legate a differenze infrastrutturali. Per ottenere questa uniformità, sono state adottate tecnologie che permettono di replicare fedelmente la configurazione di produzione.

L'intero sistema di sviluppo è stato \textit{containerizzato}\footnote{Il termine ``containerizzare'' si riferisce al processo di incapsulamento di un'applicazione e delle sue dipendenze all'interno di un container, ovvero un'unità standardizzata di software che garantisce che l'applicazione possa essere eseguita in modo coerente in qualsiasi ambiente. (\url{https://www.docker.com/resources/what-container})} tramite \textit{Docker}, con i vari servizi orchestrati da \textit{Docker Compose}. Questo permette di avviare l’ambiente di sviluppo in modo rapido e riproducibile su diverse macchine, riducendo problemi di configurazione tra membri del team. Inoltre, l'uso di \textit{Docker-in-Docker} nella pipeline \textit{Continuous Integration/Continuous Deployment} (CI/CD) consente di replicare l’ambiente di esecuzione anche nei \textit{job} automatici di test e build.

Per la gestione delle configurazioni, ogni microservizio utilizza un file \texttt{.env}, che contiene le variabili d’ambiente necessarie per la configurazione dinamica dello stesso, come le credenziali di accesso al database e gli endpoint delle API. Durante l’esecuzione della pipeline CI/CD, questo file viene iniettato nei container, permettendo di mantenere un ambiente coerente tra sviluppo, test e produzione, senza la necessità di configurazioni separate per ciascun contesto.

Per migliorare l'affidabilità dei test, è stata implementata una strategia di \textit{seeding} del database, che consente di popolare l’ambiente di test con dati coerenti a ogni esecuzione, assicurando risultati riproducibili e verifiche affidabili.

\subsection{Struttura della repository}
L'intero codice del progetto è gestito tramite una \textit{monorepo} ospitata su \textit{GitLab}. La scelta di adottare una monorepo è motivata dalla necessità di mantenere in un unico repository sia il frontend che tutti i microservizi del backend, semplificando la gestione delle dipendenze, la coerenza tra i moduli e l'esecuzione delle pipeline di CI/CD.

Per ottimizzare la gestione della monorepo, è stato utilizzato \texttt{Turborepo}\footnote{\url{https://turbo.build}}, un tool specificamente progettato per lo sviluppo di applicazioni in TypeScript. Questo strumento consente di affrontare in modo efficiente il problema dei lunghi processi di compilazione, tipici delle monorepo con numerosi pacchetti e molteplici task (come compilazione, testing e linting). Il tool pianifica l'esecuzione dei task in modo ottimizzato, parallelizzandoli su tutti i cores disponibili e implementando un avanzato sistema di caching. Questo permette di ridurre drasticamente i tempi di build successivi al primo, migliorando la produttività del team di sviluppo.

L'organizzazione della monorepo sfrutta la funzionalità \textit{workspace} di \texttt{pnpm}, suddividendo i pacchetti in due macrocategorie:
\begin{itemize}
  \item \textbf{Apps}: Contiene tutti i servizi che vengono eseguiti in maniera indipendente e che costituiscono i diversi microservizi del sistema. Qui sono presenti il frontend e tutti i componenti del backend che operano come unità autonome.
  \item \textbf{Packages}: Include pacchetti di supporto utilizzati dalle \textit{apps}. Tra questi vi sono il package \texttt{commons}, che contiene definizioni e metodi condivisi tra i vari servizi, i pacchetti dedicati alla gestione dello schema di \textit{Prisma} e all'esportazione del client per il database, oltre ai pacchetti per il seeding di quest'ultimo nella fase di test.
\end{itemize}

\subsubsection{Modello di branching}
Per organizzare al meglio il lavoro di sviluppo del team, è stato adottato un modello di branching strutturato, che consente di gestire in modo chiaro e controllato il ciclo di vita del codice. Esso prevede i seguenti branch principali:
\begin{itemize}
  \item \textbf{Main}: Contiene il codice stabile e rappresenta l'unico branch dal quale si effettuano rilasci in produzione.
  \item \textbf{Unstable}: Include le funzionalità candidate al rilascio in produzione. Il codice qui presente è testato in un ambiente di staging prima di essere eventualmente integrato nel branch \textit{main}.
  \item \textbf{Branch personali}: Ogni sviluppatore lavora principalmente su un branch personale, denominato con il proprio cognome, per eseguire i commit delle proprie modifiche prima di unire il codice nei branch condivisi.
  \item \textbf{Feature branches}: Per lo sviluppo di funzionalità specifiche, possono essere creati branch temporanei, indipendenti dai branch personali, in modo da favorire un'organizzazione più modulare del codice.
\end{itemize}
%
Questa strategia consente di mantenere un flusso di sviluppo ordinato, con una chiara separazione tra il codice in produzione, il codice in fase di test e le modifiche in sviluppo. Inoltre, grazie all'integrazione con le pipeline CI/CD di GitLab, il sistema è in grado di eseguire automaticamente test e build per ogni \textit{push} e \textit{merge request}, garantendo un'elevata affidabilità prima della promozione del codice verso l'ambiente di produzione.

\subsubsection{Pipeline CI/CD}
L'automazione del processo di sviluppo è stata realizzata attraverso una pipeline di \textit{CI/CD}, integrata direttamente in GitLab. La pipeline è suddivisa in più fasi, organizzate per garantire una validazione progressiva del codice prima del rilascio in staging.

L'intero processo di build e test si basa su Docker-in-Docker\footnote{\url{https://www.docker.com/resources/docker-in-docker-containerized-ci-workflows-dockercon-2023}} (DinD), una soluzione che permette alla pipeline di eseguire e gestire container Docker all'interno di ambienti di esecuzione basati sulla stessa tecnologia. Questo approccio consente di costruire immagini Docker direttamente nei job della pipeline, riducendo la dipendenza da infrastrutture esterne e garantendo isolamento e una maggiore coerenza tra gli ambienti di sviluppo, test e deploy.
%
Il flusso di lavoro della pipeline prevede le seguenti fasi:
\begin{enumerate}
  \item \textbf{Build dell'ambiente di sviluppo}: I servizi vengono compilati e avviati tramite il comando \texttt{docker-compose}, con supporto alla parallelizzazione dei task e caching per ottimizzare i tempi di build.
  \item \textbf{Esecuzione dei test}: Ogni microservizio è testato in container isolati. Viene eseguito il processo di seeding del database per simulare scenari reali e ottenere una copertura completa del codice.
  \item \textbf{Build per la produzione}: Se i test risultano superati, viene generata una build ottimizzata per il rilascio, con tagging e push delle immagini Docker nel registry di GitLab.
  \item \textbf{Deploy sull'ambiente di staging}: Il codice viene distribuito aggiornando i container in esecuzione e ripristinando il database con i dati necessari per il test pre-produzione.
\end{enumerate}

La pipeline utilizza strategie di caching avanzate per ridurre i tempi di esecuzione. I pacchetti gestiti da \texttt{pnpm} e la cache di \texttt{Turborepo} vengono riutilizzati tra i job, evitando ricompilazioni non necessarie. Inoltre, il sistema di \textit{retry} proprio della pipeline, assicura la ripetizione automatica dei job in caso di errori transitori, migliorando l'affidabilità del processo.

Grazie a questa configurazione è possibile garantire un flusso di sviluppo strutturato e sicuro, assicurando che solo codice stabile e testato venga promosso verso l'ambiente di staging e, successivamente, verso quello di produzione.

\subsection{Test e controllo qualità}
La verifica della qualità del codice e la copertura tramite test automatizzati hanno rappresentato un aspetto fondamentale del piano di sviluppo. Il processo di testing è stato strutturato seguendo queste linee guida:

\begin{itemize}
  \item Garantire una copertura del 100\% del codice sorgente dell'intero sistema, attraverso test unitari e di integrazione.
  \item Mantenere standard elevati di qualità del codice attraverso strumenti di analisi statica come \texttt{ESLint}\footnote{\url{https://eslint.org}} e \texttt{Prettier}\footnote{\url{https://prettier.io}}, utilizzati per il rispetto delle best practices di sviluppo (il primo) e per il controllo della formattazione (il secondo).
  \item Valutare le performance e il comportamento del sistema in scenari critici mediante test prestazionali.
  \item Per il backend, è stata utilizzata la libreria \texttt{Mocha}\footnote{\url{https://mochajs.org}}, che consente di eseguire test unitari e di integrazione in un ambiente controllato, verificando il corretto funzionamento delle API e della logica applicativa.
  \item Per il frontend, è stato adottato \texttt{Playwright}\footnote{\url{https://playwright.dev}}, uno strumento per il testing end-to-end che permette di simulare interazioni utente su diversi browser.
\end{itemize}

\subsection{Approccio alla reingegnerizzazione}
Come descritto nella \Cref{sec:reingegnerizzazione-approcci-fasi}, ogni modello di reingegnerizzazione presenta vantaggi e criticità differenti. Il processo adottato per lo sviluppo del nuovo sistema si pone a metà tra l'approccio ``Big Bang'' e quello ``Evolutivo'' in quanto procede in modo incrementale, senza tuttavia una fase di coesistenza con il vecchio. Una volta completato, esso sostituirà integralmente il sistema precedente, fatta eccezione per alcune funzionalità che continueranno a essere gestite dal legacy senza necessità di retrocompatibilità.

La transizione avverrà in modo diretto, senza un rilascio graduale in produzione. Questo semplifica la migrazione, evitando la necessità di interfacce di compatibilità tra i due sistemi. Tuttavia, a differenza di un classico approccio ``Big Bang'', lo sviluppo non è stato affrontato come una riscrittura monolitica. Il nuovo sistema, infatti, è stato progettato fin dall'inizio con un'architettura a microservizi, organizzata sulla base delle funzionalità piuttosto che sulla replica della struttura esistente.

Questo approccio ibrido permette di bilanciare i vantaggi dei due modelli. La sostituzione completa del vecchio sistema elimina la necessità di mantenere allineate due versioni in parallelo, riducendo la complessità operativa. Allo stesso tempo, l'architettura modulare migliora la manutenibilità e facilita l'integrazione di nuove tecnologie nel tempo. Questa fusione di metodologie garantisce una transizione più controllata, riducendo il rischio di regressioni e assicurando una maggiore stabilità per il sistema finale.

\section{Implementazione della gestione degli errori}
La gestione degli errori rappresenta un aspetto cruciale dell’implementazione del nuovo sistema, garantendo uniformità e coerenza tra backend e frontend. Come descritto nella \Cref{sec:error-management} del capitolo precedente, il design della gestione degli errori è stato strutturato attorno a un modello tipizzato e scalabile, che consente di categorizzare le anomalie in base alla loro natura e al contesto in cui si verificano. In questa sezione verranno discussi gli aspetti salienti relativi all'implementazione di questo modello, partendo dalla struttura del backend e analizzando successivamente il meccanismo in funzione del frontend.

\subsection{Gestione degli errori nel backend}
Nel backend, la gestione degli errori è stata implementata attraverso una struttura tipizzata e centralizzata, che garantisce una rappresentazione coerente delle anomalie e facilita la comunicazione con il frontend. Il sistema si basa su una gerarchia di classi che categorizzano le diverse tipologie di errore, sfruttando appieno il sistema di tipi di TypeScript per garantire sicurezza e scalabilità. In questa sezione si analizzeranno nel dettaglio le componenti principali dell’implementazione, tra cui il meccanismo di mappatura degli errori tramite \texttt{ErrorDetailsMapping}, la classe base \texttt{AppError} e il tipo \texttt{AppErrorPayload}.

\subsubsection{Struttura di \texttt{ErrorDetailsMapping}}
Un elemento centrale nella gestione degli errori è rappresentato dalla struttura \texttt{ErrorDetailsMapping}, che definisce una mappatura tra i codici d’errore e le informazioni associate a ciascuno di essi. Questo meccanismo consente di tipizzare in modo rigoroso le eccezioni, garantendo una gestione strutturata e uniforme tra backend e frontend.

Il tipo \texttt{ErrorDetailsMapping} è stato realizzato utilizzando il concetto di \textit{Discriminated Union}\footnote{\url{https://www.typescriptlang.org/docs/handbook/2/narrowing.html\#discriminated-unions}}, una tecnica avanzata di TypeScript che consente di associare a ogni codice d’errore una struttura dati ben definita. Questo approccio elimina il rischio di assegnare dettagli incoerenti o incompleti, assicurando che ogni errore sia rappresentato in modo chiaro e prevedibile.
%
Di seguito è riportata una porzione di codice relativa al tipo \texttt{ErrorDetailsMapping}:

\lstinputlisting[
  language=typescript,
  caption={Porzione di \texttt{ErrorDetailsMapping}},
label={lst:errorDetailsMapping}]
{listings/errorDetails.ts}
%
La mappatura degli errori è organizzata in categorie, ciascuna identificata da un prefisso che segue una convenzione precisa. Ad esempio, gli errori di validazione degli input sono contrassegnati dal prefisso ``\texttt{validation}'', mentre quelli relativi alla gestione delle entità utilizzano prefissi come ``\texttt{user}'' o ``\texttt{organization}''. Questa scelta non è casuale: i prefissi sono stati definiti in modo da rispecchiare la struttura delle cartelle nel frontend, che organizza i file di traduzione per le stringhe di testo presenti nell’applicazione. In questo modo, la corrispondenza tra i codici di errore del backend e le chiavi di traduzione nel frontend è diretta e immediata.

Ogni chiave in \texttt{ErrorDetailsMapping} è associata a un oggetto che rappresenta i dettagli dell’errore corrispondente. La struttura di questi dettagli varia in base al tipo di errore: alcuni codici, come ``\texttt{user.not\_found}'', sono associati a un semplice oggetto contenente un identificativo, ad esempio il campo \texttt{userId}, che indica l’utente non trovato. Altri errori, invece, richiedono una descrizione più articolata del problema.
%
Un esempio di questa maggiore complessità si trova negli errori relativi alla creazione di un'entità, come ``\texttt{user.create\_failed}''. In questo caso, l’oggetto associato include un campo \texttt{cause}, che fornisce una rappresentazione schematica della causa del fallimento. Questo campo non è un semplice identificativo, ma può assumere valori differenti a seconda della natura dell’errore. La sua definizione è basata sugli \textit{Union Types}\footnote{\url{https://www.typescriptlang.org/docs/handbook/2/everyday-types.html\#union-types}} di Typescript, ovvero un’unione di tipi che permettono di distinguere tra cause generali, come ``\texttt{already\_exists}'', e problemi più specifici legati al database, come violazioni di unicità o vincoli di chiave esterna.
%
La seguente definizione mostra come queste categorie di errore sono tipizzate:

\lstinputlisting[
  language=typescript,
  caption={Definizione del tipo \texttt{cause} per operazioni di creazione},
label={lst:createErrorCause}]
{listings/errorTypes.ts}
%
Questo approccio offre un controllo statico rigoroso sugli errori generati dal backend, assicurando che ogni valore di \texttt{CreateErrorCause} sia riconducibile a una stringa predefinita che rappresenta la causa esatta dell’errore. Ciò evita l’uso di messaggi arbitrari e favorisce una comunicazione strutturata tra backend e frontend.
%
Un ulteriore vantaggio di questa struttura è la sua leggibilità. I codici di errore e le relative cause sono stati progettati per essere facilmente convertibili in stringhe comprensibili, consentendo al frontend di generare messaggi chiari per l’utente finale. Questo meccanismo centralizza la gestione delle stringhe di errore, riducendo la necessità di definizioni ridondanti e migliorando la manutenibilità del sistema.
%
L’adozione delle discriminated unions garantisce inoltre che:
\begin{itemize}
  \item Ogni codice d’errore sia associato ai dati corretti, evitando ambiguità.
  \item Il frontend possa interpretare le informazioni senza controlli manuali sui tipi.
  \item L’aggiunta di nuovi errori sia semplice e non impatti il codice esistente.
\end{itemize}

\subsubsection{Classe \texttt{AppError}}
La gestione degli errori nel backend si basa anche su una classe astratta denominata \texttt{AppError}, che funge da base per la definizione di tutte le eccezioni applicative. Questa classe fornisce un modello coerente per la rappresentazione degli errori, assicurando che ogni anomalia venga espressa attraverso un formato strutturato e facilmente interpretabile dal sistema. L'obiettivo principale è quello di centralizzare la gestione degli errori, evitando soluzioni dispersive e garantendo una chiara separazione tra la logica applicativa e il trattamento delle anomalie.

Per garantire la massima flessibilità e sicurezza dei tipi, la classe \texttt{AppError} sfrutta i \textit{Generics}\footnote{\url{https://www.typescriptlang.org/docs/handbook/2/generics.html}} di TypeScript. Questo approccio consente di vincolare ogni istanza di errore a un preciso codice identificativo, associandolo a un insieme di dettagli specifici definiti nel tipo \texttt{ErrorDetailsMapping}. Grazie a questa struttura, ogni errore è caratterizzato da tre elementi principali: un codice univoco, uno stato HTTP corrispondente e un set di dettagli contestuali che forniscono informazioni aggiuntive sull'errore.

\lstinputlisting[
  language=TypeScript,
  label={lst:appError},
caption={Implementazione della classe \texttt{AppError}}]
{listings/appError.ts}

\paragraph{Definizione del tipo generico per \texttt{AppError}}
Un elemento fondamentale dell'implementazione della classe \texttt{AppError} è l'uso della keyword \texttt{keyof}\footnote{\url{https://www.typescriptlang.org/docs/handbook/2/keyof-types.html}} durante la definizione del tipo generico \texttt{K}. Questo costrutto di TypeScript, visionabile a riga 2 del \Cref{lst:appError}, consente di ottenere un'unione di stringhe che rappresentano tutti i possibili codici d'errore previsti nel sistema. In altre parole, \texttt{keyof ErrorDetailsMapping} genera dinamicamente un insieme di stringhe che corrispondono esattamente ai nomi delle chiavi definite in \texttt{ErrorDetailsMapping}.
%
Consideriamo, ad esempio, il seguente sottoinsieme di codici di errori:
\begin{lstlisting}[language=typescript, caption={Esempio di unione di stringhe generate da \texttt{keyof}}]
  "validation.input_error" | "internal.database_error" | "user.create_failed"
\end{lstlisting}
%
Grazie a questo meccanismo, il parametro generico \texttt{K} della classe \texttt{AppError} è automaticamente limitato ai valori definiti in \texttt{ErrorDetailsMapping}. In questo modo, qualsiasi istanza della classe d'errore può essere creata solo con un codice conforme alle specifiche del sistema, evitando l'utilizzo di stringhe arbitrarie che potrebbero causare errori di inconsistenza.

\paragraph{Vincolo sul tipo dei dettagli dell’errore}
Oltre a garantire che il codice di errore appartenga a un insieme predefinito, il tipo generico \texttt{K} consente anche di vincolare il formato dei dettagli associati all’errore. Ogni codice d’errore definito in \texttt{ErrorDetailsMapping} è associato a un oggetto contenente informazioni specifiche per quel particolare tipo di anomalia. La classe \texttt{AppError} assicura che il campo \texttt{details} rispetti esattamente la struttura prevista per il codice d’errore selezionato.
%
Ad esempio, se si crea un'istanza di errore per rappresentare un utente non trovato, TypeScript garantirà che i dettagli contengano esclusivamente le proprietà previste per quel tipo di errore. Il codice qui sotto mostra un caso d’uso concreto:

\lstinputlisting[
  language=typescript,
  caption={Esempio di utilizzo della classe \texttt{AppError}},
label={lst:appErrorUsage}]
{listings/appErrorUsage.ts}
%
In questo caso, il codice dell'errore è limitato alla stringa ``\texttt{user.not\_found}'', e i dettagli sono strutturati esattamente come specificato in \texttt{ErrorDetailsMapping}. Se si tentasse di fornire un oggetto dettagli con proprietà non conformi, TypeScript genererebbe un errore a tempo di compilazione, impedendo di introdurre inconsistenze nel sistema.

\paragraph{Definizione del tipo \texttt{AppErrorPayload}}
Un altro elemento centrale di \texttt{AppError} è il metodo \texttt{toJSON()}, il quale ha il compito di serializzare l’errore in un formato compatibile con la comunicazione tra backend e frontend. Questo metodo restituisce un oggetto che segue la struttura imposta dal tipo \texttt{AppErrorPayload}, di seguito definito:

\lstinputlisting[
  language=typescript,
  caption={Definizione del tipo \texttt{AppErrorPayload}},
label={lst:appErrorPayload}]
{listings/appErrorPayload.ts}
%
La sintassi utilizzata in questa definizione sfrutta alcune funzionalità avanzate di TypeScript:
\begin{itemize}
  \item \textbf{Mapped Types}: la parte di codice a linea 2 itera su ciascuna chiave \texttt{K} presente in \texttt{ErrorDetailsMapping}. Per ogni chiave, viene creato un nuovo \textit{object} avente due proprietà:
    \begin{itemize}
      \item \texttt{code} di tipo \texttt{K}, che rappresenta il codice d’errore;
      \item \texttt{details} di tipo \texttt{ErrorDetailsMapping[K]}, che contiene i dettagli specifici associati a quel codice.
    \end{itemize}

  \item \textbf{Indexed Access Types}: l'intera mappatura è poi indicizzata con \texttt{[keyof ErrorDetailsMapping]} a linea 6. Questa operazione estrae dall'oggetto mappato una unione di tutti i tipi generati per ogni chiave. In altre parole, il risultato finale è l'unione dei tipi object definiti per ciascuna chiave in \texttt{ErrorDetailsMapping}.
\end{itemize}
%
In sintesi, \texttt{AppErrorPayload} rappresenta uno union type, in cui ogni membro dell'unione è un oggetto con due proprietà:
\begin{itemize}
  \item \texttt{code}, il quale contiene un identificativo dell'errore;
  \item \texttt{details}, il quale fornisce informazioni specifiche associate a quell'errore, basate sulla mappatura definita in \texttt{ErrorDetailsMapping}.
\end{itemize}

Questo design consente a \texttt{AppError} di gestire in modo robusto e tipizzato una serie di errori diversi, garantendo che ognuno di essi abbia una struttura dati coerente e ben definita.

\subsubsection{Gestione degli errori di validazione degli input}\label{sec:input-validation-errors}
Nel backend, la validazione degli input ricevuti dalle API è affidata alla libreria \texttt{Zod}\footnote{\url{https://zod.dev}}, la quale fornisce un sistema di definizione di schemi e di verifica automatica della conformità dei dati. Quando un input non rispetta lo schema previsto, Zod genera un'eccezione di tipo \texttt{ZodError}, che contiene informazioni dettagliate sugli errori riscontrati. Tuttavia, questo errore non è direttamente utilizzabile per la comunicazione con il frontend, motivo per cui è necessario un processo di elaborazione che trasformi i dati forniti da Zod in un formato strutturato e coerente con il modello di gestione degli errori che si sta discutendo.
%
A tale scopo, il backend intercetta l’errore lanciato da Zod e lo elabora per costruire un oggetto di tipo \texttt{InputError}. Questo oggetto contiene:
\begin{itemize}
  \item Il nome del campo che ha fallito la validazione;
  \item Un insieme di dettagli strutturati che permettono di comprendere la natura dell'errore.
\end{itemize}
Questi dettagli includono il codice d'errore specifico di Zod (come ``\texttt{too\_small}'', ``\texttt{invalid\_string}'', ``\texttt{unrecognized\_keys}'', ecc.), il valore atteso e quello effettivamente ricevuto, e altri parametri che variano a seconda del tipo di errore (come i limiti minimi e massimi nel caso di valori numerici o stringhe con vincoli di lunghezza). Ancora una volta l'obiettivo è quello di produrre un’informazione chiara e strutturata, che possa essere facilmente interpretata dal frontend per la generazione di messaggi d’errore dettagliati relativi agli input non validi.

Il risultato di questa elaborazione viene quindi incluso nei \texttt{details} dell’errore con codice ``\texttt{validation.input\_error}'', accompagnato dall’entità coinvolta e dal tipo di operazione che ha generato la validazione fallita. In questo modo, il frontend, ricevendo un errore di questo tipo, è in grado di analizzare i dettagli forniti e costruire dinamicamente un messaggio specifico per ogni campo errato, mantenendo la piena compatibilità con il sistema di internazionalizzazione.

\subsubsection{Considerazioni finali}
La gestione degli errori nel backend è stata strutturata in maniera modulare e tipizzata, garantendo uniformità e scalabilità. L’uso dei \textit{generics} di TypeScript e della classe \texttt{AppError} ha permesso di standardizzare la rappresentazione delle anomalie, facilitando l’integrazione con il frontend.
%
Grazie alla mappatura definita in \texttt{ErrorDetailsMapping}, ogni errore include informazioni specifiche e contestualizzate, assicurando una comunicazione chiara tra i livelli del sistema. Inoltre, la gestione degli errori di validazione tramite Zod permette di trasformare le anomalie sugli input in messaggi strutturati e facilmente interpretabili.

Questa architettura garantisce che il backend restituisca risposte coerenti e prevedibili, semplificando la generazione dei messaggi d'errore lato client e migliorando la manutenibilità complessiva del sistema.

\subsection{Gestione degli errori nel frontend}
Nel frontend, la gestione delle anomalie provenienti dal backend è stata progettata per garantire una comunicazione efficace e multilingua nei messaggi destinati all’utente. Gli errori restituiti dai vari microservizi, come è già ben noto, non possono essere utilizzati direttamente per fornire un feedback comprensibile, poiché si limitano a rappresentare il problema in modo schematico. Per questo motivo, è necessario un sistema di interpretazione che analizzi le proprietà dell'oggetto errore e generi dinamicamente un messaggio esplicativo.

\paragraph{Definizione e utilizzo di \texttt{Translator}}
Per garantire un efficace supporto alla traduzione dei messaggi di errore, è stato definito il tipo \texttt{Translator}, che rappresenta una funzione generica in grado di convertire una chiave testuale e un insieme opzionale di parametri in una stringa localizzata. Questa astrazione consente di mantenere il sistema di gestione degli errori indipendente dalla libreria specifica utilizzata per la multilingua, rendendolo più flessibile ed estensibile. La definizione di \texttt{Translator} è riportata nel \Cref{lst:translator}.

Per integrare questo meccanismo nei componenti del frontend, è stato realizzato un apposito hook denominato \texttt{useTranslator}, che fornisce un'interfaccia uniforme per la traduzione dei messaggi, senza vincolare il codice a una particolare implementazione. Questo hook consente inoltre di specificare un prefisso per le chiavi di traduzione, facilitando l'organizzazione e la gestione delle stringhe localizzate.

\lstinputlisting[
  language=typescript,
  caption={Definizione del tipo \texttt{Translator} e dell'hook \texttt{useTranslator}},
label={lst:translator}]
{listings/useTranslator.ts}
%
Grazie a questa astrazione, ogni componente del frontend può ottenere facilmente un’istanza della funzione di traduzione senza dipendere direttamente da una libreria specifica. Questo approccio migliora la modularità e la manutenibilità del codice, permettendo di sostituire o aggiornare il sistema di internazionalizzazione senza modificare la logica di gestione degli errori.

\subsubsection{Costruzione dinamica dei messaggi di errore}
Nel frontend, ogni volta che viene effettuata una chiamata HTTP al backend, se la risposta ricevuta ha un codice di stato HTTP diverso da \texttt{2XX}, il sistema sa che si è verificato un errore e attiva il meccanismo di generazione del messaggio corrispondente. Per fare ciò, viene analizzato l’oggetto di tipo \texttt{AppErrorPayload} che rappresenta il risultato dell'invocazione di \texttt{toJSON()} da parte del backend su un'istanza di \texttt{AppError}.

Un primo livello di gestione è implementato attraverso l'invocazione del metodo \texttt{parseBackendError()}, che, sulla base del codice di stato HTTP ricevuto, chiama metodi specifici per la costruzione del messaggio di errore appropriato. In particolare, esso opera come segue:
\begin{itemize}
  \item Se lo stato HTTP è \texttt{400}, si tratta di un errore di validazione. In questo caso, il metodo estrae i dettagli dell'errore dalla risposta del backend e, se il codice restituito è ``\texttt{validation.input\_error}'', accede alla proprietà \texttt{inputErrors} all'interno dei \texttt{details}. Questo array contiene l’elenco dei campi che non hanno superato la validazione e le relative anomalie. Successivamente, il metodo \texttt{getValidationErrorMessage()} elabora queste informazioni per generare un messaggio chiaro e contestualizzato per l’utente.
  \item Se lo stato HTTP è \texttt{409}, si tratta di un errore di conflitto, tipicamente legato a operazioni fallite su entità esistenti. In questo scenario, il metodo analizza il codice del problema, estraendo il nome dell’entità coinvolta e il tipo di errore. Successivamente, invoca metodi specifici come \texttt{getUserErrorMessage()}, che costruiscono il messaggio contestualizzato all'entità coinvolta.
  \item Se lo stato HTTP è \texttt{500} o un altro valore non gestito, il metodo restituisce un messaggio generico di errore, indicando all’utente che si è verificato un problema imprevisto. Questo è l'unico caso in cui i messaggi d'errore vengono costruiti lato backend, in quanto non è possibile prevedere tutti i possibili scenari. Questi messaggi sono in lingua inglese e non vengono tradotti.
\end{itemize}

\subsubsection{Gestione degli errori relativi alle entità}
Nel frontend, la generazione dei messaggi d'errore per le diverse entità del sistema, come utenti e organizzazioni, avviene attraverso metodi specifici per ciascuna di esse. Ogni entità ha infatti caratteristiche e dettagli di errore peculiari, che richiedono un'elaborazione dedicata per costruire un messaggio chiaro e informativo per l'utente.

Ogni metodo di gestione degli errori riceve tre parametri principali: il tipo di errore associato all'entità coinvolta, i dettagli forniti dal backend e la funzione di traduzione. I dettagli dell’errore contengono informazioni strutturate, come l’identificativo dell’entità o la causa specifica del problema. Sulla base di questi dati, il metodo elabora dinamicamente il messaggio da restituire.

A seconda del tipo di errore, il messaggio viene costruito in modi diversi. Ad esempio, per un errore di creazione, la funzione di gestione verifica se la causa è una violazione di unicità o un problema di integrità referenziale e genera un messaggio che spiega all’utente il motivo del fallimento. In caso di un’entità non trovata, il messaggio include l’identificativo cercato, offrendo un'indicazione chiara sulla natura del problema. Se invece l’errore è legato a una modifica o eliminazione non riuscita, il messaggio può evidenziare l’operazione specifica che non è stata eseguita con successo.

Questa suddivisione in metodi specifici per ogni entità permette di mantenere il codice organizzato e facilmente estendibile. Nuovi tipi di errore possono essere aggiunti senza modificare la logica generale, assicurando che il sistema di gestione delle anomalie rimanga flessibile e scalabile.

\subsubsection{Gestione degli errori di validazione}
Come descritto nella \Cref{sec:input-validation-errors}, gli errori di validazione degli input vengono strutturati in modo da fornire tutte le informazioni necessarie per costruire un messaggio dettagliato. Nel frontend, il metodo \texttt{getValidationErrorMessage()} di seguito riportato, si occupa di elaborare questi dati per generare una stringa comprensibile e localizzata da mostrare all’utente.
%
\lstinputlisting[
  language=typescript,
  caption={Metodo \texttt{getValidationErrorMessage()}},
label={lst:getValidationErrorMessage}]
{listings/validationMessage.ts}
%
Il metodo analizza gli errori ricevuti, associando ogni campo con i problemi segnalati dal backend. Per ciascuna anomalia, viene costruito un messaggio traducendo dinamicamente il codice d’errore di Zod (ad esempio, ``\texttt{too\_small}'' o ``\texttt{invalid\_string}'') e integrando informazioni come il valore atteso, quello ricevuto e altri dettagli rilevanti. Infine, i vari messaggi vengono concatenati in un'unica stringa strutturata, così da fornire un’indicazione chiara sui problemi riscontrati e facilitare la correzione dei dati inseriti.

\subsubsection{Considerazioni finali}
Il sistema di gestione degli errori nel frontend è stato progettato per garantire scalabilità, modularità e supporto alla multilingua. Una delle sue caratteristiche principali è la separazione tra la logica di interpretazione degli errori e la libreria di internazionalizzazione utilizzata, resa possibile dall’uso del tipo astratto \texttt{Translator}. Questo approccio consente di adattare facilmente il sistema a diverse soluzioni di traduzione, senza influire sulla logica di elaborazione degli errori.

Oltre a questa astrazione, il frontend gestisce gli errori attraverso metodi dedicati che costruiscono dinamicamente i messaggi sulla base del tipo di errore ricevuto. Gli errori di validazione vengono elaborati per restituire un feedback dettagliato sui campi non conformi, mentre quelli relativi alle entità, come utenti e organizzazioni, vengono interpretati in base alla loro specificità, includendo informazioni rilevanti per l’utente finale.

Grazie a questa architettura, il frontend è in grado di convertire gli errori restituiti dal backend in messaggi informativi chiari e contestualizzati, migliorando l’esperienza utente e garantendo un’interazione fluida con il sistema. La modularità del sistema facilita inoltre l’estensione della gestione degli errori, permettendo di integrare nuove casistiche senza modificare la logica esistente.
