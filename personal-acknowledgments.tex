\begin{acknowledgements}
  Che emozione essere qui a scrivere questi ringraziamenti.
  %
  Non immaginate quanto sia glorioso per me questo momento, quanto io lo abbia desiderato e quanto impegnativo sia stato raggiungerlo.

  Per me questo traguardo significa molto più di un semplice ``oh che bello, finalmente mi sono laureato''.
  %
  Significa aver vinto per la seconda volta contro le difficoltà, le insicurezze, contro i momenti di crisi.
  %
  Significa aver superato innumerevoli esami, che all'inizio mi sembravano insormontabili, con un voto ben più alto di ogni mia aspettativa.
  %
  Per non parlare della scrittura di questa tesi, operazione che ho sempre temuto e rimandato all'ultimo momento.
  %
  E ancora, significa aver combattuto con l'onnipresente pensiero di essere ancora uno studente, cosa che negli ultimi mesi mi stava consumando dentro.
  %
  Ora, però, posso solo dire di avercela fatta e di essere meravigliosamente contento del risultato finale!
  %
  Pur ammettendo che l'averci impiegato così tanto tempo, mi ha fatto soffrire non poco.

  Adesso però basta di fare tutti questi giri di parole, altrimenti rischio che qualcuno mi rovini le pagine con le sue lacrime!
  %
  È giunto il momento di presentare e ringraziare tutte le persone che hanno reso tutto ciò possibile e che mi hanno accompagnato fino a qui (e spero anche oltre!).

  Non posso che iniziare con la mia famiglia, senza la quale non avrei nemmeno intrapreso questo grande  e significativo percorso! Grazie \emph{mamma} e grazie \emph{babbo} per avermi sempre trasmesso l'importanza dello studio, cosa che mi ha continuamente spronato a considerare la rinuncia agli studi come un'opzione assolutamente impraticabile! Grazie anche per avermi coccolato durante i lunghi anni di studio, preparandomi da mangiare, lavandomi i vestiti, scarrozzandomi qua e là, e standomi vicino nei momenti più difficili. Qui un ringraziamento speciale va pure a mio fratello, che non vive più con noi oramai da anni, ma non perde occasione di unirsi a noi nei momenti di ritrovo familiare! Grazie \emph{Semmi} per il sostegno che mi hai dato, per le chiacchierate su quelle cose che solo con te posso affrontare e in generale per il tuo spirito di felicità ed entusiasmo che trasmetti in ogni momento! Sei stato fondamentale per consentirmi di arrivare fin qui, in particolare, c'è stata una frase che mi dicevi spesso, la quale mi ha fatto aprire gli occhi sul fatto che probabilmente stavo dedicando poco tempo allo studio. Spesso, prima del fine settimana, eri solito a dirmi: ``Allora domani inizia il weekend, eh?!''. Ti confesso che un po' mi dava fastidio perché mi rendevo conto che effettivamente il fine settimana non lo dedicavo mai allo studio, il che aveva il negativo effetto di allontanare da me il giorno della laurea. In breve, mi hai dato un motivo per \textit{pompare nero} e terminare gli studi più in fretta!

  Le prossime righe le voglio dedicare alla mia amata \emph{Lallina}, anche lei una presenza essenziale in questo mio percorso e ancor di più nella mia vita! Grazie di cuore (rigorosamente arancione) per avermi accompagnato per più della metà del mio ``viaggio'' universitario! Grazie, sei stata la persona che più mi è stata vicina e che più mi ha dato conforto negli innumerevoli momenti di crisi! Grazie per tutto l'affetto e l'amore che mi dai, per i bellissimi momenti passati insieme e per tutte le prime esperienze che ho passato, e che passerò, con te! Un grazie speciale \emph{Amore}, che sei più entusiasta e carica del sottoscritto per questo mio traguardo!

  Vorrei concludere questi cari ringraziamenti citando i miei amici e i compagni di corso che hanno giocato un ruolo molto importante nella riuscita di questo percorso.
  %
  In particolare, ringrazio \emph{Edoardo} e \emph{Giorgio}, due persone che conosco da poco ma che si stanno rivelando grandi amici! Vi ringrazio per le serate insieme, per le risate e per aver sempre creduto in me! Grazie anche ai miei amici di lunga data \emph{Andrea}, \emph{Gianmarco} e \emph{Thomas}, con i quali non ho mai smesso di uscire fin dai primi anni delle scuole medie! In ultimo, grazie a quei compagni di corso con i quali ho superato la maggior parte dei famigerati esami e, senza i quali, oggi non sarei qui con lo stesso spirito. E quindi grazie \emph{Lorenzo}, \emph{Alex}, \emph{Andrea}, \emph{Marco}, \emph{Fabio} e \emph{Michele} per il vostro prezioso contributo. Spero che anche da parte vostra valga la stessa cosa!

  Ed eccomi qui, ancora una volta alla conclusione della tesi, che segna la fine di un lungo e duro percorso e apre la strada al mio futuro, in cui potrò finalmente considerarmi un vero e proprio \textit{Ingegnere}!
\end{acknowledgements}