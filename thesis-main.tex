\documentclass[12pt,a4paper,openright,twoside]{book}
\usepackage[utf8]{inputenc}
\usepackage{disi-thesis}
\usepackage{code-lstlistings}
\usepackage{notes}
\usepackage{shortcuts}
\usepackage{acronym}

\school{\unibo}
\programme{Corso di Laurea Magistrale in Ingegneria e Scienze Informatiche}
\title{Progettazione e Reingegnerizzazione di un Sistema Legacy per il DNS Filtering}
\author{Alberto Spadoni}
\date{\today}
\subject{Paradigmi di Programmazione e Sviluppo}
\supervisor{Prof. Mirko Viroli}
\cosupervisor{Dott. Nicolas Farabegoli}
\morecosupervisor{Dott. Gianluca Aguzzi}
\session{IV}
\academicyear{2023-2024}

% Definition of acronyms
\acrodef{IoT}{Internet of Thing}
\acrodef{vm}[VM]{Virtual Machine}

\mainlinespacing{1.241} % line spacing in mainmatter, comment to default (1)

\begin{document}

\frontmatter\frontispiece

\begin{abstract}
  Max 2000 characters, strict.
\end{abstract}

\begin{dedication} % this is optional
  Alla mia famiglia ed a Lara
\end{dedication}

%----------------------------------------------------------------------------------------
\tableofcontents
% \listoffigures     % (optional) comment if empty
% \lstlistoflistings % (optional) comment if empty
%----------------------------------------------------------------------------------------

\mainmatter

%----------------------------------------------------------------------------------------
\chapter{Introduction}
\label{chap:introduction}
%----------------------------------------------------------------------------------------

Write your intro here.
\sidenote{Add sidenotes in this way. They are named after the author of the thesis}

You can use acronyms that your defined previously,
such as \ac{IoT}.
%
If you use acronyms twice,
they will be written in full only once
(indeed, you can mention the \ac{IoT} now without it being fully explained).
%
In some cases, you may need a plural form of the acronym.
%
For instance,
that you are discussing \acp{vm},
you may need both \ac{vm} and \acp{vm}.

\paragraph{Structure of the Thesis}

\note{At the end, describe the structure of the paper}

\chapter{Stato dell'Arte}

\section{Introduzione al DNS e al filtraggio}
Prima di trattare l'argomento cardine del presente capitolo, si ritiene opportuno fare una breve panoramica sui concetti importanti ad esso collegati. Verrà in prima batttuta presentato il Domain Name System (DNS), che può essere definito come uno dei pilastri fondamentali di tutta l'architettura della rete Internet. Successivamente, ci si sposterà sull'ambito del filtraggio in Internet, che rappresenta il contesto più ampio di cui il filtraggio DNS fa parte.

\subsection{Cos'è il DNS e il suo ruolo in Internet}
Il Domain Name System è un database gerarchico e distribuito che contiene le associazioni tra nomi di dominio ed altre importanti informazioni, tra cui gli indirizzi IP.

Questo fondamentale sistema consente agli utenti di localizzare le risorse sulla rete andando a convertire nomi di dominio familiari ed in formato leggibile dagli umani in indirizzi numerici ai quali un computer può connettersi. Un'analogia comune che si uutilizza per spiegare il ruolo dei sistema DNS è che esso serve da rubrica telefonica per Internet, andando a tradurre i nomi di computer comprensibili agli umani nei relativi indirizzi numerici interpretabili dalle macchine. Per fare un esempio, il nome di dominio \texttt{www.airbus.com} viene tradotto dal DNS nell'indirizzo IPv4 \texttt{107.154.76.155}.

\subsubsection{Vulnerabilità del DNS e necessità di filtraggio}
Il DNS rappresenta una porzione cruciale della rete Internet e per questo motivo la sua messa in sicurezza risulta molto importante. Infatti, se un individuo malintenzionato dovesse riuscire a comprometterlo, sarebbe in grado di bloccare, o comunque ridurre, le normali attività che avvengono sulla rete.

Il sistema in oggetto è stato progettato negli anni '80 per rispondere alla necessità di una risoluzione dei nomi rapida e scalabile su una rete in continua espansione. Al momento della sua concezione, come descritto nelle specifiche originarie RFC 1034 \cite{rfc1034} e RFC 1035 \cite{rfc1035}, l'attenzione era principalmente concentrata sulla funzionalità e sull'efficienza, senza considerare i potenziali problemi di sicurezza che sarebbero emersi con la crescita esponenziale di Internet. Oltretutto, la solida fiducia che le specifiche RFC trasmettevano ai professionisti IT dell’epoca non li spinse a preoccuparsi o a indagare sui potenziali rischi di sicurezza che tale sistema poteva comportare \cite{hudaib2014dns}. Questa scelta rifletteva il contesto storico in cui è nato il DNS: la rete era allora utilizzata principalmente da enti accademici e governativi, con un livello di fiducia reciproca tra i partecipanti. Tuttavia, con l'apertura di Internet a un pubblico globale, il DNS ha rivelato vulnerabilità intrinseche, tra cui la mancanza di meccanismi nativi che garantiscano l'autenticazione delle risposte e l'integrità delle informazioni da esso fornite. Tali lacune hanno reso possibile una serie di attacchi che sfruttano le debolezze proprie del sistema DNS. Tra questi, si annoverano il DNS Spoofing, gli attacchi di tipo Distributed Denial of Service (DDoS) nei confronti dell'infrastruttura DNS e il DNS Hijacking.

\paragraph{DNS Spoofing}
Il DNS spoofing, noto anche come Cache Poisoning, consiste nell'iniettare dati malevoli nella cache dei server DNS bersaglio, inducendoli a restituire informazioni errate agli utenti. Questo attacco permette ai malintenzionati di reindirizzare il traffico verso siti controllati da essi, facilitando il furto di credenziali o altre forme di attacchi avanzati. Un esempio storico di DNS spoofing è l'attacco di Eugene Kashpureff del 1997, il quale riuscì a reindirizzare tutti i visitatori del dominio \texttt{internic.net} verso il sito della compagnia Alternic, di cui era il fondatore \cite{lioy2000dns}. In generale, questa tipologia di attacco sfrutta la mancanza di autenticazione nelle risposte DNS e l'assenza di integrità nelle informazioni memorizzate nella cache. I metodi per mitigare il DNS spoofing includono sostanzialmente l'implementazione di DNSSEC, che garantisce l'autenticità delle risposte attraverso firme digitali \cite{rfc2535}.

\paragraph{Attacchi di amplificazione DNS}
Gli attacchi Distributed Denial of Service (DDoS) rappresentano una minaccia critica e persistente per la sicurezza informatica. In generale, essi mirano a compromettere la disponibilità dei servizi di un sistema bersaglio, costringendolo a un riavvio forzato o esaurendone le risorse, come cicli della CPU, memoria RAM o larghezza di banda di rete. A conseguenza di ciò, il sistema attaccato diventa incapace di fornire i servizi previsti agli utenti legittimi.

Tra le più comuni applicazioni del DDoS nei confronti dell'infrastruttura DNS è possibile trovare i cosiddetti DNS Amplification Attacks, in cui un attaccante invia richieste al Domain Name System con un indirizzo IP sorgente falsificato, facendole apparire come provenienti dalla vittima. A questo punto, i server DNS rispondono con pacchetti di dimensioni molto superiori rispetto alla richiesta iniziale, amplificando così il traffico diretto alla vittima.
%
Il principio alla base di questa tecnica risiede nella capacità degli attaccanti di sfruttare la differenza tra la dimensione delle richieste e quella delle risposte. Ad esempio, una richiesta per un grande file di zona DNS, con l'indirizzo IP sorgente falsificato, viene inviata a un numero significativo di server DNS pubblici. Questi, ignari della natura fraudolenta delle richieste, rispondono inviando i dati direttamente alla vittima designata. La differenza di dimensioni tra la richiesta, tipicamente piccola, e la risposta, significativamente più grande, permette agli attaccanti di moltiplicare il volume di traffico generato, compromettendo gravemente il sistema bersaglio \cite{Alieyan2016}.

Un esempio di amplificazione DNS è quello che ha colpito Spamhaus nel 2013, considerato uno dei più grandi attacchi DDoS mai accaduti. Esso è stato caratterizzato da una richiesta di 36 byte che ha generato una risposta di 3.000 byte, amplificando il traffico di un fattore 100 e generando un volume di dati in entrata ai server della compagnia pari a 75GBps \cite{Bonasera2021}.

\paragraph{DNS Hijacking}
Il DNS hijacking prevede la compromissione di server DNS o la manipolazione delle configurazioni DNS di un utente per reindirizzare il traffico verso destinazioni controllate dall'attaccante. Questo attacco può essere implementato attraverso malware che modifica le impostazioni DNS locali o mediante la compromissione diretta dei server DNS \cite{hudaib2014dns}.

Le conseguenze includono il furto di credenziali, la diffusione di malware e la censura di contenuti web. Alcuni provider di servizi Internet (ISP) utilizzano questa tecnica per scopi commerciali, come la visualizzazione di pubblicità. Per mitigare il DNS hijacking, si consiglia di utilizzare configurazioni DNS sicure, abilitare DNSSEC e controllare in maniera scrupolosa le modifiche ai record DNS \cite{hudaib2014dns}.

\paragraph{Necessità di un filtraggio DNS}
Per affrontare queste problematiche, il filtraggio DNS rappresenta una soluzione efficace e scalabile. Oltre a bloccare domini malevoli, consente di ottimizzare l'uso delle risorse di rete e garantire conformità normativa. Tuttavia, la sua implementazione richiede un bilanciamento tra sicurezza e accessibilità, minimizzando l'impatto sulle prestazioni.

\subsection{Il filtraggio Internet e il posizionamento del DNS filtering}


\chapter{Contribution}

You may also put some code snippet (which is NOT float by default), eg: \cref{lst:random-code}.

\lstinputlisting[float,language=Java,label={lst:random-code}]{listings/HelloWorld.java}

\section{Fancy formulas here}

%----------------------------------------------------------------------------------------
% BIBLIOGRAPHY
%----------------------------------------------------------------------------------------

\backmatter

\nocite{*} % Remove this as soon as you have the first citation

\bibliographystyle{alpha}
\bibliography{bibliography}

% \begin{acknowledgements} % this is optional
% Optional. Max 1 page.
% \end{acknowledgements}

\end{document}
