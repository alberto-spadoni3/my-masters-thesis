\documentclass[12pt,a4paper,openright,twoside]{book}
\usepackage[utf8]{inputenc}
\usepackage{disi-thesis}
\usepackage{code-lstlistings}
\usepackage{notes}
\usepackage{shortcuts}
\usepackage{acronym}
\usepackage{rotating}
\usepackage[nobiblatex]{xurl}

\school{\unibo}
\department{\disi}
\programme{Corso di Laurea Magistrale in Ingegneria e Scienze Informatiche}
\title{Processo di Transizione e Reingegnerizzazione di un Pannello Web per DNS Filtering}
\author{Alberto Spadoni}
\date{\today}
\subject{Paradigmi di Programmazione e Sviluppo}
\supervisor{Prof. Mirko Viroli}
\cosupervisor{Dott. Nicolas Farabegoli}
\morecosupervisor{Dott. Gianluca Aguzzi}
\session{IV}
\academicyear{2023-2024}

\mainlinespacing{1.241} % line spacing in mainmatter, comment to default (1)

\begin{document}

\frontmatter\frontispiece

\renewcommand{\abstractname}{Abstract} % change the abstract title from italian to english

\begin{abstract}
  Negli ultimi anni, le minacce informatiche sono diventate sempre più diffuse, con la maggior parte degli attacchi che avvengono attraverso Internet. Per mitigare questi rischi, le aziende adottano diverse strategie di sicurezza, tra cui il filtraggio DNS, una pratica efficace per bloccare accessi a domini malevoli e applicare politiche di sicurezza a livello di rete. Questo approccio risulta particolarmente strategico in quanto opera a un livello intermedio tra gli endpoint e le applicazioni, bilanciando efficacia, semplicità di implementazione e limitato impatto sulle risorse di sistema.

  \medskip

  In questo contesto, la presente tesi si concentra sulla reingegnerizzazione del pannello per la configurazione e gestione del filtro DNS di \textit{FlashStart Group srl}. L'obiettivo è la progettazione e l’implementazione di una nuova infrastruttura moderna e scalabile, basata su un’architettura a microservizi. Il lavoro ha riguardato la riprogettazione del database, l’introduzione di una dashboard avanzata per MSP e la definizione di un sistema strutturato per la gestione degli errori.

  \medskip

  L’approccio adottato ha seguito una strategia di reingegnerizzazione ibrida, ossia un mix di sviluppo incrementale e sostituzione totale, evitando così la coesistenza tra vecchio e nuovo sistema. Sono state implementate le prime funzionalità chiave, mentre ulteriori sviluppi riguarderanno l’integrazione di un sofisticato sistema di controllo degli accessi, la completa documentazione del sistema e l’ottimizzazione dell’interfaccia utente.

  \medskip

  I risultati ottenuti finora evidenziano le potenzialità del nuovo sistema, che supera la versione legacy in termini di usabilità, sicurezza e manutenibilità. Questo progetto rappresenta un primo passo verso una piattaforma più efficiente, orientata alle esigenze di un mercato enterprise. L'architettura rinnovata costituisce una solida base per sviluppi futuri, consentendo all’azienda di rafforzare la propria posizione competitiva in un settore in rapida evoluzione.
\end{abstract}

\begin{dedication}
  Alla mia famiglia, a Lara ed al mio futuro
\end{dedication}

%----------------------------------------------------------------------------------------
\tableofcontents
\listoffigures
%----------------------------------------------------------------------------------------

\mainmatter

\chapter{Introduzione}

La presente tesi nasce dall’esperienza di tirocinio svolta presso un’azienda specializzata nello sviluppo di soluzioni per il filtraggio DNS. Durante tale esperienza, è stato possibile contribuire alle prime fasi di reingegnerizzazione del pannello Web per la configurazione di un filtro DNS. Questo pannello, concepito per offrire funzionalità di gestione e controllo dei domini da filtrare, presentava tuttavia limiti strutturali e tecnologici che ne ostacolavano la manutenibilità e l’estensibilità. L’esigenza di modernizzare l’architettura, migliorare l’esperienza utente e garantire maggiore sicurezza è nata anche dalla volontà di posizionare l’azienda in un mercato più orientato alle medie-grandi organizzazioni. Questo obiettivo mira a favorirne la crescita e a consolidarne il ruolo come punto di riferimento a livello globale nel settore del filtraggio DNS.

Il presente lavoro si inserisce in questo contesto, con l’obiettivo di progettare e sviluppare una nuova versione del pannello di configurazione del filtro DNS aziendale, adottando un’architettura moderna basata su microservizi. Questo approccio mira a garantire maggiore efficienza, sicurezza e manutenibilità rispetto alla versione precedente, superando le rigidità del legacy e introducendo nuove funzionalità essenziali.

Scopo dell’elaborato è fornire una visione organica del processo di reingegnerizzazione svolto, illustrando le motivazioni alla base della necessità di aggiornare il pannello, i principali obiettivi progettuali e i risultati raggiunti nelle prime fasi di sviluppo.

Il contesto teorico di riferimento comprende sia il Domain Name System (DNS), con le sue logiche di funzionamento e la sua importanza nella gestione del traffico in rete, sia le metodologie di reingegnerizzazione software, che offrono linee guida per affrontare in modo sistematico la modernizzazione di soluzioni esistenti.

Dal punto di vista metodologico, il progetto ha richiesto un'attenta fase di analisi, seguita dalla definizione di un’architettura più flessibile e scalabile, fino all’implementazione di un primo set di funzionalità.

Durante lo sviluppo, sono state affrontate diverse sfide, tra cui la transizione da un’architettura monolitica a una basata su microservizi, la riprogettazione della base dati e la creazione di un sistema avanzato di gestione degli errori, in grado di operare in modo strutturato e pervasivo su tutti i livelli del sistema.

\section{Struttura dell'elaborato}
L'elaborato è suddiviso nei seguenti capitoli, ciascuno dedicato a un aspetto specifico del progetto:

\begin{itemize}
  \item \textbf{Capitolo 2: Background} – Introduce il funzionamento del DNS e il filtraggio DNS, oltre alle metodologie di reingegnerizzazione software.
  \item \textbf{Capitolo 3: Analisi} – Esamina il sistema legacy, evidenziandone l’architettura, le tecnologie adottate e le limitazioni riscontrate. Inoltre, introduce i requisiti del nuovo sistema e la modellazione del dominio.
  \item \textbf{Capitolo 4: Design} – Presenta la nuova architettura proposta, sia lato frontend che backend, illustrando le interazioni tra i componenti e le principali scelte tecnologiche adottate.
  \item \textbf{Capitolo 5: Implementazione} – Descrive il processo di sviluppo del nuovo sistema, soffermandosi sull’infrastruttura, sull’organizzazione della repository e sull'implementazione del sistema di gestione degli errori.
  \item \textbf{Capitolo 6: Valutazione} – Analizza i requisiti soddisfatti e le funzionalità implementate, confrontandole con gli obiettivi iniziali. Inoltre, discute i miglioramenti pianificati e gli aspetti ancora da sviluppare.
  \item \textbf{Capitolo 7: Conclusione} – Riassume il lavoro svolto, evidenziando le principali lezioni apprese e proponendo possibili sviluppi futuri del sistema.
\end{itemize}

Grazie a questa suddivisione, l’elaborato mira a fornire una panoramica completa del percorso di reingegnerizzazione del sistema, dall’analisi dei requisiti iniziali fino alle valutazioni finali e ai suggerimenti per successivi sviluppi. L’obiettivo ultimo è offrire un contributo concreto alla modernizzazione del software, ponendo le basi per futuri interventi di mantenimento e potenziamento del sistema.

\chapter{Background}

\section{Introduzione al DNS e al filtraggio}
Prima di trattare l'argomento cardine del presente capitolo, si ritiene opportuno fare una breve panoramica sui concetti importanti ad esso collegati. Verrà in prima batttuta presentato il Domain Name System (DNS), che può essere definito come uno dei pilastri fondamentali di tutta l'architettura della rete Internet. Successivamente, ci si sposterà sull'ambito del filtraggio in Internet, che rappresenta il contesto più ampio di cui il filtraggio DNS fa parte.

\subsection{Cos'è il DNS e il suo ruolo in Internet}
Il Domain Name System è un database gerarchico e distribuito che contiene le associazioni tra nomi di dominio ed altre importanti informazioni, tra cui gli indirizzi IP.

Questo fondamentale sistema consente agli utenti di localizzare le risorse sulla rete andando a convertire nomi di dominio familiari ed in formato leggibile dagli umani in indirizzi numerici ai quali un computer può connettersi. Un'analogia comune che si uutilizza per spiegare il ruolo dei sistema DNS è che esso serve da rubrica telefonica per Internet, andando a tradurre i nomi di computer comprensibili agli umani nei relativi indirizzi numerici interpretabili dalle macchine. Per fare un esempio, il nome di dominio \texttt{www.airbus.com} viene tradotto dal DNS nell'indirizzo IPv4 \texttt{107.154.76.155}.

\subsubsection{Vulnerabilità del DNS}
Il DNS rappresenta una porzione cruciale della rete Internet e per questo motivo la sua messa in sicurezza risulta molto importante. Infatti, se un individuo malintenzionato dovesse riuscire a comprometterlo, sarebbe in grado di bloccare, o comunque ridurre, le normali attività che avvengono sulla rete.

Il sistema in oggetto è stato progettato negli anni '80 per rispondere alla necessità di una risoluzione dei nomi rapida e scalabile su una rete in continua espansione. Al momento della sua concezione, come descritto nelle specifiche originarie RFC 1034 \cite{DBLP:journals/rfc/rfc1034} e RFC 1035 \cite{DBLP:journals/rfc/rfc1035}, l'attenzione era principalmente concentrata sulla funzionalità e sull'efficienza, senza considerare i potenziali problemi di sicurezza che sarebbero emersi con la crescita esponenziale di Internet. Oltretutto, la solida fiducia che le specifiche RFC trasmettevano ai professionisti IT dell’epoca li ha portati a trascurare o a non approfondire i potenziali rischi di sicurezza associati a tale sistema \cite{hudaib2014dns}. Questa scelta rifletteva il contesto storico in cui è nato il DNS: la rete era allora utilizzata principalmente da enti accademici e governativi, con un livello di fiducia reciproca tra i partecipanti. Tuttavia, con l'apertura di Internet a un pubblico globale, il DNS ha rivelato vulnerabilità intrinseche, tra cui la mancanza di meccanismi nativi che garantiscano l'autenticazione delle risposte e l'integrità delle informazioni da esso fornite. Tali lacune hanno reso possibile una serie di attacchi che sfruttano le debolezze intrinseche del sistema DNS. Tra questi, figurano il DNS Spoofing, il DNS Amplification Attack e il DNS Hijacking.

\paragraph{DNS Spoofing}\label{paragraph:dns_spoofing}
Il DNS spoofing, noto anche come Cache Poisoning, consiste nell'iniettare dati malevoli nella cache dei server DNS bersaglio, inducendoli a restituire informazioni errate agli utenti. Questo attacco permette ai malintenzionati di reindirizzare il traffico verso siti controllati da essi, facilitando il furto di credenziali o altre forme di attacchi avanzati. Un esempio storico di DNS spoofing è l'attacco di Eugene Kashpureff del 1997, il quale riuscì a reindirizzare tutti i visitatori del dominio \texttt{internic.net} verso il sito della compagnia Alternic, di cui era il fondatore \cite{lioy2000dns}.

In generale, questa tipologia di attacco sfrutta la mancanza di autenticazione nelle risposte DNS e l'assenza di integrità nelle informazioni memorizzate nella cache. I metodi per mitigare il DNS spoofing includono sostanzialmente l'implementazione di DNSSEC, che garantisce l'autenticità delle risposte attraverso firme digitali \cite{DBLP:journals/rfc/rfc2535}.

\paragraph{DNS Amplification Attacks}
Gli attacchi di tipo Distributed Denial of Service (DDoS) rappresentano una delle minacce più critiche per la sicurezza informatica, mirate a compromettere la disponibilità di un sistema. Questi attacchi si basano sull'invio massivo di richieste a un bersaglio per esaurirne le risorse, come CPU, memoria o larghezza di banda, rendendo il servizio inaccessibile agli utenti legittimi.

Un esempio rilevante nell'ambito del DNS è rappresentato dagli attacchi di amplificazione, una tipologia particolarmente efficace di DDoS. In questi attacchi, l'aggressore invia richieste DNS utilizzando un indirizzo IP sorgente falsificato, facendole sembrare provenienti dalla vittima. I server DNS, ingannati, rispondono con pacchetti significativamente più grandi delle richieste originali, amplificando così il volume del traffico diretto alla vittima. Sfruttando la differenza tra la dimensione delle richieste e delle risposte, gli attaccanti sono in grado di moltiplicare il traffico generato, saturando rapidamente le risorse della vittima e compromettendone il funzionamento \cite{DBLP:conf/ictc/AlieyanKARA16}.

Per difendersi dagli attacchi di amplificazione DNS, i fornitori di servizi Internet possono limitare l'accesso al proprio server DNS solamente agli utenti autorizzati, monitorare il traffico in entrata per individuare anomalie, e impostare dei limiti alla quantità di dati che il server DNS può inviare in risposta a una query.

Un caso pratico di amplificazione DNS è quello che ha colpito Spamhaus nel 2013, considerato uno dei più grandi attacchi DDoS mai accaduti. Esso è stato caratterizzato da una richiesta di 36 byte che ha generato una risposta di 3.000 byte, amplificando il traffico di un fattore 100 e generando un volume di dati in entrata ai server della compagnia pari a 75GBps \cite{Bonasera2021}.

\paragraph{DNS Hijacking}
Il DNS hijacking prevede la compromissione di server DNS o la manipolazione delle configurazioni DNS di un utente per reindirizzare il traffico verso destinazioni controllate da un malintenzionato. Questo tipo di attacco può essere implementato seguendo due strategie \cite{hudaib2014dns}:
\begin{itemize}
  \item attraverso l'uso di specifici malware che modificano le impostazioni DNS locali, oppure
  \item mediante la compromissione diretta dei server DNS fidati, in modo che non si comportino come da specifica.
\end{itemize}

Le conseguenze di questo tipo di attacco includono il furto di credenziali, la diffusione di malware e la censura di contenuti web. Tuttavia, alcuni provider di servizi Internet (ISP) utilizzano questa tecnica per scopi commerciali, come la visualizzazione di pubblicità o la raccolta di statistiche sulla navigazione dei propri clienti. Per mitigare il DNS hijacking, si consiglia di utilizzare configurazioni DNS sicure e controllare in maniera scrupolosa le modifiche ai record DNS locali \cite{hudaib2014dns}.

\subsection{Il filtraggio di Internet}
A seguito dell'analisi riguardante le vulnerabilità intrinseche del sistema DNS, emerge chiaramente l'importanza di adottare meccanismi di prevenzione per garantire una navigazione più sicura e controllata. A questo proposito, una soluzione interessante riguarda il filtraggio di Internet, che, in generale, rappresenta un insieme di tecniche e tecnologie volte a limitare o bloccare l'accesso a contenuti considerati inappropriati, illegali o dannosi. Le principali pratiche di filtraggio nell'ambito di Internet si dividono nelle tre seguenti tipologie:
\begin{itemize}
  \item \textbf{IP Packet Filtering}: ovvero il blocco del traffico basato su indirizzi IP o numeri di porta. Questa tecnica analizza gli header dei pacchetti per decidere se consentirne il passaggio o eliminarli. Sebbene sia semplice ed efficiente, può causare sovrafiltraggio, rendendo inaccessibili tutti quei contenuti legittimi ospitati su un indirizzo IP considerato malevolo;

  \item \textbf{DNS Poisoning}: consiste nella manipolazione delle risposte DNS per reindirizzare richieste a indirizzi errati o per bloccarle completamente. Sebbene sia nata come una pratica malevola (\Cref{paragraph:dns_spoofing}), i suoi principi tecnici sono stati adattati ed applicati ad un contesto legittimo di filtraggio;

  \item \textbf{URL Blocking}: implementa il blocco di URL specifici tramite proxy HTTP o tecniche come la Deep Packet Inspection (DPI). Questi sistemi analizzano il contenuto delle richieste HTTP per identificare e bloccare pagine o risorse specifiche, offrendo una maggiore precisione rispetto ai metodi sopra elencati. L'unico svantaggio di quest'ultimo approccio deriva dal fatto che risulta più complesse da implementare e richiede una maggiore potenza di calcolo rispetto alle altre tecniche.
\end{itemize}

\subsubsection{Punti di applicazione del filtraggio di Internet}
Il filtraggio di Internet può essere implementato in diversi livelli dell'architettura di rete, con implicazioni specifiche in termini di efficacia, costi e complessità. Secondo V. Varadharajan \cite{DBLP:journals/ieeesp/Varadharajan10}, i principali punti di applicazione includono:

\begin{itemize}
  \item \textbf{A monte dell'ISP}: i filtri possono essere posizionati nei fornitori internazionali di connettività o presso le dorsali di rete. Questo approccio consente di bloccare contenuti indesiderati prima che raggiungano l'ISP locale, risultando particolarmente utile per gestire il traffico proveniente da server internazionali. Tuttavia, questa strategia richiede una stretta collaborazione tra le entità coinvolte e una chiara mappatura dei gateway internazionali.

  \item \textbf{All'interno dell'ISP}: posizionare i filtri nel cuore della rete dell'ISP permette di monitorare e controllare il traffico a livello centralizzato per tutti i suoi utenti. Sebbene sia efficace, questa soluzione può introdurre colli di bottiglia e richiedere investimenti significativi in hardware sofisticato, soprattutto per ISP con un alto volume di traffico.

  \item \textbf{Tra l'ISP e il cliente}: applicare il filtraggio nel collegamento tra l'ISP e il cliente finale offre un controllo mirato sul traffico diretto verso reti locali o singoli utenti. Questa configurazione consente di personalizzare le policy di filtraggio in maniera più granulare rispetto ai precedenti punti di applicazione, ma può comportare costi operativi elevati per ISP con grandi numeri di clienti.

  \item \textbf{Tra due clienti dello stesso ISP}: in reti locali o interne a un ISP, i filtri possono essere configurati per impedire lo scambio di contenuti indesiderati tra utenti dello stesso provider. Questo approccio è utile in contesti aziendali o comunità chiuse.

  \item \textbf{Tra un cliente e un sito web ospitato a livello internazionale}: i filtri possono essere utilizzati per monitorare e bloccare il traffico tra un utente e server situati all'estero. Questa strategia è fondamentale per controllare contenuti provenienti da altri paesi, ma richiede un'infrastruttura adeguata e la capacità di gestire connessioni internazionali.
\end{itemize}

La possibilità di distribuire le tecniche di filtraggio su diversi livelli offre un'elevata flessibilità nell'implementazione delle strategie di controllo dei contenuti. Tuttavia, ogni livello si porta con sé alcuni compromessi: il posizionamento a monte può bloccare grandi volumi di traffico, ma con rischi di sovrafiltraggio, mentre i filtri localizzati vicino agli utenti offrono maggiore precisione ma richiedono risorse significative. Una combinazione di approcci, come quella messa in atto dalla British Telecom’s CleanFeed, può ottimizzare l'efficacia del filtraggio, bilanciando precisione e scalabilità \cite{DBLP:journals/ieeesp/Varadharajan10}.

\subsection{Il filtraggio DNS}
Tra le varie tecniche di Internet filtering descritte in precedenza, il filtraggio a livello DNS si distingue per la sua semplicità e versatilità. Al contrario di altri approcci, esso agisce direttamente sul sistema che costituisce la base della navigazione online. Questo lo rende un punto di controllo strategico ed efficace per implementare politiche di sicurezza e controllo dell'accesso. Oltretutto, i cybercriminali stanno sempre di più prendendo di mira l'infrastruttura DNS per condurre le loro attività malevole. Per questo motivo, le tecniche di filtraggio a questo livello assumono un ruolo centrale nella missione di rendere la rete più sicura e proteggere gli utenti dai rischi di attacchi. Nei paragrafi successivi verrà fornita una definizione del metodo in questione, insieme a dettagli sulle sue applicazioni pratiche e sulle modalità che ne abilitano il funzionamento.

Il \textit{DNS Filtering} consiste nell'applicare regole predefinite per bloccare o consentire richieste e risposte DNS, analogamente a quello che fa un firewall. Questa tecnica è in grado di prevenire l'accesso a contenuti malevoli o inappropriati, supportare l'applicazione di politiche aziendali e proteggere gli utenti da minacce provenienti dalla rete \cite{DBLP:journals/corr/abs-2401-03864}. Volendo essere più specifici, il filtraggio DNS trova applicazione in una vasta gamma di scenari pratici \cite{Murdoch2008,DBLP:journals/ieeesp/Varadharajan10}:
\begin{itemize}
  \item \textbf{Sicurezza informatica}: viene utilizzato per bloccare siti di phishing o che distribuiscono malware e altri contenuti dannosi;

  \item \textbf{Applicazione di politiche aziendali}: permette alle organizzazioni di imporre restrizioni sull'accesso a siti non correlati al lavoro, come social media o piattaforme di streaming, durante l'orario lavorativo;

  \item \textbf{Controllo parentale}: consente la creazione di un ambiente sicuro per i minori, limitando l'accesso a contenuti non appropriati come siti per adulti o violenti;

  \item \textbf{Gestione della larghezza di banda}: aiuta a limitare l'accesso a siti che consumano molta banda, come i servizi di streaming video, ottimizzando così l'utilizzo delle risorse di rete.
\end{itemize}

Dopo aver introdotto il concetto di filtraggio DNS e averne illustrato le principali applicazioni pratiche, è importante approfondire le tecniche utilizzate per la sua implementazione. Sebbene il DNS Filtering sia ampiamente riconosciuto come uno strumento fondamentale per migliorare la sicurezza delle reti, la sua efficacia dipende dall'applicazione di metodi avanzati per identificare e bloccare le minacce. Come evidenziato in letteratura, la ricerca ha individuato tre tecniche principali che rappresentano lo stato dell'arte in questo campo, ovvero le \textit{Response Policy Zones} (RPZ), i \textit{Threat Intelligence Feeds} (TIF) e il rilevamento di domini generati algoritmicamente (\textit{Domain Generation Algorithm}, DGA) \cite{DBLP:journals/corr/abs-2401-03864}.

Queste tecniche non solo permettono di affrontare le minacce più comuni, come phishing e malware, ma forniscono anche strumenti per rispondere a sfide più complesse, come il rilevamento di domini dinamici utilizzati per comunicare con le botnet. Tuttavia, la letteratura sottolinea anche alcune lacune, in particolare la mancanza di un'analisi integrata di queste metodologie. Approfondire ciascuna di esse consente di comprendere meglio il loro funzionamento, le applicazioni e i limiti, fornendo una base solida per ulteriori miglioramenti nel campo del DNS Filtering. A questo proposito, le sezioni seguenti saranno dedicate all'approfondimento delle tecniche appena citate, per poi fornire uno spunto su come potrebbero essere unite insieme per migliorare significativamente le performance del filtri.

\subsubsection{Response Policy Zones}
L'espressione Response Policy Zones descrive una funzionalità dei resolver DNS che consente agli amminsitratori di rete di specificare politiche per le risposte DNS, sulla base dei nomi di dominio e del contesto. Questa tecnologia opera intercettando le query e le risposte DNS, per poi confrontarle con un elenco di politiche definite dall'amministratore. Quando viene rilevata una corrispondenza, il resolver restituisce una risposta in base alla politica, che poi rappresenta l'azione da compiere. Esempi di possibili azioni, sono bloccare una certa query oppure reindirizzarla verso un sito sicuro (\Cref{fig:rpz}).

\begin{figure}
  \centering
  \includegraphics[width=1.0\linewidth]{figures/Response_Policy_zones.pdf}
  \caption{Un resolver DNS che sfrutta le Response Policy Zones}
  \label{fig:rpz}
\end{figure}

Le RPZ rappresentano un potente strumento per l'applicazione delle politiche di sicurezza e per proteggere la rete da attività malevole. Inoltre, il loro essere flessibili e personalizzabili, consente agli amministratori di definire policy molto specifiche, che si addatano a qualsiasi necessità. Tuttavia, per evitare falsi positivi e per mantenere l'efficacia del sistema nel tempo, le RPZ necessitano di una corretta configurazione e manutenzione. Infatti, le policy hanno costantemente bisogno di essere mantenute aggiornate, partendo da informazioni accurate ed affidabili.

Infine, si ritiene opportuno puntualizzare un limite di questa tecnologia. In particolare, le RPZ sono in grado di proteggere solamente da minaccie conosciute. Ciò deriva dalla natura stessa di questa tecnica di filtraggio, che fa affidamento su delle policy definite sulla base di minaccie necessariamente già inquadrate e studiate.

\subsubsection{Threat Intelligence Feeds}
Prima di descrivere la tecnica di filtraggio in sé, si ritiene necessario presentare brevemente i concetti ad essa associati: l'espressione \textit{cyber threat intelligence} si riferisce ad un insieme di informazioni relative a minacce informatiche potenziali o reali, utili per migliorare la situazione di un'organizzazione in termini di sicurezza.
%
Queste informazioni possono includere dettagli su tattiche, tecniche e procedure (TTP) adottate da attori malevoli noti, nonché indicatori di compromissione (IOC) come indirizzi IP, nomi di dominio e hash di malware. Nel contesto del DNS, i Threat Intelligence Feeds costituiscono dati aggiornati sui domini malevoli e relativi indirizzi IP.

I TIF possono provenire da fonti diverse, tra cui intelligence open-source, fornitori commerciali e gruppi di condivisione delle minacce specifici per settore. Questi flussi includono elenchi di IOC generati analizzando il traffico di rete, i log di sicurezza o altre fonti di intelligence. I resolver, quindi, possono utilizzare questi dati per bloccare o reindirizzare le query DNS verso domini malevoli noti.

In generale, l'integrazione dei TIF nel filtraggio DNS offre una difesa efficace contro le minacce conosciute e aiuta le organizzazioni a rilevare e prevenire gli attacchi prima che possano causare danni. Tuttavia, è importante ricordare che i flussi in questione non rappresentano una soluzione definitiva e devono essere utilizzati insieme ad altre misure di sicurezza per garantire una protezione completa.

\subsubsection{Domain Generation Algorithms}
Rappresenta una metodologia che attuano i malware per la generazione di nomi di dominio univoci, volti a consentire la comunicazione tra le vittime ed i cosiddetti server \textit{Command-and-Control} (C2). Tali domini, vengono solitamente generati partendo dalla data e ora corrente, rendendo il processo di blocco molto difficile per i  sistemi di filtraggio che si basano su liste statiche di indirizzi IP malevoli. Però, grazie a specifici algoritmi di Machine Learning, è possibile individuare questi nomi casuali, riconoscendo i pattern generati algoritmicamente. In particolare, analizzando la struttura dei domini creati dai malware, insieme al comportamento e al contesto delle query DNS, è possibile identificare e bloccare le comunicazioni con i server C2, anche se i nomi dannosi vengono utilizzati una sola volta.

\subsubsection{Uso combinato di RPZ, TIF e rilevamento DGA}
L'uso combinato di Response Policy Zones, Threat Intelligence Feeds e tecniche di rilevamento di algoritmi di generazione di domini, rappresenta un approccio promettente per migliorare l'efficacia del filtraggio DNS. Le RPZ possono intercettare le query DNS, confrontandole con blocklist e allowlist, mantenute costantemente aggiornate grazie ai TIF. A loro volta, i TIF possono integrare tecniche avanzate di rilevamento DGA per identificare domini malevoli conosciuti e sconosciuti, analizzando query anonimizzate.

Un framework che integri queste tecniche può migliorare notevolmente la capacità di rilevare e bloccare domini malevoli, minimizzando i falsi positivi e garantendo un'intelligence tempestiva e accurata. La trasparenza nel processo di filtraggio, che si concretizza nel remdere pubblicamente disponibili le liste di filtri, può favorire la fiducia degli utenti e prevenire abusi come la censura. Inoltre, il coinvolgimento della community attraverso contributi di esperti e ricercatori nell'ambito della sicureza informatica, può ulteriormente rafforzare l'efficacia del sistema.

Se da un lato ci sono dei vantaggi nel rendere pubbliche tali informazioni, dall'altro potrebbero emergere argomentazioni a sfavore di questa attività. Infatti, i malintenzionati sarebbero in grado di adattare le loro metodologie di attacco sulla base delle politiche di filtraggio facilmente consultabili. Ciononostante, questo approccio potrebbe generare dei vantaggi indiretti, derianti dal fatto che i creatori di malware sarebbero costretti a modificare i propri metodi di attacco, rendendo il loro comportamento più visibile all'interno del traffico di rete. Questo cambiamento, a sua volta, puù aiutare i ricercatori, e in generale i difensori, a venrie a conoscenza delle nuove tecniche di evasione emsse in campo dai cybercriminali \cite{DBLP:journals/corr/abs-2401-03864}.

\section{Metodologie di reingegnerizzazione software}
All'inizio degli anni '90 è emersa la forte necessità di reingegnerizzazione, alimentata soprattutto da tutti coloro che stavano migrando i sistemi informativi aziendali verso interfacce Web più moderne \cite{DBLP:conf/icse/MullerJSSTW00}. Ciò ha fatto si che il campo della reigegnerizzazione si sviluppasse per affrontare i problemi legati ai sistemi software divenuti ormai obsoleti. Tali sistemi, definiti con la parola inglese \emph{legacy}, risultavano tuttavia fondamentali per le operazioni ed i processi aziendali. Di conseguenza, da allora fino ad oggi, i ricercatori stanno lavorando per progettare dei modelli e dei piani d'azione che siano flessibili e riutilizzabili per poter essere applicati al processo di reigegnerizzazione dei software legacy \cite{Majthoub2018}.

In questa sezione verranno esaminati i concetti chiave e gli obiettivi della reingegnerizzazione, seguiti dalla descrizione di un modello generale che ne rappresenta l'architettura. Successivamente, saranno esplorati gli approcci più comuni e le fasi operative che caratterizzano questo processo. Infine, si fornirà una breve panoramica sui principali rischi associati alla reingegnerizzazione.

\subsection{Definizione e obiettivi}
Sono stati Chikofsky e Cross tra i primi ricercatori ad inquadrare la reingegnerizzazione in maniera chiara e strutturata. La loro interessante definizione descrive questo concetto come, testualmente: ``\textit{the examination and alteration of software systems to reconstitute it in new form and the subsequent implementation of the new form}'' \cite{DBLP:journals/software/ChikofskyC90}.

In termini più attuali, la reingegnerizzazione software si definisce come il processo di riprogettazione e reimplementazione di sistemi legacy con l'obiettivo di migliorarne la manutenibilità. Ciò richiede un'analisi della documentazione esistente per produrne una nuova ed equivalente, una revisione dell'architettura del sistema per riorganizzarne la struttura, e infine una reimplementazione con tecnologie o linguaggi moderni che soddisfano le esigenze attuali del mercato. Tutto questo dovrebbe essere fatto cercando di mantenere quante più funzionalità possibili del sistema rivisitato \cite{sommerville2011software}.

A differenza dell'ingegneria del software, che parte da specifiche e conduce alla creazione di un sistema attraverso design e implementazione, la reingegnerizzazione parte da un software esistente, che viene analizzato e trasformato per derivare un sistema target.

Questa distinzione non è solo teorica, ma trova applicazione concreta in una serie di situazioni che rendono la reingegnerizzazione una scelta indispensabile. Più nel detaglio, questa pratica risulta particolarmente necessaria:
\begin{itemize}
  \item quando il codice sorgente non ha più una struttura pulita e chiara, unito al fatto che la documentazione potrebbe non esistere;
  \item quando il supporto hardware e software nel sistema attuale diventa obsoleto e superato a causa di cambiamenti nelle politiche organizzative o nella competizione del mercato;
  \item quando i sistemi legacy, dopo anni di modifiche, diventano difficili ed onerosi da modificare.
\end{itemize}

\subsubsection{Obiettivi della reingegnerizzazione}
Sebbene gli obiettivi del processo di reigegnerizzazione possano variare sulla base dello scopo delle organizzazioni che lo attuano, è possibile individuarne 4 che risultano generali ed applicabili in ogni situazione \cite{rosenberg1996software}:
\begin{enumerate}
  \item \textbf{Preparazione per l'ampliamento funzionale:} la reingegnerizzazione non dovrebbe avere lo scopo diretto di migliorare le funzionalità di un sistema esistente, ma, piuttosto, di prepararlo per futuri miglioramenti. Questo include l'analisi delle attuali caratteristiche del sistema legacy per confrontarle con le specifiche desiderate, consentendo la costruzione di un sistema target più flessibile. Ad esempio, se le migliorie desiderate seguono il paradigma di programmazione oriantato agli oggetti, il sistema target può essere riprogettato utilizzando tecnologie orientate agli oggetti per semplificare il processo di incremento delle funzionalità;

  \item \textbf{Miglioramento della manutenibilità:} con il passare del tempo, i sistemi tendono a diventare sempre più complessi e costosi da mantenere a causa della struttura del codice poco chiara e della documentazione assente o obsoleta. La reingegnerizzazione mira a ridisegnare il sistema suddividendolo in moduli funzionali meglio definiti e con interfacce esplicite. Questo processo include l'aggiornamento della documentazione interna ed esterna, facilitando così gli interventi futuri;

  \item \textbf{Migrazione:} L'industria informatica è in continua evoluzione, con l'introduzione di nuove tecnologie che spesso superano e rendono obsolete quelle esistenti in tempi molto brevi. La migrazione può quindi diventare necessaria per mantenere la compatibilità con i nuovi sistemi, per sfruttare hardware e software moderni, e per evitare costi crescenti legati alla manutenzione di tecnologie superate. La reingegnerizzazione facilita questa transizione, progettando ed applicando una migrazione controllata verso nuove piattaforme hardware, sistemi operativi più moderni o linguaggi più attuali;

  \item \textbf{Miglioramento dell'affidabilità:} i sistemi legacy, nella maggior parte dei casi, non hanno mai raggiunto un livello di affidabilità particolarmente elevato. Nel tempo, le molteplici modifiche apportate a tali sistemi hanno spesso generato i cosiddetti ``effetti a catena'', ovvero situazioni in cui un intervento introduce involontariamente nuovi problemi. Per questo motivo, uno degli obiettivi principali della reingegnerizzazione è ottenere, nel sistema target, un livello di affidabilità significativamente superiore rispetto a quello del sistema originario.
\end{enumerate}

\subsection{Modello generale}
Il processo di reingegnerizzazione del software si basa su un modello generale che definisce come un sistema legacy viene trasformato in un nuovo sistema. Questo modello si articola nei tre principi fondamentali \textit{abstraction}, \textit{alteration} e \textit{refinement}, che vengono di seguito amalizzati \cite{rosenberg1996software}:
\begin{itemize}
  \item \textbf{Abstraction} si riferisce al processo di aumentare gradualmente il livello di astrazione di un sistema. Questo avviene sostituendo le informazioni dettagliate esistenti con informazioni più generali e astratte, creando una rappresentazione che enfatizza alcune caratteristiche del sistema, andando ad eliminarne altre. L'obiettivo è comprendere meglio la struttura e il funzionamento del sistema, mettendo in evidenza solo gli aspetti più rilevanti. Questo processo è parte della reverse engineering, supportata da specifici strumenti e tecniche;
  \item \textbf{Alteration} riguarda l'apportare una o più modifiche alla rappreseentazione di un sistema senza alterarne il livello di astrazione. Ad esempio, è possibile modificare il design di un componente software senza alterare i requisiti funzionali del sistema;
  \item \textbf{Refinement} è il processo inverso dell'abstraction, ovvero la discesa graduale verso un livello di astrazione più basso. Partendo da una rappresentazione astratta del sistema target, si procede verso la sua implementazione concreta, definendo dettagli sempre più specifici. Questo processo, noto come forward engineering, ricalca in parte lo sviluppo tradizionale del software, ma con alcune differenze dovute al fatto che si parte da un sistema già in essere.
\end{itemize}

Il modello generale illustra come, a seconda dell'obiettivo della reingegnerizzazione, si possa intervenire su diversi livelli di astrazione. Ad esempio, se si vuole solo tradurre il codice da un linguaggio ad un altro, non è necessaria la reverse engineering, poiché il lavoro si svolge già al livello di astrazione più basso, ovvero l'implementazione. Invece, se si vuole riprogettare l'architettura del sistema, il processo di reverse engineering diventa fondamentale e dovrebbe arrivare fino al livello dei requisiti, in modo da ricavare le caratteristiche funzionali dal design e dall'implementazione.

Nel corso di questa sezione si è fatto riferimento più volte ai concetti di reverse engineering e forward engineering, senza tuttavia approfondirne i dettagli. Di seguito, si propone una breve descrizione di queste tecniche, al fine di chiarirne il ruolo e le caratteristiche principali nel contesto della reingegnerizzazione:

\begin{itemize}
  \item Il \textbf{Reverse Engineering} è un processo fondamentale nell'ambito in questione. Il suo scopo è quello di analizzare il sistema legacy per identificarne i componenti, le loro interrelazioni e creare rappresentazioni a un livello di astrazione superiore. In pratica, si tratta di recuperare informazioni dal codice sorgente, dalla documentazione esistente e dagli esperti del sistema per comprendere appieno la struttura del software, i requisiti e il funzionamento.
  \item Il \textbf{Forward Engineering}, invece, è il processo di creazione di un nuovo sistema passando dai livelli di astrazione più alti a quelli più bassi, andando a sostituire gradualmente le informazioni astratte con dettagli più specifici. Questo movimento verso il basso corrisponde al normale ciclo di sviluppo software, che passa dai requisiti al design, fino ad arrivare all'implementazione fisica del sistema.
\end{itemize}

\subsection{Approcci e fasi del processo}\label{sec:reingegnerizzazione-approcci-fasi}
Esistono tre doversi approcci alla reingegnerizzazione del software, i quali si differenziano per la quantità e la velocità con cui il sistema legacy viene sostituito con il sistema target. Ogni approccio è caratterizzato da vantaggi e rischi specifici, che vengono ora discussi nel dettaglio \cite{Majthoub2018,rosenberg1996software}:
\begin{enumerate}
  \item \textbf{Big Bang}: questo approccio prevede la sostituzione dell'intero sistema in un unico momento, e di solito viene messo in campo per affrontare problemi immediati come una migrazione di architettura. Il vantaggio principale è che il sistema viene trasferito in un nuovo ambiente senza dover sviluppare interfacce tra i componenti vecchi e nuovi. Lo svantaggio, però, è che il risultato finale tende ad essere un progetto monolitico, caratteristica non sempre adeguata, soprattutto per i sistemi di grandi dimensioni. Inoltre, le modifiche apportate al sistema vecchio durante la transizione devono essere integrate nel nuovo, rendendo difficile mantenere l'allineamento e preservare tutte le funzionalità esistenti. In più, il rischio di questo approccio risulta elevato in quanto il nuovo sistema deve essere funzionalmente intatto e operare in parallelo con quello vecchio fino alla completa sostituzione;

  \item \textbf{Incrementale}: l'approccio in questione prevede la reingegnerizzazione graduale di sezioni del sistema, che vengono integrate in nuove versioni quando sopraggiunge la necessità per soddisfare nuovi obiettivi. Questo metodo suddivide il progetto in porzioni più contenute basate sulla struttura esistente del sistema, permettendo così uno sviluppo più rapido delle componenti e una maggiore facilità nel tracciare gli errori. Inoltre, gli stakeholders possono monitorare i progressi attraverso le versioni intermedie e segnalare tempestivamente eventuali funzionalità mancanti o problemi individuati. Tuttavia, questo approccio richiede tempi più lunghi per il completamento del sistema, dato che le versioni intermedie necessitano di un controllo accurato della configurazione. Inoltre, non permette una completa ristrutturazione dell'architettura, limitandosi alle modifiche interne delle sezioni reingegnerizzate. Nonostante ciò, il rischio complessivo è inferiore rispetto all'approccio ``Big Bang'', poiché i rischi legati a ciascuna componente possono essere gestiti e monitorati separatamente;

  \item \textbf{Evolutivo}: l'approccio ``Evolutivo'' prevede la sostituzione graduale delle sezioni del sistema originale con parti reingegnerizzate, selezionate in base alla loro funzionalità piuttosto che alla struttura del sistema esistente (come avviene nell'approccio ``Incrementale''). Il sistema target viene costruito utilizzando moduli coesi dal punto di vista funzionale, riorganizzando le componenti attuali per funzioni specifiche. Tra i vantaggi principali vi sono un design modulare che facilita la manutenzione e un focus su unità funzionali più semplici, che rende particolarmente agevole l'adozione di tecnologie orientate agli oggetti. Tuttavia, questo approccio richiede un'attenta identificazione delle funzioni simili, spesso distribuite nel sistema originale, e può introdurre difficoltà di integrazione tra le componenti, con possibili peggioramenti delle prestazioni dovuti alla reingegnerizzazione isolata delle sezioni funzionali.
\end{enumerate}

Dopo aver analizzato i principali approcci adottati nella reingegnerizzazione, è importante approfondire le fasi fondamentali che guidano la trasformazione del sistema legacy in un nuovo sistema. Queste fasi, articolate in cinque passaggi chiave, hanno compiti specifici volti a garantire una transizione efficace verso il sistema target \cite{rosenberg1996software}:
\begin{enumerate}
  \item \textbf{Formazione del team di reingegnerizzazione}: il team guiderà l'intero processo, richiedendo una formazione approfondita sulla gestione del cambiamento tecnologico, sui principi della reingegnerizzazione e sui processi di sviluppo del sistema target. Le loro attività comprenderanno la definizione di obiettivi, strategie e piani d’azione, oltre all'identificazione e all'acquisto di strumenti adeguati. Sarà loro compito promuovere il progetto all’interno dell’organizzazione, offrire supporto al personale e garantire l’applicazione corretta del processo. Infine, saranno essenziali buone capacità relazionali per affrontare eventuali resistenze al cambiamento;

  \item \textbf{Analisi di fattibilità del progetto}: il primo compito del team di reingegnerizzazione è valutare i bisogni organizzativi e gli obiettivi soddisfatti dal sistema esistente, garantendo che la strategia sia allineata alle esigenze aziendali. È fondamentale analizzare il software attuale per identificare problemi, scopi e vincoli, stimando il ritorno sull’investimento. Gli obiettivi, come il miglioramento delle prestazioni o la riduzione dei costi, devono essere misurabili e confrontati con i costi della reingegnerizzazione e il valore aggiunto previsto;

  \item \textbf{Analisi e pianificazione}: questa fase si articola in tre passaggi principali: l'analisi del sistema legacy, la definizione delle caratteristiche del sistema target e la creazione di un ambiente di test. L'analisi inizia con la raccolta e lo studio del codice e della documentazione disponibili, al fine di identificare peculiarità e problemi di qualità del software esistente. Successivamente, vengono definite le modifiche richieste, come la piattaforma hardware, il sistema operativo, la struttura del design e il linguaggio. Infine, si sviluppa un ambiente di test standard, necessario per garantire l’equivalenza funzionale tra il nuovo sistema e quello originale, assicurando la tracciabilità delle funzioni esistenti;

  \item \textbf{Implementazione della reingegnerizzazione}: dopo che sono stati definiti gli obiettivi e l'approccio, e dopo aver analizzato il sistema legacy, iniziano le fasi di reverse e forward engineering. La prima scompone le funzioni del sistema legacy fino a raggiungere il livello di astrazione desiderato, utilizzando strumenti che siano compatibili con gli obiettivi. In seguito, la seconda riprende dal livello raggiunto per sviluppare il nuovo design, seguendo il normale processo di sviluppo software. Durante questa fase, è fondamentale evitare modifiche o aumenti di funzionalità per semplificare la validazione, applicando tecniche per il controllo della qualità, e monitorando continuamente i miglioramenti e i potenziali rischi;

  \item \textbf{Test e transizione}: man mano che le funzionalità del nuovo sistema aumentano, è necessario effettuare test per individuare eventuali errori introdotti durante la reingegnerizzazione. Le tecniche di testing sono le stesse utilizzate per lo sviluppo di un sistema da zero. Se i requisiti del nuovo sistema coincidono con quelli del sistema legacy, è possibile riutilizzare i test e l'ambiente sviluppati durante la fase di pianificazione, confrontando i risultati per verificare la funzionalità del sistema target. Inoltre, la documentazione del sistema legacy deve essere aggiornata o sostituita per riflettere il nuovo sistema e fornire le informazioni necessarie per la sua gestione e manutenzione.
\end{enumerate}

Seguire queste fasi in modo strutturato è cruciale per garantire il successo di un progetto di reingegnerizzazione del software, riducendo i rischi e ottimizzando l'utilizzo di tempo e risorse.

\subsection{Panoramica sui rischi}
Per concludere la sezione corrente e fornire un quadro ancora più completo sulla reingegnerizzazione, si ritiene opportuno fare una breve panoramica sui potenziali rischi che questo importante processo si porta dietro, soprattutto se non viene gestito in maniera appropriata.

La reingegnerizzazione del software, pur offrendo vantaggi significativi, comporta rischi che devono essere identificati e gestiti con attenzione per garantire il successo del progetto. Questi rischi si manifestano in diverse aree chiave, tra cui il processo, il reverse engineering, il forward engineering, il personale, gli strumenti e la strategia \cite{rosenberg1996software}.
\begin{itemize}
  \item Un rischio rilevante riguarda l'elevato costo del \textbf{processo} di reingegnerizzazione manuale, che potrebbe non garantire benefici proporzionati nei tempi previsti. Inoltre, la mancanza di supporto da parte del management o una selezione inadeguata dei sottosistemi da reingegnerizzare possono compromettere il progetto. Altri rischi includono una gestione insufficiente della configurazione, l'assenza del controllo qualità o di un programma di metriche adeguato, e la carenza di esperti con conoscenze sugli applicativi locali.
  \item Nel \textbf{reverse engineering}, una delle principali difficoltà è il recupero di requisiti e informazioni di progettazione dal codice sorgente, specialmente se il linguaggio utilizzato non consente una rappresentazione astratta efficace. La perdita di informazioni aziendali incorporate nel codice costituisce un ulteriore rischio critico.
  \item Anche il \textbf{forward engineering} presenta sfide significative. L'introduzione di nuovi requisiti o funzionalità può complicare il processo di validazione, mentre la migrazione dei dati esistenti e una preparazione insufficiente durante il reverse engineering possono influire negativamente sull'esito del progetto.
  \item I principali rischi legati al \textbf{personale} riguardano la mancanza di competenze adeguate nella reingegnerizzazione e la scarsa esperienza, che possono influire negativamente sull'efficacia del progetto. Inoltre, i programmatori potrebbero lavorare in modo volutamente meno efficiente per ostacolare un'iniziativa percepita come impopolare.
  \item Le problematiche connesse agli \textbf{strumenti} includono la dipendenza da soluzioni che non funzionano come previsto o che risultano nuove e immature. Anche la stabilità dei fornitori e la qualità complessiva degli strumenti possono rappresentare un rischio significativo.
  \item  I rischi \textbf{strategici} includono l'impegno prematuro in una soluzione di reingegnerizzazione per l'intero sistema, la mancanza di una visione a lungo termine e l'adozione di obiettivi irrealistici o poco solidi. Altri problemi possono derivare dalla scelta di approcci che non rispettano gli scopi aziendali, i limiti di budget o le scadenze, oltre che dall'assenza di un piano per l'utilizzo degli strumenti di reingegnerizzazione.
\end{itemize}

\chapter{Analisi e requisiti del sistema}
Il presente capitolo fornisce un'analisi introduttiva sul sistema legacy, descrivendone le funzionalità principali, l'architettura e le tecnologie impiegate. Verranno evidenziati i limiti strutturali e operativi che hanno portato l'azienda a intraprendere un processo di reingegnerizzazione, con l'obiettivo di sviluppare un sistema più moderno, scalabile e manutenibile.

Un aspetto cruciale di questa analisi riguarda la strategia di transizione dal sistema legacy al nuovo pannello di configurazione. Inizialmente era stata prevista una coesistenza graduale tra i due sistemi, ma la direzione aziendale ha successivamente deciso di adottare un approccio differente. Il nuovo sistema verrà sviluppato fino a includere tutte le macrofunzionalità fondamentali, che gli garantiranno una quasi completa operatività. Inoltre, sarà introdotta una nuova funzionalità che, oltre a rappresentare un'evoluzione rispetto al legacy, contribuirà a differenziare il nuovo sistema e a valorizzarne le potenzialità.
%
Solo una volta completato, il nuovo pannello sostituirà integralmente il legacy con una transizione netta, eliminando quindi la necessità di retrocompatibilità e riducendo la complessità della fase transitoria.

Proseguendo, il capitolo illustrerà gli obiettivi e i requisiti principali che hanno guidato la progettazione del nuovo sistema, preparando la discussione del capitolo successivo, in cui verranno descritte nel dettaglio le scelte architetturali adottate per superare le limitazioni del sistema legacy e rispondere alle nuove esigenze aziendali e di mercato.

Infine, la parte finale del capitolo sarà dedicata all'analisi del dominio applicativo, con l'obiettivo di identificare i concetti e le entità fondamentali del sistema legacy che verranno utilizzati per la progettazione del nuovo sistema. Questa analisi è essenziale per definire una base strutturata su cui modellare le funzionalità del nuovo pannello di configurazione, garantendo una transizione coerente e ben organizzata dalle logiche del sistema precedente a quelle della nuova implementazione.

\section{Introduzione al sistema esistente}
Il sistema legacy qui esaminato rappresenta un pannello Web utilizzato dall'azienda \textit{FlashStart Group srl}\footnote{\url{https://flashstart.com}} per la gestione e la configurazione del suo filtro DNS. L’analisi di questo sistema, effettuata nell’ambito di un tirocinio in preparazione della presente tesi, si basa sulle osservazioni dirette fatte durante lo sviluppo del nuovo sistema, dato che la documentazione del legacy risulta pressoché assente.

\subsubsection{Terminologia e definizioni}
Da qui in avanti verranno utilizzate alcune terminologie specifiche che descrivono concetti e funzionalità del sistema legacy. Pertanto, prima di proseguire con la sua analisi, è opportuno fornire una breve spiegazione di questi termini, utilizzati dall'azienda per rappresentare concetti chiave del dominio applicativo.

\begin{itemize}
  \item \textbf{Rete}: identifica un indirizzo IP registrato nel sistema e associato alla sede di un cliente, come ad esempio una scuola. Le reti possono essere statiche (IP pubblico fisso) o dinamiche (IP pubblico variabile).

  \item \textbf{Endpoint}: rappresenta un dispositivo fisico specifico registrato nel sistema, come un computer o uno smartphone. A differenza della rete, un endpoint è associato direttamente a un dispositivo, piuttosto che a un indirizzo IP generico.

  \item \textbf{Dealer}: un rivenditore che vende licenze del filtro DNS ai clienti finali, i quali devono, in autonomia, configurare le proprie policy di sicurezza tramite il pannello Web.

  \item \textbf{Managed Service Provider (MSP)}: è un particolare tipo di Dealer che, oltre a vendere licenze, si occupa anche della configurazione e della gestione delle policy di sicurezza per i propri clienti. A differenza dei precedenti fornitori di servizi, gli MSP impediscono ai clienti finali di accedere al pannello, occupandosi interamente della gestione.

  \item \textbf{Whitelabel}: una funzionalità che consente ai fornitori di servizi, come gli MSP, di personalizzare il pannello Web con il proprio logo, colori e branding, nascondendo ogni riferimento al produttore originale del sistema.

  \item \textbf{ClientShield}: Un'applicazione per computer e dispositivi mobili sviluppata dall'azienda, che consente di registrare un endpoint nel sistema e sfruttare il filtro DNS anche al di fuori della rete usuale.
\end{itemize}

\subsection{Panoramica sulle funzionalità}
Il pannello Web legacy rappresenta lo strumento principale per la gestione e configurazione del filtro DNS offerto dall'azienda. Nonostante le limitazioni architetturali e tecnologiche, il sistema fornisce un insieme di funzionalità che consentono agli utenti di configurare, monitorare e amministrare le policy di filtraggio DNS. Di seguito viene fornita una panoramica delle principali funzionalità offerte dal suddetto pannello.

\subsubsection{Gestione delle policy di protezione}
Il pannello consente di creare e configurare diversi profili di protezione, ciascuno dei quali può includere una combinazione personalizzata di filtri per:
\begin{itemize}
  \item Bloccare minacce informatiche come malware e phishing;
  \item Limitare l'accesso a contenuti specifici, come siti per adulti o contenuti inappropriati;
  \item Bloccare l'accesso ad applicazioni o servizi specifici, divisi per categoria di appartenenza;
  \item Bloccare l'accesso a pagine Web e servizi provenienti da determinate aree geografiche.
\end{itemize}
Per aumentare il grado di flessibilità del filtro, gli utenti possono creare delle liste di accesso personalizzate, tra cui:
\begin{itemize}
  \item \textbf{Allow list}: per consentire l'accesso a domini specifici;
  \item \textbf{Block list}: per bloccare domini o indirizzi IP specifici.
\end{itemize}
Queste liste possiedono una priorità più elevata rispetto alle funzioni di protezione citate in precedenza. Per questo motivo, esse consentono di specificare delle eccezioni rispetto alle normali policy di sicurezza. Ad esempio, è possibile concedere l'accesso ad un contenuto di norma non consentito, oppure bloccare un dominio che risulta legittimo.

\subsubsection{Gestione delle reti di protezione}
Un'altra funzionalità fondamentale del pannello è la possibilità di specificare quali reti devono essere sottoposte al filtraggio, garantendo un controllo preciso e mirato sulle attività DNS. Questo permette di configurare reti aziendali o domestiche in modo che tutte le richieste DNS generate da tali indirizzi IP passino attraverso le policy di protezione impostate. La configurazione delle reti di protezione può essere adattata a diverse esigenze, supportando due principali modalità:
\begin{itemize}
  \item \textbf{IP pubblico statico}: questa configurazione è utilizzata quando la rete dispone di un IP pubblico statico, ovvero un indirizzo IP assegnato in modo permanente dal proprio ISP. In questo caso, il pannello consente di associare le policy di filtraggio a una rete identificata da uno specifico IP, garantendo che tutte le richieste DNS provenienti da tale rete siano sottoposte ai controlli e ai filtri impostati;

  \item \textbf{IP pubblico dinamico}: per le reti che non dispongono di un IP pubblico statico, il sistema supporta la configurazione tramite DynamicDNS\footnote{\url{https://www.rfc-editor.org/rfc/rfc2136.html}}. Questo approccio consente di monitorare e filtrare le richieste DNS anche quando l’indirizzo IP della rete varia nel tempo, utilizzando un sistema di aggiornamento dinamico che associa un nome di dominio all’IP corrente della rete. Questo garantisce continuità nella protezione senza la necessità di aggiornamenti manuali;
\end{itemize}

\subsubsection{Visualizzazione e analisi dei report}
Il pannello offre una sezione dedicata alla generazione e analisi dei report relativi al traffico DNS della rete, permettendo di monitorare l’efficacia delle policy di filtraggio e di ottenere informazioni dettagliate sull’attività di rete in un determinato intervallo di tempo. Questi report forniscono una visione chiara e organizzata del comportamento della rete, aiutando gli utenti a identificare potenziali minacce e a ottimizzare le configurazioni esistenti.

\paragraph{Tipologie di report disponibili}
Tramite un menù a cascata, è possibile selezionare diversi tipi di report, tra cui:
\begin{itemize}
  \item \textbf{Bloccati per categoria}: mostrano le richieste DNS bloccate verso siti indesiderati, raggruppandole per categoria o macro-categoria;
  \item \textbf{Consentiti per paese o categoria}: forniscono il numero di richieste consentite, organizzate per paese o per categoria di contenuti;
  \item \textbf{Malware e minacce bloccate}: presentano un’analisi delle richieste che hanno attivato il filtro, indicando malware o altre minacce bloccate;
  \item \textbf{Traffico per fasce orarie o giorni}: permettono di analizzare le richieste DNS effettuate in specifiche fasce orarie o giorni della settimana;
  \item \textbf{Report geografici}: forniscono una mappa del mondo che evidenzia il traffico DNS suddiviso per paesi e continenti.
\end{itemize}

Dopo aver configurato i parametri di analisi, gli utenti possono generare i report secondo diverse modalità. Essi possono essere esportati in formato PDF, oppure inviati direttamente via e-mail a destinatari predefiniti. Inoltre, il pannello offre una funzione di pianificazione che consente di programmare l'invio automatico dei report a intervalli regolari, ad esempio su base settimanale, rendendo più efficiente il monitoraggio continuo.

\subsubsection{Gestione dei dispositivi protetti (Endpoint)}
Il pannello include anche una funzionalità per gestire i dispositivi su cui è installato il ClientShield. In particolare, questa configurazione permette di tracciare con precisione quale dispositivo ha originato una determinata richiesta DNS, fornendo un maggiore controllo e un livello di dettaglio maggiore sulle attività della rete.

\subsubsection{Funzionalità aggiuntive}
Oltre alle funzionalità principali già descritte, il pannello Web offre una serie di strumenti utili per migliorare la gestione e il controllo delle configurazioni. Tra queste, vi è la possibilità di personalizzare la pagina di blocco che viene visualizzata dagli utenti ogni volta che tentano di accedere a un dominio non consentito.
%
Il pannello consente anche di verificare facilmente a quale categoria appartiene un determinato sito Web, aiutando gli utenti a valutare come configurare al meglio le politiche di filtraggio.
%
Un’altra funzionalità interessante è la visualizzazione del traffico DNS in tempo reale, che fornisce un monitoraggio immediato dell’attività di rete. Inoltre, il sistema supporta l'importazione di domini in formato batch, permettendo di aggiungere rapidamente liste di siti Web consentiti o bloccati attraverso file di testo.

Per concludere la panoramica sulle funzionalità, il pannello possiede un'importante integrazione con la tecnologia Active Directory di Microsoft, che consente di ottenere informazioni dettagliate non solo sul dispositivo che ha originato una determinata richiesta DNS, ma anche sull’utente utilizzato per accedere a tale macchina.

\subsection{Architettura e tecnologie utilizzate}
Il pannello Web in esame, fino ad ora descritto solo dal punto di vista delle funzionalità, presenta un'architettura monolitica, tipica dei vecchi sistemi Web-based. Nonostante sia possibile identificare due macro-sezioni, denominate \textit{Customer area} e \textit{Pannello cloud}, non vi è una reale modularizzazione del frontend e del backend. Tutto il codice risulta scritto in modo procedurale, senza l’adozione di un paradigma orientato agli oggetti, e con una scarsa separazione delle responsabilità.

\subsubsection{Tecnologie utilizzate}
Il backend è interamente sviluppato in PHP, utilizzando un approccio "plain", ossia privo di framework moderni come Laravel\footnote{\url{https://laravel.com}}. Per il frontend sono stati utilizzati HTML, CSS e una versione obsoleta di jQuery\footnote{\url{https://jquery.com}} (\texttt{1.x}), che limita le possibilità di modernizzazione dell'interfaccia. In alcuni casi, il codice PHP si occupa anche di generare dinamicamente script JavaScript, i quali eseguono ulteriori chiamate a codice PHP lato server.

\subsubsection{Integrazione con API esterne}
L'azienda ha sviluppato un set di API pubbliche che consentono di gestire il filtro DNS senza dover necessariamente utilizzare il pannello Web in questione. Queste API offrono agli utenti la possibilità di integrare il filtro DNS in applicazioni personalizzate o di sviluppare un proprio client per la gestione delle configurazioni.

Nel tempo, alcune operazioni che il sistema legacy eseguiva direttamente sono state trasferite alle suddette API, le quali centralizzano la business logic e gestiscono l'interazione con il database. Il pannello Web, in questi casi, funge da semplice interfaccia per chiamare le API. Tuttavia, molte operazioni continuano a risiedere direttamente sul sistema, implementando la logica applicativa e accedendo alla base dati.

\subsubsection{Gestione del database}
Il sistema legacy utilizza due database distinti per gestire le sue funzionalità e garantire la persistenza delle configurazioni. Entrambi adottano un motore di database \emph{relazionale}, che organizza i dati in tabelle collegate tra loro tramite un sistema di chiavi. La differenza principale risiede nella tecnologia sottostante: una base dati utilizza MySQL\footnote{\url{https://www.mysql.com}}, mentre l’altra è basata su PostgreSQL\footnote{\url{https://www.postgresql.org}}. Ciascuna di esse è destinata a scopi specifici e presenta caratteristiche diverse in termini di configurazione e prestazioni.

Il database MySQL è dedicato esclusivamente alla gestione delle licenze e dei dati anagrafici dei clienti. Esso viene ospitato su un server interno all’azienda e non è replicato in altre regioni. Questa configurazione rappresenta un collo di bottiglia significativo per gli utenti distanti dalla sede aziendale, poiché tutte le richieste relative ai dati dei clienti o delle licenze (usate soprattutto nella fase di accesso al pannello) devono necessariamente essere inviate al server centrale per essere elaborate, causando latenze elevate.

Il database PostgreSQL, invece, è utilizzato per tutti gli altri dati, inclusi i report, le regole di protezione e le liste dei domini da bloccare. Questo database è configurato in replica globale, garantendo così prestazioni più elevate e tempi di risposta migliori per i clienti situati in diverse aree geografiche. Grazie a questa configurazione, i dati necessari al funzionamento del filtro DNS possono essere accessibili rapidamente da qualunque parte del mondo.

Un aspetto di fondamentale importanza per l'azienda è rappresentato dal contenuto del suddetto database, che costituisce un valore strategico significativo. Al suo interno, infatti, è presente una lista dei domini associati alla relativa categoria di appartenenza. Questa categorizzazione è utilizzata direttamente dal filtro DNS per bloccare l'accesso a determinati siti in base alle regole configurate dagli utenti. L'operazione di categorizzazione viene gestita internamente all'azienda e si basa su un approccio di intelligenza artificiale. Tale sistema analizza i testi delle pagine Web e determina automaticamente la categoria di appartenenza di ciascun dominio, migliorando l'efficacia del filtro DNS e arricchendo continuamente il patrimonio informativo del database.

\subsection{Limitazioni riscontrate}
Il sistema legacy presenta numerose limitazioni che hanno reso necessaria una completa reingegnerizzazione. Queste riguardano sia l’assenza di funzionalità fondamentali, come il supporto alla multiutenza, sia problematiche strutturali e di usabilità che compromettono la flessibilità e l’efficienza operativa dello stesso.

\subsubsection{Mancanza del supporto alla multiutenza}
Una delle principali limitazioni del sistema legacy, e tra quelle più sentite dai clienti dell'azienda, è l’assenza di un supporto per la multiutenza. Attualmente, infatti, il sistema permette a ciascun cliente di disporre di un unico account per accedere al pannello di configurazione, che possiede i privilegi di amministratore.

Questa mancanza rappresenta un ostacolo significativo, soprattutto per i dealer o gli MSP che integrano il filtro DNS in altri prodotti o lo rivendono ad aziende terze. Questi ultimi si trovano spesso a dover gestire configurazioni e politiche di protezione per conto dei loro clienti, ma l'assenza di un sistema multiutente impedisce di delegare determinate operazioni o di offrire accesso limitato a figure specifiche all’interno delle organizzazioni servite. Allo stesso modo, chi utilizza direttamente il pannello non può creare profili con permessi ridotti, ad esempio per utenti che necessitano soltanto di monitorare le configurazioni o consultare i report senza possibilità di modificarli.

Implementare la funzionalità in questione nel sistema attuale richiederebbe modifiche strutturali profonde, che non sono realisticamente attuabili senza un suo completo stravolgimento.

\subsubsection{Profili di protezione non condivisibili}
Un’altra significativa limitazione del sistema legacy riguarda l’impossibilità di condividere i profili di protezione tra diversi utenti. Attualmente, ogni profilo è strettamente associato a un singolo cliente, senza alcuna possibilità di essere condiviso o ereditato da altri. Questa carenza rappresenta un ostacolo rilevante, specialmente per gli MSP che gestiscono clienti con esigenze simili, come un gruppo di scuole o aziende dello stesso settore. In tali casi, sarebbe estremamente utile disporre di profili condivisi che consentano di applicare la stessa configurazione a più clienti contemporaneamente.

Inoltre, la mancanza di questa funzionalità aumenta il carico di lavoro in caso di modifiche alle regole di protezione. La situazione attuale, infatti, è tale per cui ogni variazione deve essere riportata manualmente su ciascun cliente, imponendo un processo lungo e soggetto a errori.

\subsubsection{Limiti e debolezze architetturali}
Il sistema legacy presenta numerosi limiti dovuti a scelte architetturali e tecnologiche datate, che influiscono negativamente sulla manutenibilità e sull'evoluzione del software. La mancanza di una chiara separazione tra frontend e backend complica la gestione del codice, rendendo difficoltosa l’adozione di nuove tecnologie. Oltretutto, pratiche quali l’utilizzo di PHP per generare dinamicamente codice JavaScript introducono ulteriore complessità ed opacità dei sorgenti, limitando la modularità e aumentando il rischio di commettere errori.

L’architettura monolitica rappresenta una delle principali debolezze del sistema. Qualsiasi modifica, anche minima, comporta interventi che possono avere ripercussioni su altre parti del codice, a causa dell’assenza di un design accurato e di una netta separazione delle responsabilità. Questo approccio non solo rallenta il ciclo di sviluppo, ma aumenta significativamente il rischio di regressioni e rende difficoltoso il debugging. La mancanza di modularità aggrava ulteriormente il problema: ogni nuova funzionalità o aggiornamento richiede un lavoro complesso e rischioso, che spesso si traduce in un incremento della fragilità del sistema.

\subsubsection{Problemi di usabilità}
L'interfaccia utente del sistema, sviluppata con tecnologie ormai obsolete, presenta diverse limitazioni che compromettono l’esperienza degli utenti finali. L’utilizzo di una versione datata di jQuery, unito all’assenza di un layout moderno, rende l’interfaccia poco intuitiva e difficoltosa da navigare. Questi problemi non solo riducono l’efficienza operativa degli utenti, ma influenzano negativamente anche la percezione complessiva del sistema.

Un design grafico superato e poco efficiente limita inoltre la capacità del sistema di competere con soluzioni contemporanee, riducendone l’attrattiva sia per gli utenti attuali che per potenziali nuovi clienti.

\subsubsection{Conclusione}
Le limitazioni evidenziate, che includono l’assenza di funzionalità essenziali come la multiutenza e i profili condivisibili, le debolezze architetturali e i problemi legati all’usabilità e all’interfaccia utente, mettono in luce la rigidità e l'obsolescenza del sistema legacy. Tali carenze ostacolano non solo l'efficienza operativa e la scalabilità del sistema, ma anche la capacità dell'azienda di rispondere alle richieste di mercato e di competere con soluzioni moderne. Questi fattori rendono indispensabile una completa reingegnerizzazione per soddisfare le esigenze attuali e future, garantendo al contempo un sistema moderno, scalabile ed efficiente.

\section{Esigenze di transizione tra i due sistemi}
La transizione dal sistema legacy al nuovo pannello di configurazione seguirà una strategia differente rispetto a quanto inizialmente previsto. Non sarà più adottato un approccio di coesistenza graduale tra i due sistemi, bensì una transizione netta non appena il nuovo sistema sarà pronto per il rilascio. Durante la fase di sviluppo, il vecchio pannello rimarrà l’unico utilizzato dagli utenti, mentre il nuovo verrà completato fino a includere tutte le macrofunzionalità fondamentali necessarie per garantirne l’operatività, oltre a una nuova funzionalità dedicata alla gestione dei profili condivisi. Solo a quel punto avverrà il passaggio definitivo al nuovo sistema.

Poiché gli utenti si ritroveranno i propri dati direttamente nel nuovo pannello, la migrazione delle informazioni dal legacy al nuovo sistema diventa un requisito fondamentale. Questo processo avverrà internamente all’azienda e dovrà garantire che tutti i dati necessari alle funzionalità trasferite siano disponibili nel nuovo sistema prima del rilascio. Le strategie di migrazione adottate dovranno assicurare la completa integrità e coerenza delle informazioni, evitando perdite o incongruenze tra i due ambienti.

Dopo il rilascio, la fase di transizione sarà limitata esclusivamente al periodo necessario per completare il trasferimento delle funzionalità minori ancora rimaste nel legacy. Tuttavia, rispetto all'approccio iniziale, non sarà più richiesta alcuna retrocompatibilità tra i due sistemi. Infatti, le funzionalità che continueranno a essere utilizzate nel legacy fino alla loro migrazione definitiva non dovranno interagire con il nuovo sistema, evitando così complessità aggiuntive nella gestione dei dati.

Questa strategia di transizione consente di semplificare il passaggio al nuovo sistema, evitando la necessità di mantenere attivi entrambi i pannelli per un lungo periodo. Inoltre, riduce i rischi legati alla gestione della retrocompatibilità e permette di concentrare gli sforzi sullo sviluppo del nuovo sistema senza vincoli imposti dall’integrazione con il legacy.

\section{Analisi del nuovo sistema}
\subsection{Obiettivi}
Il nuovo sistema nasce con l'obiettivo di posizionarsi come una soluzione moderna e professionale, capace di soddisfare le esigenze di un mercato in espansione e di attirare anche clienti di grandi dimensioni. Per raggiungere questo scopo, è fondamentale che il pannello abbia un'interfaccia più curata e in linea con gli standard richiesti da organizzazioni complesse. Questo aspetto si inserisce in un più ampio percorso di rinnovamento della \textit{brand identity} aziendale, che include un nuovo logo, una palette di colori aggiornata e uno stile comunicativo uniforme.

Dal punto di vista tecnico, il sistema dovrà superare le limitazioni strutturali del legacy, modernizzando l'architettura per renderla più robusta e scalabile, in modo da poter gestire un numero crescente di utenti. Una delle priorità è quella di ridurre i tempi necessari per la manutenzione e l’aggiornamento, facilitando al contempo l’integrazione di nuove funzionalità.

Un obiettivo chiave della prima release del pannello è l'introduzione di funzionalità avanzate pensate per i fornitori di servizi, con particolare attenzione ai dealer e agli MSP. Tra queste, un ruolo centrale è ricoperto dai profili condivisi, noti anche come \textit{template}, che consentono di definire e applicare regole di protezione comuni a più clienti. Grazie ad i template, gli MSP potranno gestire in modo più efficiente le configurazioni, evitando di doverle replicare manualmente per ciascun cliente e permettendo la propagazione automatica delle modifiche.
%
Un'altra caratteristica fondamentale che differenzia questa prima versione da quella legacy è la presenza di una dashboard dedicata ai fornitori di servizi gestiti. Questo strumento consentirà loro di gestire in maniera completa ed efficace i clienti sotto la loro supervisione, offrendo una panoramica dettagliata sul traffico, sulle configurazioni applicate e sugli eventi di sicurezza rilevati. La dashboard sarà progettata per agevolare l’amministrazione su larga scala, semplificando operazioni come la modifica delle policy di protezione, l'assegnazione dei template e il monitoraggio centralizzato delle reti gestite.

Infine, una funzionalità fondamentale che differenzia questa prima versione da quella legacy è la gestione avanzata della multiutenza, una caratteristica assente nel sistema precedente. Nel nuovo pannello, sarà possibile associare più utenti a un singolo cliente, ciascuno con ruoli e permessi differenziati. Questo rappresenta un miglioramento significativo rispetto al passato, in cui l’accesso al pannello di configurazione era ristretto a un unico account per cliente. Grazie a questa funzione, più operatori all'interno di un'organizzazione potranno collaborare alla gestione delle policy di filtraggio DNS, con livelli di accesso che vanno dall'amministrazione completa alla sola consultazione dei report.

\subsection{Requisiti}
I requisiti riportati di seguito forniscono una panoramica delle funzionalità previste per il nuovo sistema. Tuttavia, solo una parte di essi è stata sviluppata o analizzata direttamente nel contesto del tirocinio in preparazione della presente tesi. Le funzionalità che verranno introdotte successivamente, pur essendo parte integrante del sistema finale, sono incluse in questa sezione per offrire una visione completa, ma non saranno trattate nel dettaglio nel capitolo di design.

\subsubsection{Requisiti funzionali}
\begin{itemize}
  \item \textbf{Operatività del nuovo sistema al rilascio}
    \begin{itemize}
      \item Il nuovo sistema deve includere fin dalla prima release tutte le macrofunzionalità necessarie per garantire la piena operatività del pannello di configurazione, in modo da consentire la transizione completa dal sistema legacy.
      \item Le funzionalità essenziali devono comprendere la gestione delle reti, la gestione utenti e permessi (compresa la multiutenza), la configurazione delle policy di filtraggio e la reportistica avanzata.
      \item Oltre a queste caratteristiche ereditate dal legacy, il nuovo sistema dovrà introdurre almeno una caratteristica distintiva rispetto alla versione precedente. In particolare, sarà implementata la gestione avanzata dei template, che consentirà di definire profili condivisi per applicare configurazioni comuni a più clienti.
      \item Prima della dismissione del legacy, tutte le funzionalità previste dovranno essere completamente operative e testate.
    \end{itemize}

  \item \textbf{Migrazione interna dei dati}:
    \begin{itemize}
      \item Tutti i dati del sistema legacy devono essere migrati nel nuovo sistema prima del rilascio, garantendo continuità operativa per gli utenti.
      \item Il processo di migrazione deve avvenire internamente all’azienda, senza richiedere interventi manuali da parte degli utenti finali.
      \item L’integrità e la coerenza dei dati devono essere assicurate, evitando perdite o incongruenze tra il vecchio e il nuovo sistema.
    \end{itemize}

  \item \textbf{Multiutenza e gestione utenti}:
    \begin{itemize}
      \item Il sistema deve supportare la multiutenza con ruoli differenziati, quali SuperAdmin, Admin e altri con permessi limitati.
      \item La creazione e la gestione degli utenti deve essere riservata agli amministratori.
    \end{itemize}

  \item \textbf{Dashboard per MSP}:
    \begin{itemize}
      \item Deve essere presente una schermata dedicata agli MSP, che fornisce loro una panoramica dei clienti gestiti.
      \item Tale panoramica deve includere report e statistiche aggregate che consentano di monitorare il traffico sulla rete e avere una visione completa sull'utilizzo del filtro.
      \item Deve inoltre essere presente la possibilità di impersonare uno dei clienti in gestione al fornitore di servizi, visualizzando il pannello come se tale cliente avesse effettuato l'accesso con il proprio account utente. L’MSP deve essere in grado non solo di visionare i report e le statistiche relative al cliente, ma anche di monitorare ed eventualmente modificare la sua configurazione del filtro.
      \item La schermata deve adattarsi dinamicamente in base al tipo di utente (MSP o cliente finale), mostrando solo i componenti grafici e le funzionalità a cui l’utente ha accesso.
    \end{itemize}

  \item \textbf{Gestione dei template di protezione}:
    \begin{itemize}
      \item Il sistema deve consentire la creazione e la gestione di template, che possono essere assegnati a diversi clienti.
      \item I template devono supportare opzioni di configurazione come: categorie e applicazioni da bloccare, paesi da bloccare, attivazione SafeSearch, blocco completo tranne alcuni domini specifici.
      \item Le modifiche apportate ai template devono essere propagate automaticamente a tutti i clienti a cui sono stati assegnati.
      \item Devono essere disponibili template di eccezioni personalizzati, configurabili per soddisfare esigenze specifiche.
      \item Le liste di eccezioni devono poter essere importate nel sistema anche tramite file di testo in formato \texttt{.txt} o \texttt{.csv}.
    \end{itemize}

  \item \textbf{Reportistica}:
    \begin{itemize}
      \item Il sistema deve permettere di creare, visualizzare ed esportare report personalizzati.
      \item I report devono poter coprire intervalli temporali più estesi rispetto al legacy, passando da un mese a 3, 6 o anche 12 mesi per singolo report.
    \end{itemize}

  \item \textbf{Notifiche}:
    \begin{itemize}
      \item Il sistema deve fornire notifiche direttamente nel pannello Web e via email.
      \item Le notifiche devono riguardare nuove funzionalità rilasciate, errori di rete e malware rilevati.
      \item Le notifiche via email devono includere report pianificati dagli utenti.
    \end{itemize}

  \item \textbf{Personalizzazione Whitelabel}:
    \begin{itemize}
      \item Il sistema deve supportare la personalizzazione whitelabel riservata agli MSP, consentendo di modificare elementi come: nome, logo, palette di colori, menù, messaggi personalizzati al login, email per il supporto, notifiche e redirect.
    \end{itemize}

  \item \textbf{Audit log}:
    \begin{itemize}
      \item Il sistema deve registrare in un log dettagliato tutte le azioni eseguite dagli utenti, inclusi gli accessi al pannello e le modifiche alla configurazione del filtro.
    \end{itemize}
\end{itemize}

\subsubsection{Requisiti non funzionali}
\begin{itemize}
  \item \textbf{Modellazione del dominio e gestione del database}:
    \begin{itemize}
      \item Il sistema deve prevedere una nuova modellazione del dominio, al fine di migliorare la struttura dei dati e rendere il database più robusto ed efficace.
      \item Il database deve essere progettato in modo da garantire una gestione più coerente delle relazioni tra entità, evitando le problematiche riscontrate nel sistema legacy.
    \end{itemize}

  \item \textbf{Sicurezza}:
    \begin{itemize}
      \item Il sistema deve implementare l’autenticazione a più fattori (MFA) per tutti gli utenti.
      \item La sicurezza deve essere garantita tramite un sistema di autorizzazione centralizzato, che gestisca permessi e operazioni in base ai ruoli degli utenti e alle licenze possedute. Questo servizio dedicato deve applicare le autorizzazioni a tutti i livelli, includendo le API, l’interfaccia del pannello e la gestione delle policy.
    \end{itemize}

  \item \textbf{Usabilità e interfaccia utente}:
    \begin{itemize}
      \item L’interfaccia deve essere moderna, accattivante e progettata per garantire una user experience migliorata.
      \item Il design deve avere un look professionale e orientato a clienti di grandi organizzazioni (look enterprise).
      \item L’interfaccia deve essere responsive e ottimizzata per l’utilizzo su qualsiasi dispositivo.
    \end{itemize}

  \item \textbf{Manutenibilità}:
    \begin{itemize}
      \item Il sistema deve essere sviluppato in modo modulare, per semplificare la manutenzione e consentire l’estensione con nuove funzionalità.
      \item Ci deve essere una netta distinzione tra frontend e backend, così come la suddivisione del backend in molteplici microservizi.
    \end{itemize}
\end{itemize}

Le scelte progettuali adottate per l'implementazione delle funzionalità sviluppate nel contesto di questa tesi verranno discusse nel capitolo successivo.

\subsection{Analisi del dominio}
Per progettare correttamente il nuovo sistema, è stato necessario analizzare il dominio applicativo, partendo dalla struttura concettuale del sistema legacy per definire in modo più chiaro le entità fondamentali e le loro relazioni. Questa analisi fornisce una base strutturata per la modellazione dei dati e per l'implementazione delle funzionalità previste nel nuovo pannello di configurazione.

Attualmente, la modellazione è ancora in fase di sviluppo e la seguente descrizione rappresenta una versione parziale del sistema. In particolare, le entità \textbf{Protection}, \textbf{Network}, \textbf{License} e \textbf{Report} non sono state ancora completamente definite insieme al team aziendale. Tuttavia, si è scelto di includerle ugualmente per fornire un quadro più completo del dominio e delineare le aree che necessiteranno di ulteriori approfondimenti.
%
Al contrario, le entità \textbf{Organization} e \textbf{User} sono state completamente modellate e costituiscono il nucleo del sistema multi-tenant.

In \Cref{fig:domain_model} è riportato il diagramma delle classi in formato UML che rappresenta il dominio applicativo del nuovo sistema. Tale rappresentazione, non ha la pretesa di essere esaustiva, soprattutto dal punto di vista degli attributi e dei metodi delle classi, ma mira a fornire una visione d'insieme delle entità coinvolte e delle relazioni tra di esse.

\begin{figure}
  \centering
  \includegraphics[width=1\textwidth]{./figures/domain-model.png}
  \caption{Diagramma relativo al dominio applicativo del nuovo sistema.}
  \label{fig:domain_model}
\end{figure}

\subsubsection{Organization}
L'entità \texttt{Organization} rappresenta il fulcro del sistema multi-tenant ed è il punto di riferimento per tutte le altre componenti. Ogni utente, configurazione di filtraggio e reportistica è direttamente associato a un'organizzazione, che costituisce il contesto primario in cui avvengono le operazioni all'interno del sistema.

L'organizzazione è l'elemento che definisce l'esistenza stessa di un utente all'interno del sistema. Questi ultimi, infatti, non esistono al di fuori del contesto organizzativo: ogni utente deve appartenere a una specifica organizzazione e non può operare al di fuori di essa. Al momento della creazione di una nuova organizzazione, viene automaticamente generato un primo utente con permessi di amministratore. Tale amministratore ha la facoltà di creare nuovi utenti all'interno della propria organizzazione, o in quelle da essa gestite.

L'entità in questione non si limita a definire il perimetro degli utenti, ma è il nodo centrale attorno al quale ruotano tutte le altre configurazioni del sistema. Ogni organizzazione è associata a una o più reti (Network), che rappresentano gli indirizzi IP da sottoporre al filtraggio DNS. Le configurazioni di protezione (Protection), che definiscono cosa filtrare, sono legate all'organizzazione e non ai singoli utenti. Questo significa che tutte le policy di filtraggio sono decise a livello organizzativo. I dati statistici relativi all’utilizzo del sistema di filtraggio DNS sono raccolti e visualizzati a livello di organizzazione, consentendo agli utenti di monitorare il traffico filtrato e le attività correlate.

Ogni organizzazione è identificata da un id univoco, un nome e una tipologia che ne determina il ruolo all'interno del sistema. Esistono due macrotipologie di organizzazione:
\begin{itemize}
  \item \textbf{Organizzazioni che gestiscono altre organizzazioni}, ovvero Managed Service Provider e Dealer.
    \begin{itemize}
      \item \textbf{MSP}: oltre a rivendere il servizio di filtraggio DNS, si occupano direttamente della gestione della protezione per i loro clienti. In questo modello, il cliente finale non ha autonomia sulle proprie policy di protezione, in quanto è l'MSP a definirle e amministrarle.
      \item \textbf{Dealer}: rivendono il servizio di filtraggio DNS, ma senza occuparsi della configurazione delle policy. I clienti finali hanno piena libertà di gestione delle proprie protezioni.
    \end{itemize}
  \item \textbf{Clienti finali}, che possono essere:
    \begin{itemize}
      \item \textbf{Altre organizzazioni}, come scuole, hotel, aziende, attività commerciali.
      \item \textbf{Singoli individui}, che desiderano proteggere la propria rete domestica e i propri dispositivi personali.
    \end{itemize}
\end{itemize}

\subsubsection{User}
L'entità \texttt{User} rappresenta qualsiasi soggetto in grado di accedere al pannello di configurazione di un'organizzazione ed eseguire azioni sulla base dei permessi definiti dal proprio ruolo. Gli utenti operano sempre all'interno del contesto di un'organizzazione, non potendo esistere in modo indipendente. Il sistema adotta un modello a ruoli per differenziare le capacità operative di ciascun utente, limitandone o ampliandone i privilegi a seconda delle necessità organizzative.

Esistono cinque tipologie di utenti, ciascuna con permessi specifici nell'ambito della propria organizzazione:
\begin{itemize}
  \item \textbf{SuperAdmin}: rappresenta l'utente con i più alti privilegi possibili. Questo ruolo è riservato al team di supporto dell'azienda produttrice del filtro DNS e gode di accesso illimitato a tutte le organizzazioni registrate nel sistema. Un SuperAdmin può creare ed eliminare utenti e organizzazioni, modificare qualsiasi policy e visionare tutti i report disponibili, indipendentemente dal cliente a cui sono associati.
  \item \textbf{Admin}: è l'utente amministratore della propria organizzazione e di tutte quelle al di sotto di essa nella gerarchia. Ha pieno controllo sulle entità amministrate ed è l'unico utente con il permesso di creare nuovi account all'interno del proprio contesto organizzativo.
  \item \textbf{Policy}: ha la facoltà di gestire le reti (Network) e i profili di protezione (Protection) all'interno della propria organizzazione. Non dispone di permessi amministrativi e non può gestire gli utenti.
  \item \textbf{Editor}: ha privilegi simili all'utente di tipo Policy, ma con un ambito più ristretto. In particolare, può gestire solo le configurazioni di sicurezza, senza poter modificare le impostazioni delle reti.
  \item \textbf{ReadOnly}: è l'utente con il livello di accesso più limitato. Ha esclusivamente il permesso di visionare report e statistiche, senza poter modificare alcuna configurazione. Il suo ruolo è strettamente legato al monitoraggio passivo delle attività del filtro DNS per una data organizzazione.
\end{itemize}

Ogni utente è identificato da un id, un nome, un ruolo e un username. Il campo username contiene l'indirizzo email dell'utente ed è utilizzato per l'autenticazione all'interno del sistema.

\subsubsection{License}
L'entità \texttt{License} rappresenta una delle componenti ancora in fase di modellazione, ma il suo ruolo nel sistema è chiaro: essa determina il livello di servizio del prodotto associato a un'organizzazione e influisce sui permessi e sulle funzionalità disponibili per quest’ultima. Ogni organizzazione deve disporre di almeno una licenza attiva per poter usufruire del servizio di filtraggio DNS.

Esistono diverse tipologie di licenze, tra cui quelle di prova, che vengono fornite gratuitamente con lo scopo di illustrare e dimostrare le capacità del prodotto prima di un eventuale acquisto. Oltre alle licenze di prova, sono previste licenze a pagamento con diversi livelli di servizio, i cui dettagli non sono ancora stati completamente definiti.

L'assegnazione di una licenza può essere effettuata esclusivamente da un utente con il ruolo di SuperAdmin, oppure dall’Admin di un’organizzazione di tipo MSP o Dealer. Ciò significa che solo il produttore del software e gli intermediari autorizzati possono concedere o modificare una licenza per un'organizzazione cliente.

\subsubsection{Report}
L'entità \texttt{Report} rappresenta la capacità del sistema di generare e visualizzare dati analitici relativi all'attività di filtraggio DNS per una o più organizzazioni. Lo scopo principale di un report è quello di fornire un monitoraggio dettagliato sull'utilizzo del filtro, offrendo statistiche utili per comprendere il traffico di rete e l'efficacia delle policy di protezione adottate. Tra le informazioni contenute in un report vi sono, ad esempio, il numero di richieste DNS bloccate, le cinque categorie di siti più frequentemente filtrate, il numero di domini non risolti e altri indicatori rilevanti per la sicurezza della rete.

Una caratteristica distintiva del nuovo sistema è la possibilità di generare report aggregati, che includono dati provenienti da più organizzazioni. Questa funzionalità è particolarmente utile per gli MSP, in quanto consente loro di ottenere una panoramica completa e centralizzata sui clienti che gestiscono. I report aggregati possono includere informazioni come l’elenco dei malware rilevati e i dispositivi su cui sono stati identificati, il numero di sessioni attive e altre metriche relative alla sicurezza e alla gestione del traffico DNS.

L'entità Report non è ancora stata completamente modellata, ma il suo ruolo nel sistema è fondamentale per garantire una visibilità chiara e approfondita sull’efficacia delle configurazioni di filtraggio applicate alle reti delle organizzazioni.

\subsubsection{Protection}
L'entità \texttt{Protection} rappresenta il concetto di protezione associato a una rete (Network), determinando in che modo il sistema di filtraggio DNS deve bloccare o consentire l'accesso ai contenuti. Ogni istanza di Protection è composta da una lista di policy di sicurezza, ognuna delle quali specifica una particolare configurazione del filtro. Generalmente, una Protection viene creata da un'organizzazione e assegnata a una o più delle proprie reti, definendo le regole di filtraggio applicate al traffico generato da tali indirizzi IP.

Con lo sviluppo del nuovo sistema, l’entità in esame introduce un'importante evoluzione: la possibilità di essere condivisa tra più organizzazioni. Se un’Organization crea una Protection e la contrassegna come condivisibile, tale configurazione potrà essere visualizzata e assegnata dai clienti gestiti dall'organizzazione creatrice. Questa funzionalità risulta particolarmente utile per gli MSP e i Dealer, che possono fornire ai propri clienti configurazioni predefinite e standardizzate senza richiedere loro di crearne di nuove.

Un ulteriore vantaggio del nuovo modello è la propagazione automatica delle modifiche per le Protection condivise. Se l’organizzazione creatrice aggiorna una configurazione di protezione condivisa, tutte le organizzazioni che la utilizzano vedranno applicate automaticamente le modifiche senza necessità di intervento manuale. Per supportare questa logica, la modellazione prevede due relazioni distinte tra Protection e Organization:
\begin{enumerate}
  \item Una relazione di \textit{associazione semplice} che esprime la creazione, collegando un’Organization alla Protection che ha generato.
  \item Una relazione di \textit{aggregazione} che esprime l’utilizzo, collegando una Protection a una o più Organization che la utilizzano. Questa relazione è valida sia per le Protection non condivise, che per quelle condivise.
\end{enumerate}

Nonostante la sua importanza, Protection da sola non è sufficiente a garantire il livello di granularità richiesto da un moderno sistema di filtraggio DNS. Per questo motivo, essa è composta da una lista di configurazioni più specifiche, rappresentate dall'entità ProtectionPolicy, che definisce nel dettaglio il comportamento della protezione.

\subsubsection{ProtectionPolicy}
L'entità \texttt{ProtectionPolicy} rappresenta un insieme di regole appartenenti alla stessa tipologia di protezione, definita dall'attributo \texttt{ProtectionType}. Poiché il sistema di filtraggio DNS prevede diverse categorie di protezione, la modellazione adotta una struttura a due livelli: l'entità Protection mantiene la lista delle policy applicate, mentre ogni elemento di questa lista (ProtectionPolicy) contiene le regole specifiche relative a una determinata categoria di protezione.

Questa suddivisione consente di organizzare in modo chiaro e modulare le configurazioni di sicurezza. Un'istanza di Protection può essere composta da più ProtectionPolicy, ciascuna responsabile di un particolare aspetto del filtraggio. Questo approccio permette alle organizzazioni di combinare diverse policy all'interno di un'unica configurazione di protezione, garantendo così una gestione flessibile e scalabile delle regole di filtraggio DNS.

I possibili tipi di protezione (ProtectionType) includono:
\begin{itemize}
  \item \textbf{ThreatProtection}: blocca siti identificati come minacce alla sicurezza, come malware e phishing.
  \item \textbf{ContentFilter}: filtra i contenuti sulla base della loro categoria tematica (es. pornografia, giochi d'azzardo, social network).
  \item \textbf{ApplicationBlock}: impedisce l’accesso a specifiche applicazioni o servizi online.
  \item \textbf{IPAddressBlock}: blocca l'accesso a determinati indirizzi IP.
  \item \textbf{SearchEnginesProtection}: applica restrizioni alle ricerche sui motori di ricerca, come l'attivazione forzata della modalità SafeSearch.
  \item \textbf{GeoBlock}: limita l’accesso a domini o indirizzi IP in base alla loro geolocalizzazione.
  \item \textbf{AllowList} e \textbf{BlockList}: definiscono eccezioni personalizzate, rispettivamente per consentire o bloccare domini specifici.
\end{itemize}

\subsubsection{Network}
L'entità \texttt{Network} riveste un ruolo fondamentale nel presente dominio applicativo, in quanto definisce l’ambito su cui devono essere applicate le regole di protezione del filtro DNS. Ogni rete è associata a un'Organization e rappresenta il punto di ingresso del traffico che verrà sottoposto a filtraggio. La configurazione della protezione di una rete è determinata dalla Protection assegnata ad essa, specificando così quali policy devono essere applicate.

Questa relazione tra Network e Organization è modellata come una \textit{composizione}, indicando che una rete non può esistere senza l’Organization a cui appartiene. Se un'organizzazione viene eliminata, anche tutte le reti ad essa associate vengono rimosse.

Una rete può appartenere a una delle seguenti tipologie, identificate dall'enumerazione \texttt{NetworkType}:
\begin{itemize}
  \item \textbf{Static}: identifica reti con un indirizzo IP statico, il cui riferimento rimane invariato nel tempo.
  \item \textbf{Dynamic}: rappresenta reti con indirizzo IP dinamico, che necessitano dell’ausilio di un servizio di DynamicDNS per garantire un'associazione stabile nel tempo.
  \item \textbf{Remote}: utilizzata per proteggere dispositivi mobili ed endpoint aziendali che richiedono la protezione del filtro DNS anche quando non sono connessi alla rete principale dell'organizzazione.
\end{itemize}

Ogni Network è identificata da un id, da un ipAddress e da un flag isCIDR, che indica se l’indirizzo IP è espresso in formato CIDR\footnote{\url{https://datatracker.ietf.org/doc/html/rfc4632}}. L’associazione tra Network e Protection permette di applicare un insieme di policy a una specifica rete, garantendo così un controllo granulare sul filtraggio del traffico DNS.

\section{Design del nuovo sistema}

Il presente capitolo descrive le scelte architetturali, tecnologiche e progettuali adottate per soddisfare gli obiettivi e i requisiti definiti nel capitolo precedente. L'attenzione si concentra sulle funzionalità sviluppate o analizzate durante il tirocinio, pur accennando alle linee guida per alcune delle funzionalità future, al fine di offrire una visione più completa del sistema.

La progettazione del nuovo sistema punta a superare i limiti identificati nel legacy, adottando un'architettura moderna, modulare e scalabile, che semplifichi la gestione e consenta l'integrazione di nuove funzionalità. Particolare attenzione è stata dedicata alla separazione delle responsabilità tra i vari componenti, alla retrocompatibilità con il sistema attuale e al miglioramento dell'esperienza utente.

Il capitolo è strutturato come segue: dopo una descrizione dell’architettura generale del sistema, si approfondiranno le principali componenti e funzionalità chiave, come la multiutenza e la gestione dei template. Infine, saranno discussi gli aspetti relativi al miglioramento del database e alle strategie messe in campo per garantire la corretta coesione tra i due sistemi.

\section{Architettura generale}

Per la progettazione del nuovo sistema è stata adottata un'architettura a microservizi, caratterizzata da una separazione netta tra frontend e backend. Questa scelta è motivata dalla necessità di superare i limiti architetturali del sistema legacy, introducendo modularità e una chiara separazione delle responsabilità tra i vari componenti.

Nel nuovo sistema, il frontend è rappresentato da un singolo microservizio dedicato alla logica di presentazione dei dati e all'interazione con l'utente. Il backend, invece, è stato suddiviso in molteplici microservizi indipendenti, ciascuno dedicato a un modulo o a una funzionalità specifica dell'applicativo. Questa scomposizione consente di ottenere una maggiore flessibilità, facilitando sia l'aggiunta di nuove funzionalità che la manutenzione del sistema.

I microservizi comunicano tra loro utilizzando il protocollo HTTP e, più specificamente, attraverso API RESTful che ciascuno di essi espone. Queste API vengono inoltre utilizzate sia dal frontend sia per la creazione delle API pubbliche aziendali. Queste ultime, già presenti insieme al sistema legacy, sono state rinnovate per adattarsi alle nuove logiche di funzionamento introdotte nel nuovo sistema.

\subsection{Frontend}
Il frontend è stato progettato come una Single-Page Application (SPA) utilizzando il framework Next.js\footnote{\url{https://nextjs.org}}, scelta che consente di offrire agli utenti un'esperienza fluida ed un accesso rapido da qualsiasi browser, senza bisogno di installare nessun applicativo. Questo microservizio si occupa esclusivamente della logica di presentazione, implementando l'interfaccia grafica e gestendo l'interazione con gli utenti finali.

L'architettura del frontend segue un approccio modulare, in cui ogni pagina e componente è organizzato in base a un sistema di routing gerarchico, denominato App Router\footnote{\url{https://nextjs.org/docs/app}} e facente parte di Next.js. Grazie a tale sistema di routing, la struttura delle cartelle rispecchia direttamente quella degli URL, permettendo una gestione chiara e intuitiva delle varie sezioni dell’applicazione.

Per velocizzare lo sviluppo dell'interfaccia e garantire un design coerente, è stato adottato il template TailAdmin\footnote{\url{https://tailadmin.com}}, una libreria di componenti predefiniti completamente personalizzabili. Questo ha permesso di ridurre il tempo di sviluppo mantenendo comunque la possibilità di adattare l’interfaccia grafica alle esigenze specifiche del progetto. Il template fornisce inoltre pagine già strutturate, che sono state opportunamente modificate per integrarsi con la logica applicativa del sistema.

Un altro aspetto chiave dell'architettura è la gestione dell'autenticazione, implementata tramite la libreria dedicata \texttt{next-auth}\footnote{\url{https://authjs.dev}}. Sebbene Next.js offra la possibilità di definire API direttamente all'interno del frontend, si è scelto di non utilizzare questa funzionalità per la gestione dell'autenticazione. Invece, tale processo è stato demandato a un microservizio che risiede nel backend del sistema, responsabile di tutto ciò che riguarda autenticazione ed autorizzazione.

Anche tutte le operazioni di elaborazione dei dati, come le computazioni o la gestione delle regole di business, sono demandate al backend. Questo approccio garantisce una maggiore coerenza e centralizzazione della logica applicativa, riducendo la complessità del frontend e migliorando la scalabilità del sistema.

\subsection{Backend}
Come già accennato, il backend non è stato concepito come un unico servizio monolitico, ma si è optato per una suddivizione in più microservizi, ciascuno responsabile di una specifica funzionalità o modulo del sistema. Questa scomposizione è stata ottenuta analizzando il sistema legacy, identificando le principali componenti e progettando un microservizio per ognuna di esse.

Tutti i microservizi del backend sono progettati secondo un'architettura uniforme, la quale implementa un servizio Web che espone un'API di tipo REST. Questo standardizzazione agevola lo sviluppo e la manutenzione del backend, rendendo i microservizi facilmente scalabili e intercambiabili. Più in dettaglio, ogni microservizio è strutturato secondo i seguenti livelli:

\begin{itemize}
  \item \textbf{Routes}: file che definiscono tutte le rotte e la loro struttura, specificando i percorsi, i parametri e i metodi HTTP supportati. Questo livello si occupa di mappare le richieste esterne ai rispettivi controller.

  \item \textbf{Controllers}: contengono i metodi responsabili della gestione delle richieste HTTP. I controller ricevono le richieste, ne validano i parametri, delegano l'elaborazione ai servizi, raccolgono i risultati e generano le risposte HTTP in formato JSON.

  \item \textbf{Services}: introducono un grado di astrazione tra i controller e il livello di accesso ai dati (repository). I servizi implementano la logica di business, interagiscono con la base di dati tramite i repository e gestiscono operazioni come la serializzazione e deserializzazione dei dati. Questo strato garantisce che la rappresentazione dei dati sul database sia trasparente agli utilizzatori delle API, come il frontend.

  \item \textbf{Repositories}: rappresentano il livello più vicino ai dati. Qui vengono implementati i metodi che accedono direttamente al database e che eseguono operazioni CRUD specifiche per i vari tipi di dati gestiti. I repository devono rimanere semplici e focalizzati esclusivamente sull'accesso ai dati, senza includere logica di business o presentazione.
\end{itemize}

Questa suddivisione in livelli garantisce una chiara separazione delle responsabilità, migliorando la manutenibilità e la scalabilità del sistema. Inoltre, l'organizzazione del backend in livelli si presta particolarmente bene a contesti come questo, in cui è necessario garantire la retrocompatibilità con una base di dati legacy, avente tabelle malformate. Con questa architettura, i livelli più alti nascondono le logiche implementative sottostanti, permettendo di gestire in modo trasparente eventuali differenze tra il nuovo schema del database e quello esistente. Inoltre, risulta agevole implementare strategie di fallback, permettendo alle funzionalità che devono ancora riferirsi al database legacy di operare senza compromettere il funzionamento generale del sistema.

\chapter{Implementazione}

\section{Piano di Sviluppo}
Il processo di sviluppo del nuovo sistema si è basato sulla metodologia \textit{Agile SCRUM}, con cicli di sviluppo iterativi e incrementali della durata di due settimane (\textit{sprint}). Al termine di ogni sprint, si è tenuto un meeting di revisione con il \textit{project manager} (PM), il quale ha supervisionato lo stato di avanzamento del lavoro, analizzato le criticità emerse e verificato la conformità delle ore lavorate rispetto a quelle preventivate. Durante questi incontri, si sono definite le attività per lo sprint successivo, garantendo un flusso di sviluppo costante e ben organizzato.

Il team di sviluppo è composto esclusivamente da \textit{Full Stack Developer}, senza una suddivisione rigida tra frontend e backend. Le attività di progettazione grafica non sono state gestite internamente, ma affidate a un designer esterno, incaricato del rinnovo della brand identity. Quindi, per quanto concerne la UI/UX, il team si è solo ocupato di implementare le nuove interfacce grafiche, seguendo le specifiche fornite dal designer.

Per la gestione e il monitoraggio delle attività, è stato utilizzato lo strumento di project management \textit{ClickUp}\footnote{\url{https://clickup.com/}}, che ha permesso di tracciare i task, assegnare priorità e garantire un'organizzazione efficace del lavoro.

Le attività svolte durante gli sprint hanno incluso lo sviluppo di nuove funzionalità, la risoluzione di bug, il refactoring del codice e l'ottimizzazione delle performance. Particolare attenzione è stata dedicata alla qualità del software, con l'integrazione di test automatici e revisioni periodiche del codice prima del rilascio delle feature.

\subsection{Ambiente di sviluppo e infrastruttura}
L'ambiente di sviluppo è stato configurato in modo da rispecchiare il più possibile l'ambiente di produzione, garantendo una maggiore affidabilità nei test e riducendo il rischio di anomalie legate a differenze infrastrutturali. Per ottenere questa uniformità, sono state adottate tecnologie che permettono di replicare fedelmente la configurazione di produzione.

L'intero sistema di sviluppo è stato containerizzato tramite Docker, con i vari servizi orchestrati tramite Docker Compose. Questo permette di avviare l’ambiente di sviluppo in modo rapido e riproducibile su diverse macchine, riducendo problemi di configurazione tra membri del team. Inoltre, l'uso di \textit{Docker-in-Docker} nella pipeline \textit{CI/CD} consente di replicare l’ambiente di esecuzione anche nei job automatici di test e build.

Per la gestione delle configurazioni, ogni microservizio utilizza un file \texttt{.env}, che contiene le variabili d’ambiente necessarie per la configurazione dinamica dello stesso, come le credenziali di accesso ai database e gli endpoint delle API. Durante l’esecuzione della pipeline CI/CD, questo file viene iniettato nei container, permettendo di mantenere un ambiente coerente tra sviluppo, test e produzione, senza la necessità di configurazioni separate per ciascun contesto.

Per migliorare l'affidabilità dei test, è stata implementata una strategia di \textit{seeding} del database, che consente di popolare l’ambiente di test con dati coerenti ad ogni esecuzione, assicurando risultati riproducibili e verifiche affidabili.

\subsection{Struttura della repository}
L'intero codice del progetto è gestito tramite una \textit{monorepo}, ospitata su GitLab. La scelta di adottare una monorepo è motivata dalla necessità di mantenere in un unico repository sia il frontend che tutti i microservizi che compongono il backend, semplificando la gestione delle dipendenze, la coerenza tra i moduli e l'esecuzione delle pipeline di \textit{Continuous Integration/Continuous Deployment} (CI/CD).

Per ottimizzare la gestione della monorepo, è stato utilizzato \texttt{Turborepo}\footnote{\url{https://turbo.build}}, un tool specificamente progettato per lo sviluppo di applicazioni in TypeScript. Turborepo consente di affrontare in modo efficiente il problema dei lunghi processi di compilazione, tipici delle monorepo con numerosi pacchetti e molteplici task (come compilazione, testing e linting). Il tool pianifica l'esecuzione dei task in modo ottimizzato, parallelizzandoli su tutti i core disponibili e implementando un avanzato sistema di caching. Questo permette di ridurre drasticamente i tempi di build successivi al primo, migliorando la produttività del team di sviluppo.

L'organizzazione della monorepo sfrutta la funzionalità \textit{workspace} di \texttt{pnpm}, suddividendo i pacchetti in due macro-categorie:
\begin{itemize}
  \item \textbf{Apps}: Contiene tutti i servizi che vengono eseguiti in maniera indipendente e che costituiscono i diversi microservizi del sistema. Qui sono presenti il frontend e tutti i componenti del backend che operano come unità autonome.
  \item \textbf{Packages}: Include pacchetti di supporto utilizzati dalle \textit{apps}. Tra questi vi sono il package \texttt{commons}, che contiene definizioni e metodi condivisi tra i vari servizi, i pacchetti dedicati alla gestione dello schema di \textit{Prisma} e all'esportazione del client per il database, oltre ai pacchetti per il seeding di quest'ultimo nella fase di test.
\end{itemize}

\subsubsection{Modello di branching}
Per organizzare al meglio il lavoro di sviluppo del team, è stato adottato un modello di branching strutturato, che consente di gestire in modo chiaro e controllato il ciclo di vita del codice. Esso prevede i seguenti branch principali:
\begin{itemize}
  \item \textbf{Main}: Contiene il codice stabile e rappresenta l'unico branch dal quale si effettuano rilasci in produzione.
  \item \textbf{Unstable}: Include le funzionalità candidate al rilascio in produzione. Il codice qui presente è testato in un ambiente di staging prima di essere eventualmente integrato nel branch \textit{main}.
  \item \textbf{Branch personali}: Ogni sviluppatore lavora principalmente su un branch personale, denominato con il proprio cognome, per eseguire i commit delle proprie modifiche prima di unire il codice nei branch condivisi.
  \item \textbf{Feature branches}: Per lo sviluppo di funzionalità specifiche, possono essere creati branch temporanei, indipendenti dai branch personali, in modo da favorire un'organizzazione più modulare del codice.
\end{itemize}

Questa strategia consente di mantenere un flusso di sviluppo ordinato, con chiara separazione tra il codice in produzione, il codice in fase di test e le modifiche in sviluppo. Inoltre, grazie all'integrazione con le pipeline CI/CD di GitLab, il sistema è in grado di eseguire automaticamente test e build per ogni \textit{push} e \textit{merge request}, garantendo un'elevata affidabilità prima della promozione del codice verso l'ambiente di produzione.

\subsubsection{Pipeline CI/CD}
L'automazione del processo di sviluppo è stata realizzata attraverso una pipeline di \textit{CI/CD}, integrata direttamente in GitLab. La pipeline è suddivisa in più fasi, organizzate per garantire una validazione progressiva del codice prima del rilascio in staging.

L'intero processo di build e test si basa su Docker-in-Docker\footnote{\url{https://www.docker.com/resources/docker-in-docker-containerized-ci-workflows-dockercon-2023}} (DinD), una soluzione che permette alla pipeline di eseguire e gestire container Docker all'interno di ambienti di esecuzione basati sulla stessa tecnologia. Questo approccio consente di costruire immagini Docker direttamente nei job della pipeline, riducendo la dipendenza da infrastrutture esterne e garantendo isolamento ed una maggiore coerenza tra gli ambienti di sviluppo, test e deploy.

Il flusso di lavoro della pipeline prevede le seguenti fasi:
\begin{enumerate}
  \item \textbf{Build dell'ambiente di sviluppo}: I servizi vengono compilati e avviati tramite \texttt{docker-compose}, con supporto alla parallelizzazione dei task e caching per ottimizzare i tempi di build.
  \item \textbf{Esecuzione dei test}: Ogni microservizio è testato in container isolati. Viene eseguito il processo di seeding del database per simulare scenari reali e ottenere una copertura completa del codice.
  \item \textbf{Build per la produzione}: Se i test risultano superati, viene generata una build ottimizzata per il rilascio, con tagging e push delle immagini Docker nel registry di GitLab.
  \item \textbf{Deploy sull'ambiente di staging}: Il codice viene distribuito aggiornando i container in esecuzione e ripristinando il database con i dati necessari per il test pre-produzione.
\end{enumerate}

La pipeline utilizza strategie di caching avanzate per ridurre i tempi di esecuzione. I pacchetti \texttt{pnpm} e la cache di \texttt{Turborepo} vengono riutilizzati tra i job, evitando ricompilazioni non necessarie. Inoltre, il sistema di \textit{retry} assicura la ripetizione automatica dei job in caso di errori transitori, migliorando l'affidabilità del processo.

Grazie a questa configurazione della pipeline è possibile garantire un flusso di sviluppo strutturato e sicuro, assicurando che solo codice stabile e testato venga promosso verso l'ambiente di staging e, successivamente, alla produzione.

\subsection{Test e controllo qualità}
Una componente fondamentale del piano di sviluppo è stata la verifica della qualità del codice e la sua copertura tramite test automatizzati. Il processo di testing ha seguito le seguenti linee guida:
\begin{itemize}
  \item \textbf{Copertura al 100\%} del codice backend e frontend tramite test unitari e di integrazione.
  \item Per i test lato backend è stata utilizzata la libreria \texttt{Mocha}\footnote{\url{https://mochajs.org}}, che permette di eseguire test unitari e di integrazione in un ambiente controllato, verificando il corretto funzionamento delle API e della logica applicativa.
  \item Per quelli lato frontend è stato invece adottato \texttt{Playwright}\footnote{\url{https://playwright.dev}}, uno strumento per il testing end-to-end che consente di simulare interazioni utente su più browser. Tuttavia, la copertura dei test frontend è ancora da ampliare per garantire una validazione più completa dell'interfaccia utente.
  \item Utilizzo di strumenti di analisi statica del codice come \texttt{ESLint}\footnote{\url{https://eslint.org}} e \texttt{Prettier}\footnote{\url{https://prettier.io}} per garantire standard di qualità e uniformità nella formattazione.
  \item Verifica delle performance e dei comportamenti critici del sistema mediante test prestazionali.
\end{itemize}

\subsection{Approccio alla reingegnerizzazione}
Come descritto nella \Cref{sez:reingegnerizzazione-approcci-fasi}, ogni modello di reingegnerizzazione presenta vantaggi e criticità differenti. Il processo adottato in questo contesto si pone a metà tra l'approccio ``Big Bang'' e quello ``Evolutivo''. Lo sviluppo del nuovo sistema è già avviato e procede in modo incrementale, senza tuttavia una fase di coesistenza con il vecchio. Una volta completato, esso sostituirà integralmente il sistema precedente, fatta eccezione per alcune funzionalità che continueranno a essere gestite dal legacy senza necessità di retrocompatibilità.

La transizione avverrà in modo diretto, senza un rilascio graduale in produzione. Questo semplifica la migrazione, evitando la necessità di interfacce di compatibilità tra i due sistemi. Tuttavia, a differenza di un classico approccio ``Big Bang'', lo sviluppo non è stato affrontato come una riscrittura monolitica. Il nuovo sistema, infatti, è stato progettato fin dall'inizio con un'architettura a microservizi, organizzata sulla base delle funzionalità piuttosto che sulla replica della struttura esistente.

Questo approccio ibrido permette di bilanciare i vantaggi dei due modelli. La sostituzione completa del vecchio sistema eliminerà la necessità di mantenere allineate due versioni in parallelo, riducendo la complessità operativa. Allo stesso tempo, l'architettura modulare migliorerà la manutenibilità e faciliterà l'integrazione di nuove tecnologie nel tempo. Questa fusione di metodologie garantirà una transizione più controllata, riducendo il rischio di regressioni e assicurando una maggiore stabilità per il sistema finale.

\chapter{Valutazione}

L'obiettivo del presente capitolo è discutere i risultati finora ottenuti, in particolare sulla base dei requisiti già soddisfatti e delle migliorie introdotte dal nuovo sistema rispetto alla versione legacy. Poiché il sistema è ancora in fase di sviluppo, non è possibile fornire una valutazione complessiva, effettuare un confronto esaustivo con il vecchio sistema, né presentare risultati di test prestazionali e di usabilità. Tuttavia, è possibile analizzare le funzionalità già implementate e progettate, valutandone il contributo rispetto agli obiettivi iniziali e identificando le garanzie che il nuovo sistema già fornisce.

\section{Requisiti soddisfatti}
Sin dall'inizio, era stato previsto che la presente tesi non avrebbe portato alla realizzazione completa del sistema, ma avrebbe riguardato principalmente le fasi fondamentali di analisi e progettazione, accompagnate dall'implementazione di una prima versione funzionale. Nonostante ciò, è possibile tracciare un bilancio dei requisiti già soddisfatti, i quali evidenziano le potenzialità del nuovo sistema rispetto al precedente. A tal fine, i requisiti verranno suddivisi in due categorie: quelli già soddisfatti sia dal punto di vista progettuale che implementativo e quelli attualmente soddisfatti solo dal punto di vista progettuale.

\subsection{Elementi progettati e implementati}
Tra i requisiti soddisfatti, alcuni non si limitano alla fase di progettazione ma sono già stati concretamente implementati, costituendo il nucleo della prima versione del sistema. Questi elementi forniscono un primo miglioramento rispetto al legacy e rappresentano una base solida per l’evoluzione futura del progetto. Di seguito vengono analizzate le funzionalità già operative, evidenziandone l'impatto e le garanzie offerte.

\subsubsection{Dashboard panoramica per MSP e Dealer}
Un'altra funzionalità significativa già implementata è la dashboard panoramica, progettata specificamente per MSP e Dealer. Nel sistema legacy, questi utenti non disponevano di una visione aggregata ed efficace dei clienti sotto la loro gestione. Il nuovo sistema, invece, include una schermata con statistiche e grafici che permettono di ottenere una panoramica dettagliata delle attività gestite. In particolare, la dashboard mostra le seguenti informazioni:

\begin{itemize}
  \item Nella parte superiore, vengono presentate statistiche di base relative alle richieste elaborate dal filtro DNS. Tra queste: il numero totale di richieste ricevute, il numero di minacce bloccate, il numero di categorie e indirizzi IP bloccati, il numero di categorie consentite, il numero di richieste DNS non risolte (NXDOMAIN) e, infine, il numero di richieste che hanno forzato l’utilizzo della SafeSearch su motori di ricerca e YouTube. Questi dati offrono un quadro generale sul funzionamento del sistema e sull'efficacia del filtraggio.

  \item Nella parte centrale, sono presenti grafici per monitorare l’andamento delle richieste DNS. In particolare, vi è:
    \begin{itemize}
      \item Un grafico a barre che mostra le cinque categorie più bloccate.
      \item Un grafico a torta che rappresenta la distribuzione percentuale delle richieste DNS in base alla loro tipologia (richieste bloccate, richieste consentite, blocchi per IP, domini non risolti).
    \end{itemize}

  \item Nella parte inferiore, è presente un grafico che mostra l’andamento delle richieste DNS nel tempo, distinguendo tra quelle bloccate e quelle consentite.

  \item Tutti e tre i grafici consentono di modificare l'intervallo temporale di visualizzazione, permettendo di analizzare i dati su finestre temporali di 24, 48 o 72 ore.
\end{itemize}
%
Oltre a offrire una panoramica complessiva, la dashboard consente di visualizzare statistiche e report specifici per ciascun cliente gestito dall’MSP o dal Dealer. È infatti possibile selezionare un cliente dalla lista e ottenere un quadro dettagliato delle sue attività. Questa funzionalità risulta particolarmente utile per monitorare eventuali anomalie o problemi specifici di un singolo cliente.

Al momento, la dashboard è ancora in una fase iniziale e non è possibile esprimere garanzie sulla qualità dell'aggregazione dei dati, che verrà affinata con le iterazioni successive dello sviluppo.

\begin{figure}
  \centering
  \includegraphics[width=1\textwidth]{figures/new-dashboard.png}
  \caption{Schermata del nuovo sistema che rappresenta la dashboard MSP.}
  \label{fig:dashboard-msp}
\end{figure}

\paragraph{Struttura e layout della dashboard}
Dal punto di vista dell’interfaccia utente, come si può vedere in \Cref{fig:dashboard-msp}, la dashboard è stata progettata con una struttura chiara e organizzata per garantire un’esperienza di navigazione fluida. Il menu di navigazione, posizionato nella parte sinistra dello schermo, è sempre visibile indipendentemente dalla pagina in cui ci si trova. Nei dispositivi con display ridotto, come gli smartphone, il menu rimane nascosto per ottimizzare lo spazio e può essere aperto su richiesta dall’utente.

Nella parte superiore dell’interfaccia, ovvero nell’header, sono presenti diversi elementi chiave:
\begin{itemize}
  \item \textbf{Il logo dell'azienda}, posizionato a sinistra.
  \item \textbf{Il selettore del cliente}: consente agli MSP di filtrare i report per un cliente specifico, oppure di lasciare l’opzione predefinita per visualizzare la panoramica aggregata di tutti i clienti gestiti.
  \item \textbf{Il selettore del tema}, situato sulla destra, che permette di passare dalla modalità scura a quella chiara.
  \item \textbf{L’area utente autenticato}, da cui è possibile accedere alle impostazioni del profilo e alla funzione di logout.
\end{itemize}

In futuro, nell’header verrà integrata una sezione notifiche, accessibile tramite un’icona che indicherà la presenza di nuovi avvisi e permetterà di visualizzarli all’interno di un pannello dedicato.

Infine, il corpo centrale della dashboard ospita i grafici e le statistiche descritte in precedenza, fornendo un’analisi dettagliata del traffico DNS e della sicurezza del sistema. L’interfaccia è stata progettata per garantire flessibilità, con un layout adattivo che ottimizza la visualizzazione su schermi di diverse dimensioni.

\subsubsection{Gestione della multiutenza}
Uno dei principali requisiti che ha motivato la realizzazione del nuovo sistema è stato l’introduzione della multiutenza. Questo requisito, fatta eccezione per il sistema di autorizzazione, è stato ampiamente soddisfatto. Attualmente, il nuovo sistema consente di gestire più credenziali di accesso per una singola organizzazione, permettendo agli utenti autorizzati di accedere al pannello e visualizzare la dashboard. Inoltre, sono già state implementate le operazioni di creazione, modifica ed eliminazione degli utenti. Questo rappresenta un significativo miglioramento rispetto al sistema precedente, nel quale la gestione degli account era estremamente limitata.

Dal punto di vista delle garanzie offerte, il controllo degli accessi non è ancora stato implementato, ma la struttura del sistema è predisposta per integrarlo in futuro con meccanismi flessibili e granulari. Inoltre, la sicurezza relativa alle credenziali di accesso è stata significativamente migliorata rispetto al legacy, grazie all’utilizzo dell'algoritmo di hashing avanzato \texttt{bcrypt}. Quest’ultimo consente non solo di rendere le password crittograficamente più sicure, ma permette anche di incrementare il costo computazionale dell’hashing, rendendolo scalabile nel tempo e aumentando la resistenza ad attacchi di forza bruta.

Per la gestione della multiutenza, il sistema fornisce un’interfaccia dedicata che consente di creare, modificare ed eliminare utenti, nonché di assegnare loro un ruolo specifico all’interno dell’organizzazione. Questa funzionalità pone le basi per l’integrazione futura di un sistema avanzato di controllo degli accessi basato su ruoli e permessi granulari.

Nella \Cref{fig:user-management} è mostrata la sezione relativa alla gestione degli utenti, con le form per l'inserimento, la modifica e la rimozione degli account utente:

\begin{figure}
  \centering
  \includegraphics[width=1\textwidth]{figures/new-user-panel.png}
  \caption{Schermata del nuovo sistema per la gestione degli utenti.}
  \label{fig:user-management}
\end{figure}

\subsubsection{Supporto multilingua}
Un ulteriore requisito già soddisfatto è il supporto multilingua. Il frontend del nuovo sistema supporta attualmente due lingue, italiano e inglese. Sebbene possa sembrare un dettaglio secondario, questa funzionalità è fondamentale per un prodotto destinato a un mercato internazionale, in cui gli utenti parlano lingue diverse. Inoltre, l’architettura progettata consente di aggiungere nuove lingue in modo semplice, senza necessità di modifiche al codice sorgente, ma solo integrando nuovi file di traduzione.

\subsubsection{Gestione avanzata degli errori}
Un altro elemento progettato e implementato nel nuovo sistema è la gestione avanzata degli errori, che introduce una strategia scalabile e strutturata per il trattamento delle anomalie sia lato backend che frontend.
%
Questa architettura garantisce diversi vantaggi:
\begin{itemize}
  \item \textbf{Robustezza e controllo a tutti i livelli del sistema}: la gestione degli errori è strutturata in modo gerarchico, coprendo dalla validazione degli input nelle API fino ai problemi a livello di database, evitando che errori non gestiti possano propagarsi in modo incontrollato.
  \item \textbf{Standardizzazione e interoperabilità}: gli errori vengono generati in modo schematizzato e dinamico, garantendo una rappresentazione uniforme e facilmente interpretabile da qualsiasi componente del sistema, sia interno che esterno.
  \item \textbf{Compatibilità con il sistema multilingua}: grazie alla struttura modulare degli errori, il frontend può costruire dinamicamente messaggi testuali in più lingue, senza dipendere da stringhe fisse predefinite nel backend.
  \item \textbf{Affidabilità del codice grazie a TypeScript}: il sistema di gestione degli errori sfrutta la tipizzazione statica di TypeScript, garantendo una maggiore sicurezza nella gestione delle eccezioni e facilitando l’individuazione di problemi già in fase di sviluppo, grazie al supporto degli strumenti di analisi statica degli IDE.
\end{itemize}

Rispetto al sistema legacy, che gestiva gli errori in modo frammentato e poco strutturato, questa soluzione garantisce una maggiore affidabilità complessiva, riducendo il rischio di comportamenti inattesi e migliorando la manutenibilità del codice. Inoltre, la sua scalabilità consente di adattarlo facilmente a nuove esigenze, rappresentando un elemento chiave per la futura evoluzione del sistema.

\subsection{Elementi solo progettati}
Oltre ai requisiti già implementati, vi sono alcuni elementi che, pur non essendo ancora concretamente sviluppati, sono stati completamente progettati e definiti nei dettagli. Sebbene al momento non sia possibile valutarne l’efficacia operativa, è possibile analizzare le loro potenzialità e il valore aggiunto che apporteranno rispetto al sistema legacy.

\subsubsection{Gestione dei profili condivisi}
Uno degli elementi più significativi in questa categoria è l’introduzione dei profili condivisi. Questa novità consente di creare profili di protezione condivisibili tra più organizzazioni all'interno della stessa gerarchia. Essi possono essere visti come template di protezione, utili soprattutto per gli MSP, che potranno così offrire configurazioni standardizzate e omogenee ai loro clienti. Un altro vantaggio di questa funzionalità è la semplificazione della gestione della protezione: eventuali modifiche a un profilo condiviso vengono automaticamente propagate a tutte le organizzazioni che lo utilizzano. Questo rappresenta un miglioramento significativo rispetto al vecchio sistema, dove le configurazioni erano gestite separatamente per ogni organizzazione, con difficoltà nella loro manutenzione e uniformità.

L’introduzione dei profili condivisi permette inoltre al sistema di allinearsi con i competitor, alcuni dei quali già dispongono di questa funzionalità. Benché questa caratteristica non sia ancora implementata, la sua progettazione dettagliata consente di avere una chiara roadmap per la sua realizzazione e integrazione futura.

\subsubsection{Riprogettazione del database}
Il design di questa funzionalità rientra in un più ampio processo di rimodellazione del database, che garantisce maggiore scalabilità e coerenza dei dati. Inoltre, la nuova struttura permette l’implementazione di funzionalità avanzate che il vecchio sistema, per sua natura, non poteva supportare. Questa revisione assicura che il sistema possa evolversi senza le limitazioni strutturali della versione legacy, rendendolo più flessibile, scalabile e adatto alle esigenze future.

Per quanto la riprogettazione del database sia stata completata principalmente a livello di schema concettuale, alcune componenti essenziali risultano già integrate nel sistema. In particolare, le nuove tabelle dedicate alla gestione degli utenti e delle organizzazioni sono già operative e costituiscono la base su cui si fonda il meccanismo di multiutenza. Queste modifiche hanno consentito di superare i vincoli del vecchio modello, che non era stato concepito per supportare una gestione avanzata degli accessi e delle gerarchie organizzative.

L’adozione del nuovo schema dati non solo migliora la struttura e la leggibilità del database, ma assicura anche una maggiore coerenza e manutenibilità, agevolando l’implementazione futura di altre funzionalità chiave. Inoltre, grazie al suo design modulare e scalabile, il database può adattarsi con facilità a nuove esigenze, garantendo flessibilità operativa e una gestione più efficiente dei dati.

\section{Miglioramenti pianificati}
Malgrado il nuovo sistema abbia già introdotto numerosi miglioramenti rispetto alla versione legacy, vi sono ancora diverse aree che necessitano di completamento, così come altre funzionalità che devono essere introdotte ex novo per garantire un'esperienza più sicura, scalabile e conforme alle esigenze operative previste. Tutti i punti trattati in questa sezione rappresentano miglioramenti già pianificati o previsti nella roadmap di sviluppo e verranno implementati nelle fasi successive rispetto a questa tesi.

\subsection{Gestione avanzata dei permessi e autenticazione}
L’attuale implementazione della multiutenza consente la creazione e la gestione di account con differenti ruoli, tuttavia, il controllo degli accessi richiede ulteriori sviluppi. In particolare, è prevista l’integrazione di un sistema \textit{Role-Based Access Control} per la gestione dei ruoli utente, combinato con un livello di autorizzazione che regoli i permessi legati all’organizzazione di appartenenza e alle licenze attive.

Per garantire un sistema di permessi strutturato ed efficace, sarà necessaria un’analisi approfondita per individuare tutte le operazioni da regolamentare e codificarle in una tabella del database. Successivamente, il sistema dovrà essere integrato a tutti i livelli della piattaforma, assicurando un controllo uniforme sugli endpoint dell’API e sulle rotte del frontend.

In aggiunta, per migliorare ulteriormente la sicurezza e conformarsi agli standard più recenti, è prevista l’implementazione di un sistema di autenticazione multifattore (MFA). Questo elemento è particolarmente richiesto dai clienti operanti in settori sensibili, dove la protezione degli accessi rappresenta un requisito imprescindibile.

\subsection{Integrazione della brand identity nel frontend}
Attualmente, il frontend del sistema è stato progettato con un approccio funzionale, senza un’integrazione effettiva della nuova brand identity aziendale. È già previsto che, nelle prossime fasi, venga incorporato il contributo del designer incaricato del rinnovo dell’immagine aziendale, assicurando così un'interfaccia coerente con il design system aziendale. Sebbene questo aspetto non influisca direttamente sulle funzionalità, è fondamentale per l’accettazione del prodotto da parte degli utenti finali, che devono percepirlo come un sistema moderno, affidabile e il linea con l’immagine aziendale.

\subsection{Implementazione della nuova struttura del database}
La rimodellazione del database è stata completata a livello concettuale, ma la sua implementazione effettiva è ancora in fase iniziale. Ad oggi, solo le tabelle relative alla gestione della multiutenza sono operative, mentre il resto della struttura non è stato ancora tradotto in un’implementazione concreta all’interno di un database di test.

Un altro aspetto chiave previsto nella roadmap è la definizione di una procedura di migrazione dei dati dal vecchio sistema alla nuova base dati. Questo processo permetterà di validare le nuove strutture, assicurando la loro coerenza e integrità sin dalle prime fasi dello sviluppo, consentendo al contempo di continuare il lavoro sulla versione migliorata del database.

\subsection{Aumento della copertura dei test}
Attualmente, l’infrastruttura di testing del frontend è stata predisposta e resa operativa, ma non dispone ancora di specifiche definite per garantire una copertura adeguata. Tuttavia, la \textit{coverage} al 100\% del codice sorgente rappresenta un requisito fondamentale per garantire la stabilità e l’affidabilità del sistema in quanto contribuisce a individuare e correggere eventuali anomalie prima del rilascio in produzione

Per soddisfare questo requisito, è necessario incrementare la copertura dei test automatizzati, con particolare attenzione ai componenti critici dell’interfaccia utente. Questo aspetto è già stato pianificato e verrà affrontato nelle prossime iterazioni dello sviluppo, al fine di consolidare la qualità del software e ridurre il rischio di regressioni.

\subsection{Implementazione delle funzionalità chiave}
Oltre ai miglioramenti tecnici e architetturali già discussi, il completamento della prima release del sistema richiede l’integrazione di alcune macrofunzionalità fondamentali per garantirne la piena operatività. In particolare, il sistema dovrà includere:
\begin{itemize}
  \item \textbf{Gestione della protezione}: interfacce e strumenti dedicati alla configurazione delle policy di filtraggio e sicurezza, con un'efficace integrazione dei profili condivisi.
  \item \textbf{Generazione e visualizzazione avanzata dei report}: estensione del modulo di reporting per offrire un’analisi più dettagliata rispetto ai dati attualmente disponibili nella dashboard panoramica, così come la possibilità di esportare i report in formati standard.
  \item \textbf{Analisi del traffico in tempo reale}: strumenti per il monitoraggio live delle richieste DNS, utili per una gestione più dinamica della sicurezza di rete. Dovrà essere inclusa una funzione di ricerca avanzata per filtrare e analizzare i dati in tempo reale.
  \item \textbf{Gestione degli Endpoint remoti}: integrazione del sistema con i dispositivi mobili dotati di \textit{ClientShield}, per consentirne la configurazione e il monitoraggio da remoto.
\end{itemize}

Attualmente, l’unico modulo già presente riguarda la visualizzazione di alcuni report nella dashboard panoramica, sebbene in forma ancora sommaria e non esaustiva. Tuttavia, poiché il nuovo sistema è destinato a sostituire integralmente il precedente, esso dovrà necessariamente integrare fin dalla prima release tutte le macrofunzionalità operative nella versione legacy, garantendo continuità nell’erogazione dei servizi e nelle capacità di gestione della protezione e del traffico DNS. L’implementazione di questi moduli non rappresenta dunque un semplice miglioramento, ma un requisito imprescindibile per assicurare una transizione senza compromessi tra il vecchio e il nuovo sistema.

\chapter{Conclusione}

Il presente lavoro ha avuto come obiettivo la reingegnerizzazione del pannello di configurazione di un sistema legacy per il filtraggio DNS, progettando e implementando un'infrastruttura moderna e scalabile in grado di superare le limitazioni della versione precedente. Il progetto si è focalizzato sull’analisi e la riprogettazione dell’architettura, gettando le basi per una transizione completa verso un sistema più efficiente e flessibile. Questo capitolo riepiloga i principali risultati ottenuti, le sfide affrontate e gli sviluppi futuri previsti per il completamento e l’evoluzione del sistema.

\section{Sintesi del lavoro svolto e lezioni apprese}
Lo sviluppo di un sistema software segue tipicamente un processo articolato in più fasi, tra cui la raccolta dei requisiti, l’analisi e progettazione, l’implementazione e il collaudo. La presente tesi è intervenuta dopo la fase di raccolta dei requisiti, concentrandosi sull'analisi e la progettazione di un sottoinsieme di caratteristiche chiave della nuova piattaforma, nonché sulla loro prima implementazione. L’obiettivo principale è stato quello di costruire una base solida su cui fondare l’evoluzione del sistema, risolvendo le principali criticità della versione legacy e introducendo soluzioni volte a migliorare scalabilità, sicurezza e manutenibilità.
%
In particolare, le milestone principali raggiunte includono:

\begin{itemize}
  \item \textbf{Multiutenza e gestione degli accessi}: una delle caratteristiche chiave del nuovo sistema rispetto al legacy è la possibilità di supportare un modello multiutente, permettendo la gestione di account con permessi differenziati. Attualmente, il sistema di autenticazione è stato implementato con successo e consente un accesso sicuro alla piattaforma. Tuttavia, il livello di autorizzazione, essenziale per regolare con precisione i permessi di ciascun utente in base al ruolo e all’organizzazione di appartenenza, è ancora in fase di definizione. La sua integrazione futura consentirà di sfruttare appieno il potenziale della multiutenza, offrendo una gestione più strutturata degli accessi e migliorando la sicurezza del sistema.

  \item \textbf{Dashboard panoramica per MSP}: l’introduzione di una dashboard dedicata ai Managed Service Provider ha reso possibile una visione aggregata delle attività dei clienti, migliorando il controllo e la gestione della protezione DNS. Questo strumento fornisce ai MSP dati centralizzati e statistiche dettagliate, semplificando il monitoraggio delle configurazioni e delle minacce bloccate. Oltre a migliorare l’esperienza degli amministratori, questa funzionalità posiziona strategicamente il prodotto in un mercato più orientato al settore enterprise, che rappresenta uno degli obiettivi primari dell’azienda in termini di crescita e competitività.

  \item \textbf{Nuova architettura del database}: la rimodellazione della base dati ha consentito di superare le rigidità della vecchia struttura, garantendo maggiore coerenza e supporto nativo alle nuove funzionalità. Oltre a risolvere le limitazioni del database legacy, la nuova architettura è stata progettata seguendo le migliori pratiche nel disegno concettuale delle strutture dati. Sono stati adottati nomi significativi per le entità, così come tipi di dato più coerenti e precisi, il tutto per evitare ambiguità e migliorare l'integrità del database. Inoltre, la suddivisione delle tabelle è stata ottimizzata per garantire un'organizzazione più chiara e una gestione più efficiente delle relazioni tra i dati, ponendo le basi per un sistema scalabile e facilmente estendibile.

  \item \textbf{Gestione avanzata e strutturata degli errori}: è stato sviluppato un sistema innovativo per la gestione degli errori, progettato per essere scalabile, facilmente integrabile tra frontend e backend e compatibile con la gestione multilingua. Questo sistema è pervasivo, operando a tutti i livelli del backend del sistema, dalla validazione degli input nelle API alla gestione delle eccezioni interne e degli errori a livello di database. La sua implementazione sfrutta la tipizzazione avanzata del linguaggio di programmazione utilizzato, garantendo un controllo rigoroso e una maggiore robustezza nel trattamento degli errori. Inoltre, l’approccio scelto lo rende poco invasivo dal punto di vista implementativo, permettendo agli sviluppatori di utilizzarlo in modo efficace senza introdurre complessità eccessive nel codice esistente. Questo aspetto ha rappresentato uno dei contributi più significativi apportati al progetto, migliorando la manutenibilità e l'affidabilità dell’intero sistema.
\end{itemize}

Durante lo sviluppo del sistema sono emerse diverse criticità, che hanno richiesto un'analisi approfondita e un approccio metodologico mirato. Tra le principali difficoltà affrontate vi sono:

\begin{itemize}
  \item \textbf{Apprendimento di nuove tecnologie}: la realizzazione del nuovo sistema ha richiesto l’utilizzo di strumenti e paradigmi non presenti nella versione legacy, rendendo necessaria un'iniziale fase di formazione e consolidamento delle conoscenze.
  \item \textbf{Rimodellazione del database}: la definizione di una nuova architettura dati ha richiesto un'attenta analisi delle tabelle esistenti e dei flussi informativi, con l'obiettivo di consolidare i dati e garantire una transizione strutturata.
  \item \textbf{Strutturazione avanzata degli errori}: la principale difficoltà è stata quella di ideare un meccanismo che fosse al contempo semplice ma efficace, in grado di strutturare gli errori in modo uniforme, garantendo compatibilità tra frontend e backend e supporto alla multilingua.
\end{itemize}

Dall’esperienza maturata in questo progetto sono emerse alcune scelte architetturali particolarmente efficaci, che si sono rivelate strategiche per la solidità del nuovo sistema:

\begin{itemize}
  \item \textbf{Rinnovo del database fin dalle prime fasi di sviluppo}: questo approccio ha garantito la costruzione di un sistema privo delle inefficienze della versione legacy, con una base dati progettata su misura per supportare le nuove funzionalità senza compromessi.
  \item \textbf{Adozione di un'architettura a microservizi}: questa scelta ha permesso di garantire modularità, scalabilità (anche geografica) e una più semplice manutenzione del sistema, rendendolo adattabile alle esigenze future.
\end{itemize}

\section{Evoluzioni e prospettive future}
Il lavoro svolto finora rappresenta solo il primo passo verso la realizzazione di un sistema completo e pienamente operativo. Per garantire una transizione efficace e una piena sostituzione del sistema legacy, sono previsti ulteriori sviluppi nelle prossime fasi del progetto. Tra le principali evoluzioni future si evidenziano:

\begin{itemize}
  \item \textbf{Gestione avanzata della multiutenza}: l’introduzione di un sistema di delega dei permessi e configurazioni personalizzate per MSP e Dealer consentirà una gestione più granulare degli accessi e delle operazioni eseguibili dagli utenti.
  \item \textbf{Integrazione con servizi esterni}: un passo importante per migliorare le capacità del sistema sarà l'integrazione con provider esterni, sul modello di soluzioni adottate da competitor come DNSFilter.
  \item \textbf{Implementazione di un sistema di autenticazione centralizzato}: la possibilità di autenticarsi una sola volta per accedere a tutti i servizi aziendali migliorerà l'esperienza utente e rafforzerà la sicurezza del sistema.
  \item \textbf{Allineamento della specifica API con le effettive rotte backend}: un sistema per garantire la coerenza tra API documentation e funzionalità implementate faciliterà lo sviluppo e la manutenzione del sistema.
\end{itemize}

L’insieme di questi sviluppi contribuirà a rendere il nuovo sistema una piattaforma completa, performante e in grado di adattarsi alle esigenze operative degli utenti finali, assicurando al contempo maggiore sicurezza e scalabilità. La roadmap futura punta non solo a colmare le funzionalità mancanti, ma anche a rendere il sistema più competitivo e innovativo rispetto alle soluzioni attuali presenti sul mercato.


%----------------------------------------------------------------------------------------
% BIBLIOGRAPHY
%----------------------------------------------------------------------------------------

\backmatter

\nocite{*}

\bibliographystyle{alpha}
\bibliography{bibliography}

% \begin{acknowledgements} % this is optional
% Optional. Max 1 page.
% \end{acknowledgements}

\end{document}
