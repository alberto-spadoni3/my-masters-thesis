\begin{acknowledgements}
  Giungere a scrivere questi ringraziamenti è un momento di grande emozione. Questo traguardo non rappresenta solo il conseguimento della laurea, ma la vittoria su difficoltà, insicurezze e momenti di crisi. Significa aver superato esami che sembravano insormontabili e aver portato a termine la stesura di questa tesi, che ho sempre temuto e rimandato. Ora posso dire di avercela fatta e di esserne orgoglioso!

  Un ringraziamento speciale alla mia famiglia: grazie \emph{mamma} e \emph{babbo} per avermi sempre sostenuto e per aver reso più leggeri gli anni di studio. Grazie a mio fratello \emph{Samuele} per il supporto e per quelle conversazioni che solo con te posso avere. Un pensiero speciale va alla mia amata \emph{Lallina}, che mi è stata accanto con affetto e incoraggiamento nei momenti più difficili di questo percorso.

  Un sentito grazie anche agli amici e ai compagni di corso che hanno contribuito a rendere questo viaggio più stimolante e ricco di momenti indimenticabili. In particolare, grazie \emph{Edoardo} e \emph{Giorgio} per il vostro supporto e la vostra amicizia, \emph{Andrea}, \emph{Gianmarco} e \emph{Thomas} per essere sempre presenti fin dai tempi delle scuole medie. Infine, grazie ai compagni con cui ho affrontato la maggior parte degli esami: \emph{Lorenzo}, \emph{Alex}, \emph{Andrea}, \emph{Marco}, \emph{Fabio} e \emph{Michele}.

  Desidero esprimere la mia profonda gratitudine al mio relatore, il professor \emph{Mirko Viroli}, e ai correlatori, \emph{Nicolas Farabegoli} (che da compagno di corso è passato a essere mio correlatore) e \emph{Gianluca Aguzzi}, per il loro prezioso supporto e la loro competenza. Il loro contributo è stato essenziale non solo per il perfezionamento di questa tesi, ma anche per lo sviluppo del sistema che ne è oggetto.

  Un ringraziamento speciale va, infine, all'azienda che mi ha ospitato per il tirocinio, in particolare al suo CEO, \emph{Francesco}, e a tutti i colleghi. Senza di loro non avrei avuto l'opportunità di partecipare a un progetto di tale portata, né di crescere professionalmente come ho fatto in questi mesi.

  Questo traguardo segna la fine di un lungo percorso e l’inizio di nuove sfide. Ora posso finalmente guardare al futuro con il titolo di \textit{Ingegnere}!
\end{acknowledgements}
